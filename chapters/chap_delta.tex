
%%%%% BOARD DELTA

\label[chap:boardDelta]
\chap Board Delta

\quad In this chapter, we address the Board Delta - a double redundant \glref{OBC} module in the PC/104 format. We explain the motivation and expectation behind the redundant systems while providing description and comments on the schematic and \glref{PCB} module design. A 3D visualization of the Board Delta is shown in figure \ref[img:delta_miniature].

\midinsert
    \clabel[img:delta_miniature]{Board Delta 3D render}
    \picw=0.95\hsize \cinspic figures/model/delta_miniature.pdf
    \caption/f 3D render of the Board Delta from the top (left) and the bottom (right) side.
\endinsert



%%%%% MOTIVATION

\sec Motivation and expectation

\quad During their launch and operation, Cubesats have to perform under extreme conditions of the surrounding environment. The most critical impacts were briefly described in the previous chapters, namely: i) damage of semiconductor devices caused by radiation (section \ref[chap:radiationAndRedundancy]), ii) exposure to wide temperature gradients and problems related to limited heat dissipation, and iii) significant mechanical stress generated by a launcher vehicle (both in section \ref[chap:componentsCertification]). Together with the practically non-existing possibility of maintenance, these problems are the key factors of a high failure rate of the already executed missions. As we have listed in section \ref[chap:existingOBCmodules], every second launched Cubesat has experienced a fatal failure by 2016, from which 20\% were caused by the \glref{OBC}s.

Designing the particular modules with multiple layers of redundancy is a common method of increasing the spacecraft's overall reliability \cite[pap:failRate2016]. \"By implementing onboard redundancy, CubeSats can meet or exceed their mission life, providing additional science data and post-mission payload testing." \cite[pap:approachToSpace] Together with telecommunications, attitude determination, and electrical power, the \glref{OBC} is one of the critical modules with high requirements for redundancy \cite[pap:satelliteLeo]. Some of the professional \glref{OBC} manufacturers have also included elements of redundancy into their designs \cite[obc:dataPatterns, obc:isis, obc:imt, obc:gauss, obc:nanosatpro]. Interestingly, the fairly popular CubeSat \glref{OBC} Kit from Pumpkin \cite[obc:pumpkin] is not one of them and its non-redundant architecture was pointed out as one of the two major weaknesses \cite[pap:reliability].

Although the Board Sierra implements some features of redundancy (triple external memories and dual peripheral data buses), it is still a single \glref{OBC} module. In a case of permanent damage to the power management or the \glref{MCU}, the Cubesat's mission would be severely jeopardized. However, this scenario can be eliminated by including another \glref{OBC} into the module design. Each of these two \glref{OBC}s is fully independent and has its own circuitry. Such a double redundant \glref{OBC} module can simply switch between the two \glref{OBC}s if one of them undergoes a serious malfunction. This approach was, for example, implemented in the professional DP-OBC-0402 by Data Patterns \cite[obc:dataPatterns].



%%%% DOUBLE REDUNDANT DESIGN

\sec Double redundant design

\quad From the beginning of the VST104 project, we thought about the Board Delta as a hardware merge of the two Board Sierras. The intention was to share the same \glref{OBC} design between the two redundant \glref{OBC}s at the Board Delta and also with the \glref{OBC} on the Board Sierra. This approach of the only one \glref{OBC} design has several advantages. Firstly, the two identical \glref{OBC}s on the Board Delta can run almost the same software and provide the same functionality to the spacecraft. After an emergency switch to the other \glref{OBC}, the Cubesat can continue to operate normally without changing the mission plan. Secondly, it allows an easy project migration between the boards. A potential user can develop and test his/her setup on the cheaper Board Sierra and then move to the more expensive Board Delta. Lastly, the approach also simplifies the development process and future maintenance of the VST104 project. It is faster and cheaper to develop and test out only one \glref{OBC} design and then integrate it into different modules. Also, if an \glref{OBC} design flaw or a possible improvement is found on one of the modules, it can be automatically implemented to the remaining modules.



%%%%% SCHEMATIC DESIGN

\secc Schematic design

\quad As the idea was to use the already existing \glref{OBC} design (described in chapters \ref[chap:boardSierra] and \ref[chap:boardSierraSubsystems]), the creation of the Board Delta schematic was straightforward. The KiCad project of the Board Delta consists of one primary sheet and two sub-sheets. The primary sheet contains the PC/104 header with assigned global signals, while each sub-sheet includes one \glref{OBC} design copied from the Board Sierra. Power and peripheral input/outputs of these so-called Left and Right \glref{OBC}s are attached to the same global signals as the PC/104 header. The resulting configuration of the module is simple: peripheral isolators and inputs of the power managements are connected directly to the PC/104 header. The individual peripheral isolators and Kill Switch function of the power management are then used to isolate and power down one of the \glref{OBC}s.

Although we have stated multiple times that the \glref{OBC}s share the same design, changes to a few PC/104 header pin assignments were required. The watchdog signal {\tt CPU_WD_1} and the Kill Switch signal {\tt GLO_KS_1} were assigned to the Left \glref{OBC}, whereas their {\tt CPU_WD_2} and {\tt GLO_KS_2} variants were assigned to the Right \glref{OBC}.

It is also important to note that our design does not exclude the possibility of running both of the \glref{OBC}s simultaneously. In such a scenario, both \glref{OBC}s should be powered on and negotiate the use of the shared PC/104 busses. The \glref{OBC} not using a particular bus should isolate itself from it. The simultaneous operation can be beneficial in missions where significant amounts of data need to be processed quickly. Also, some Cubesats may require one primary \glref{OBC}, handling the housekeeping and mission control, together with a secondary \glref{OBC}, serving payloads such as scientific instruments or cameras.



%%%%% PCB DESIGN

\secc PCB design

\quad The decision to use only one \glref{OBC} design does not apply exclusively to the schematic design but also to the \glref{PCB} design. The same (or as similar as possible) \glref{PCB} layout and routing of the \glref{OBC}s would guarantee comparable mechanical, electrical, and thermal characteristics. As the desire to create the Board Delta was clear from the beginning, we count it into the design of the Board Sierra \glref{PCB}. Its \glref{OBC} was designed exclusively on the PC/104 module northwest quarter, leaving the northeast quarter empty.

We used several tricks to ensure the best possible similarity between the \glref{OBC}s layouts. As a building template of the Board Delta \glref{PCB}, we copied the final design of the Board Sierra. The \glref{OBC} design on the template was assigned to the Left \glref{OBC}. The layout for the Right \glref{OBC} was generated using the Replicate layout plugin (mentioned in section \ref[chap:kicadPlugins]). This plugin was capable of copying the entire \glref{OBC} design, including all footprints, traces, and vias. However, to match the alignment and pinout of the PC/104 header, we had to flip this copied layout. This changed the previous top and bottom sides of the \glref{OBC} and is why the \glref{MCU}s are each on a different \glref{PCB} side. 

Some manual adjustments and routing were although required to finish the \glref{PCB} design. The second and third inner copper layers were manually reversed back to meet the copper layers assignment from section \ref[chap:pcbCharacteristics]. Also, the debug connector and its circuitry were mirrored back and rerouted. Thanks to this adjustment, both debug connectors are located on the \glref{PCB} top side with the same orientation and pinout. However, the most challenging part was to manually trace all peripheral isolators to their required PC/104 pins. Multiple attempts and sacrifice of the northern edge cutout were required to accomplish so. The complex layout and routing of both \glref{OBC}s are shown in figure \ref[img:delta_kicad_cut]. The Left \glref{OBC} is located on the left side (northwest quarter), while the Right \glref{OBC} is located on the right side (northeast quarter) of the Board Delta.

\midinsert
    \clabel[img:delta_kicad_cut]{Board Delta KiCad}
    \picw=1\hsize \cinspic figures/pcbs/delta_kicad_cut.pdf
    \caption/f \glref{PCB} design of the Board Delta captured directly in the KiCad environment. Visible is only the upper part with the PC/104 header and both \glref{OBC}s.
\endinsert
