
\chap Board Sierra - description

%%%%% CUBESATS %%%%%%%%%%%%%%%%%%%%%%%%%%%%%%%%%%%%%%%%%%%%%%%%%%%%%%%%%%%%%%%%%%%%%%%%%%%%%%%%%%%%%%%%

\sec CubeSat concept

%%%%% OBC REQUIREMENTS %%%%%%%%%%%%%%%%%%%%%%%%%%%%%%%%%%%%%%%%%%%%%%%%%%%%%%%%%%%%%%%%%%%%%%%%%%%%%%%%

\sec OBC requirements

%%%%% RADIATION

\secc Radiation and redundancy

\quad Occasionally traveling through weaker parts of the Earth's magnetic field and not shielded by the Earth's atmosphere, the CubeSats have to operate in an environment full of radiation. A direct hit of a high-energy particle might have serious consequences for \glref{OBC} functionality. These include transistor gate ruptures, memory bit flips, software upsets, or latch-ups. A proper strategy must be taken to increase the \glref{OBC} durability and ability to handle such an error, resulting in maximizing a possible mission life.

One approach is to use special radiation-hardened components. This strategy is typical for professional and more expensive satellites than the CubeSats. These components are usually bigger, more costly, and have less functionality than ordinary COTS.

Considering the size and budget requirements of our \glref{OBC}, we chose to implement another option. Instead of the previously mentioned physical hardening technique, a logical one was realized. The \glref{OBC} hosts many schematic design-related features ensuring the proper handling of any radiation-related event. These include: i) over-current sensing power management, ii) separate peripheral isolators, iii) full high-impedance mode requested by the higher logic, iv) triple-redundant memories, v) multiple temperature sensors. A fully double-redundant \glref{OBC} is presented in chapter \ref[chap3:delta].

%%%%% FEATURES

\label[chap2:features]
\secc Capabilities and features

\quad Having in mind the expectations of the \glref{VST} supervisors, features common for \glref{OBC}s by different professional manufacturers, and design requirements implied by the radiation, we created and implemented the following list of desired \glref{OBC}'s features:

\begitems
    * {\sbf Microcontroller:} 
    * {\sbf External clock sources:} The \glref{MCU} has two internal resistor-capacitor (\glref{RC}) oscillators that can be used to drive a master system clock and auxiliary clocks \cite[dat:mcu]. These internal oscillators have a significantly lower frequency stability and a higher temperature dependency than the external ones \cite[app:oscInt]. Therefore, to ensure the clock reliability in the harsh space environment, we had to implement external clock sources.
    * {\sbf Robust power management:} The electric power subsystem (\glref{EPS}) is known to be the most vital subsystem of a spacecraft. Its reliability and error handling should be ensured by the power control and distribution unit (\glref{PDCU}). However, it is a good engineering practice by professional \glref{OBC} manufacturers to include an additional power control to their designs \cite[obc:dataPatterns, obc:isis, obc:imt, obc:pumpkin, obc:gauss, obc:aac, obc:anteleope, obc:nanosatpro, obc:gos]. Our \glref{OBC} requires $3.3\rm{[V]}$ and $5\rm{[V]}$ power inputs from the main power buses. In the case of their malfunction, it is our responsibility to sense it and power down the \glref{OBC}. This feature is also important for some of the \glref{OBC} on-earth user cases. During a hardware development or a system presentation, the user might misconnect the power line or use an unsupported power source.
    * {\sbf Isolation of the peripherals:} The spacecraft \glref{OBC} is connected to many data buses shared among all other subsystems. In some scenarios, the \glref{OBC} must be able to isolate itself from a specific or multiple data buses. For example, to switch between \glref{OBC}s in a redundant configuration, handle the failure on a data bus, or prevent unintentional interferences. Standard approaches to addressing this isolation feature are analog switches \cite[the:wurzburg], optocouplers \cite[pap:satelliteLeo], or \glref{FPGA}s \cite[obc:dataPatterns]. These isolators should also guarantee that all data lines are high impedance when the \glref{OBC} is powered off.
    * {\sbf Redundant external memory:}
    * {\sbf CAN bus peripherals:}
    * {\sbf Temperature monitoring:}
    * {\sbf Maximal payload sector:}
\enditems

%%%%% CERTIFICATION

\secc Components certification

%%%%% SELECTION

\label[chap:componentsSelection]
\secc Components selection 

\quad After listing all of the technical requirements for a specific electronic part, we had to chose a particular component. As there are usually multiple similar components from various manufacturers, we had to use additional criteriums for the selection process.

As this \glref{OBC} module is not primarily designed for an actual space flight (explained in chapter \ref[chap1:introduction]), an individual component's price is not negligible. With a lower overall cost of the \glref{OBC}, a broader project expansion in the LibreCube community can be achieved. Thus components with sufficient attributes but lower price were favored.

Another criterium in the component selection was the available PCB surface. As a result of maximizing the payload sector of the PC104 module, the actual \glref{OBC} subsystem area was significantly decreased. Therefore the components of smaller dimensions available in fine-pitch packages (e.g. \glref{SSOP} or \glref{QFN}) were preferred. After researching the technical capabilities and related costs of PCB manufacturers, we decided not to use the \glref{BGA} package components. Their significantly small footprints would require more precise fanout, resulting in increased manufacturing difficulty and price.

Different components are available from various distributors, which can prolong the assembling process. We chose to use Mouser Electronics for purchasing the components. Therefore the availability of a specific component in this store was also a selection factor. This decision was influenced by the \glref{VST} supervisors and previous experiences.

%%%%% PCI104 %%%%%%%%%%%%%%%%%%%%%%%%%%%%%%%%%%%%%%%%%%%%%%%%%%%%%%%%%%%%%%%%%%%%%%%%%%%%%%%%%%%%%%%%%%

\sec PCI104 standard

%%%%% MECHANICAL

\secc Mechanical specification

%%%%% PINOUT

\secc Main header pinout

\sec PCB design and assembly
\secc PCB specifications
\secc Routing and fanout


