
%%%%% CONCLUSION

\chap Conclusion

\quad In this thesis, we have contributed to the VST104 project of CubeSat hardware development established by the company VisionSpace Technologies. For this project, we have designed a family of PC/104 format electronics boards, including a universal board, a single onboard computer board, a double redundant onboard computer board, and a FlatSat test bench. We have also managed to properly test out our primary design - the single onboard computer. On top of that, we have successfully conducted its testing under a gamma radiation source and acquire valuable experiment results. In order to provide helpful and accurate documentation of the developed boards, we have included many design details and illustrations into the main body of this thesis.

It is safe to say that our work has already been noticed by the open-source CubeSat community. We have presented our work at an Open Source CubeSat Workshop 2020, igniting a broad discussion about a united PC/104 pinout. At the time of submitting this thesis, an abstract including the VST104 project was accepted for the 5th \glref{ESA} CubeSat Industry Days. Similarly, another abstract regarding the radiation testing was accepted for the 1st Students Conference on Sensors, Systems and Measurement.

We are also pleased that our work is already being used not only by the \glref{VST} but also by other organizations. Romanian InSpace Engineering S.R.L. started the development of CubeSat subsystems based on the provided Board Sierra and Element Foxtrot. TU Darmstadt Space Technology e.V. has also received both of these boards for their development purposes. The \glref{VST} is currently using the VST104 platform for developing their Rust implementation of the telemetry and telecommand packet utilization. Future expansion of our work is planned as \glref{VST} is interested in integrating a NanoXplore NG-Medium \glref{FPGA} to the Board Sierra for the development of the POCKET+.
