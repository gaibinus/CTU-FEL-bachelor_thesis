
\chap Board Sierra - single OBC
\sec Submodules and circuit design
\secc Microcontroller

\secc External clock sources
\quad The \glref{MCU} has two internal \glref{RC} oscillators that can be used to drive the system and auxiliary clocks \cite[dat:mcu]. These internal oscillators have a lower frequency stability and a higher temperature dependency than the external ones \cite[app:oscInt]. To ensure clock reliability in the harsh space environment, we had to implement external clock sources.

A $4-48\rm{[MHz]}$ \glref{HSE} oscillator can drive the system clock. Supported types are crystal, ceramic resonator, or silicone oscillator \cite[dat:mcu]. The last option seems to be the best as it is insensitive to \glref{EMI} and vibration. The only downside is its slightly lower temperature rejection \cite[app:oscComp]. We chose the SiT8924B \cite[dat:hse], a $26\rm{[MHz]}$ silicon \glref{MEMS} oscillator. Accordingly to the clock configuration tool of the stm32cube software, we can reach various system clock frequencies up to $78\rm{[MHz]}$ (the max. is $80\rm{[MHz]}$ \cite[dat:mcu]). The circuitry follows the \glref{HSE} datasheet \cite[dat:hse]  and is shown on the right side of figure \ref[fig:clockSource]. The \glref{HSE} output OSE_IN can be enabled or disabled by the binary OSC_EN signal.

A $32.768\rm{[kHz]}$ \glref{LSE} oscillator can drive the \glref{RTC}, hardware auto calibration, or other timing functions. Table 7 in \cite[app:oscDes] recommends individual crystal resonators for this specific purpose with STM32 \glref{MCU}s. After a consideration of these options, we decided to use the ABS07AIG ceramic base crystal. The \glref{LSE} oscillator circuitry is based on the reference design in figure 5 in \cite[app:oscDes]. To achieve a stable frequency of this so-called Pierce oscillator, it is required to determine the values of two load capacitors $C_{L1}$, $C_{L2}$ and an external resistor $R_{Ext}$. This can be obtained using the equations

$$ C_L = {{C_{L1} C_{L2}}\over{C_{L1}} + C_{L2}} + C_S \quad \wedge \quad C_{L1} = C_{L2}, \eqmark $$

$$ R_{Ext} = {1\over{2\pi f C_{L2}}}, \eqmark $$

where $C_L$ is the crystal load capacitance, $f$ is the crystal oscillation frequency and $C_S$ is the stray capacitance \cite[app:oscDes]. Values of $C_L$ and $f$ are listed in the crystal datasheet \cite[dat:lse]. We can assume as a rule of thumb, that $C_S=4\rm{[pF]}$. The final \glref{LSE} circuitry with computed values of the components is shown on the left side of figure \ref[fig:clockSource]. 

\midinsert
    \label[fig:clockSource]
    \picw=0.6\hsize \cinspic figures/schemes/clockSources.pdf
    \caption/f Schematic diagram of external clock sources.
\endinsert

\secc Peripheral isolators
\secc Power management
\secc External memory
\secc CAN bus drivers
\secc Temperature sensing
\sec PCB design and assembly
\secc PCB specifications
\secc Submodles placement
\secc Routing and fanout
\secc Assembly and debug
\sec Board Delta - double OBC
\secc Circuit modification
\secc PCB modification
