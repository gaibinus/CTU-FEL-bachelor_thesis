
%%%%% BOARD SIERRA

\chap Board Sierra

\quad In this chapter, we are addressing the design process and overall characteristics of the Board Sierra. Design challenges, development decisions, and final features are explained in detail. Particular attention is given to the onboard computer, from now referred to as the \glref{OBC}. A detailed description of its subsystems is listed in chapter \ref[chap:BoardSierraSubmodules].




%%%%% OBC CHARACTERISTICS

\sec OBC characteristics

\quad An OBC of a CubeSat is a complex subsystem combining various features. Reviewing technical documentation of already existing \glref{OBC}s (chapter \ref[chap:existingOBCmodules]) showed that some features are shared among most \glref{OBC}s, whereas some are unique. In this section, we list all of the features we decided to implement, both typical and VST104 related.



%%%%% RADIATION

\label[chap:radiationAndRedundancy]
\secc Radiation and redundancy

\quad Occasionally traveling through weaker parts of the Earth's magnetic field and not shielded by the Earth's atmosphere, the CubeSats have to operate in an environment full of radiation. A direct hit of a high-energy particle might have serious consequences for \glref{OBC} functionality. These include transistor gate ruptures, memory bit flips, software upsets, or latch-ups \cite[pap:radiationSurvey, app:badBoys, pap:memoryDamage]. Proper measures have to be taken to increase the \glref{OBC} durability and ability to handle errors, resulting in maximizing a mission lifetime. This process is known as radiation hardening and can be achieved by two different techniques.

A physical approach, also known as a radiation hardening by design, relies on special radiation-hardened components. Hardened \glref{IC}s are manufactured on insulating substrates (rather than typical semiconductor wafers) or use some of the many dedicated design principles \cite[pap:radiationSurvey]. Implementation of this strategy is typical for professional and more expensive satellites than the CubeSats. These components are usually bigger, costly, and have reduced performance and functionality compared to ordinary ones.

Considering the size and budget requirements of our \glref{OBC}, we chose to implement another option. Instead of the previously mentioned physical hardening technique, a logical one was realized. The \glref{OBC} hosts multiple schematic design features ensuring the proper handling of any radiation-related event. These include: i) over-current sensing power management, ii) separate peripheral isolators, iii) full high-impedance mode requested by the higher logic, iv) triple-redundant memories, or v) multiple onboard temperature sensors. A fully double-redundant \glref{OBC} is presented in chapter \ref[chap:boardDelta].



%%%%% FEATURES

\label[chap:capabilitesAndFeatures]
\secc Capabilities and features

\quad Having in mind the expectations of the \glref{VST} supervisors, features common for \glref{OBC}s by different manufacturers (chapter \ref[chap:existingOBCmodules]), and design requirements implied by the radiation (chapter \ref[chap:radiationAndRedundancy]), we set and implemented the following \glref{OBC}'s features:

\begitems
    * {\sbf Microcontroller:} An ultra-low-power STM32L496 ARM Cortex-M4 core microcontroller drives the \glref{OBC}. This \glref{MCU} can run up to $78\rm{[MHz]}$, is equipped with a $1\rm{[MB]}$ flash memory and $320\rm{[kB]}$ \glref{RAM}, and provides a wide external connectivity. A user can program and debug this \glref{MCU} using a separate \glref{SWI} or \glref{UART} connection.
    * {\sbf External clock sources:} The \glref{MCU} is equipped with two external clock sources for improved and more reliable clock generation and timing. Low-speed external $32.768\rm{[kHz]}$ oscillator and high-speed external $26\rm{[MHz]}$ oscillator. This oscillator can be temporarily disabled by the \glref{MCU} to achieve better power consumption.
    * {\sbf Robust power management:} Two separate power lines with $3.3\rm{[V]}$ and $5\rm{[V]}$ ratings supply the \glref{OBC}. A robust power management circuitry is present separately on these lines, providing tunable over and under-voltage protection, over-current protection, amplified current monitoring output, kill switching, and simultaneous power down.
    * {\sbf Isolation of the peripherals:} The \glref{OBC} can communicate with other CubeSat modules using a 2x \glref{CAN} bus, 3x \glref{$\rm{I^2C}$}, 4x \glref{UART}, 2x \glref{SPI}, 22 \glref{GPIO} pins, and 5 maintenance signals. All the peripherals are $5\rm{[V]}$ tolerant and connected to the PC104 main header thru separate analog switches. These isolators are independently controlled by the \glref{MCU} and provide a complete or selected disconnection from the rest of the CubeSat.
    * {\sbf Redundant external memory:} Two subsystems of triple redundant external memories are available for \glref{MCU}'s additional data storage. This triple redundancy ensures data validity in the radiation-rich environment, although the single-mode is also supported. The first memory subsystem is a 3x $32\rm{[MB]}$ Flash connected via the Quad-\glref{SPI} interface, while the second one is a 3x $2\rm{[Mb]}$ F-ram using the \glref{SPI}.
    * {\sbf CAN bus peripherals:} Two \glref{CAN} transceivers provide the \glref{OBC}'s support of a high-speed \glref{CAN} bus.  These transceivers ensure a voltage level conversion ($3.3\rm{[V]}$ for the \glref{MCU}, $5\rm{[V]}$ for the CubeSat) and support communication speeds up to $250\rm{[kB\,s^{-1}]}$.
    * {\sbf Temperature monitoring:} Seven \glref{$\rm{I^2C}$} temperature sensors are spread over the entire \glref{OBC}. These sensors monitor the temperature of essential submodules, 2x \glref{MCU}, 2x power management, 2x \glref{CAN} bus drivers, and 1x external memories. The \glref{MCU} can power on or off these sensors to save some power or resolve a latch-up even.
    * {\sbf Maximal payload sector:} The PCB surface occupied by the \glref{OBC} circuitry is shrink to the minimal size possible. The remaining area is considered as a payload sector that can accommodate any user-defined modules. In our case, we decided to place a universal soldering array with exposed power lines all over this sector.
\enditems



%%%%% ELECTRONIC COMPONENTS

\label[chap:componentsSelection]
\sec Electronic components

\quad Electronic components are the building stones of each electronic hardware. Big commercial, military, or research spacecrafts use specialized components designed, modified, or particularly selected for space applications. CubeSats, on the other hand, use in most cases commercial off-the-shelf (\glref{COTS}) components that are not specifically designed for application within the space environment \cite[pap:tempModeling]. The use of \glref{COTS} components is certainly an attractive option for the development team of small satellite projects. Lead times are short, and recent developments in personal computing equipment and consumer electronic allow for high performance \cite[pap:thermal]. In the VST104 project, we also decided to follow this trend and to use COTS components. In this section, we describe various requirements applied to these components and the criteria of their selection.



%%%%% CERTIFICATION

\label[chap:componentsCertification]
\secc Components certification

\quad Spacecrafts are exposed to a harsh environment and its effects during their lifetime. This problem needs to be also addressed on the electronic components level by selecting components with adequate parameters. The two essential criteria that need to be watched are operational temperature range and mechanical stress resistance.

\"The temperature range that electronic components can encounter in orbit is quite large as the thermal control options are limited" \cite[boo:cubesatTemp]. \"The most crucial impact on the temperatures of a spacecraft is caused by the external radiation from Sun and Earth, as well as internal heat dissipation throughout electronic devices" \cite[pap:tempModeling]. Typical operational temperature for CubeSat components is in the range of $-40$ to $+85\rm{[^\circ C]}$ \cite[pap:tempModeling, boo:cubesatTemp], usually referred to as an industrial temperature range. As we wanted to be on the safe side, we decided to add a safety margin by extending the range. We chose the automotive range of $-40$ to $+125\rm{[^\circ C]}$ as a requirement for electronic components in our design.

\"During the launch, a spacecraft is subjected to various external loads resulting from vibroacoustic noise, booster ignition and burn out, propulsion system engine vibration, steady-state booster acceleration, and much more" \cite[pap:vibration]. Hence, various measures are applied to increase the robustness of the spacecraft. Regarding the \glref{OBC} design, it was necessary to select only the electronic components meeting the \glref{AEC}-Q100 and \glref{AEC}-Q200 standards. \"Components meeting these specifications are suitable for the harsh automotive environment without additional component-level qualification testing" \cite[nor:AEC].

To sum it up, all of the electronic components used in the Sierra Board i) are rated for the automotive temperature range of $-40$ to $+125\rm{[^\circ C]}$, and ii) obtained the \glref{AEC}-Q100 or \glref{AEC}-Q200 qualification. The only exception is the \glref{MCU} with no \glref{AEC} qualification.

Some of the \glref{OBC}'s potential applications may not require temperature and automotive certification of the electronics components. In such a case, some components may be replaced by their uncertified variants to save a bit of the financial budget. All of the electronics components with an official uncertified version are listed in table \ref[tab:uncertifiedParts]. Furthermore, the potential user is encouraged to replace all passive components (resistors, capacitors, ferrite beads, inductors) with unqualified parts of the same parameters. A replacement of oscillators Y[1,2] is possible, although this change is non-negligible.



%%%%% POWER CONSUMPTION

\label[chap:powerConsumption]
\secc Power consumption

\quad CubeSat's power budget is a closely monitored thing. Such a small spacecraft has a limited means of generating and storing electric power. Solar arrays are usually bound in their active surface and lack advanced tools of positioning. Similarly, battery storage systems are restricted by the available space and means of temperature control. It is crucial to minimize the power requirements of the CubeSat, including the \glref{OBC}. 

Reasonable power consumption of the \glref{OBC} can be achieved on the level of electronic components selection by prioritizing the low-power components. A downfall of this approach is usually an implied decrease in their performance, speed, or frequency. Therefore, we preferred components with similar parameters but lower power consumption. This affected mostly the selection process in cases of the \glref{MCU} and memory \glref{IC}s.

Not only a component's power consumption but also its efficiency is an important parameter. While choosing switching components such as electronic fuses, hot-side switches, or analog switches, a low on-resistance was preferred. Power losses on pull-down/pull-up resistors had to be also taken into consideration. A single $10\rm{[k\Omega]}$ pull-down resistor connected to a $3.3\rm{[V]}$ signal drains a constant current of $330\rm{[\mu A]}$, creating a power loss of approximately $1.1\rm{[mW]}$. During the schematic design, we aimed more at a subsystem's robustness rather than avoiding this particular type of power loss. Therefore, some of the pull-down/pull-up resistors may be left unpopulated and their functionality compensated by programmable resistors integrated inside the \glref{MCU}.

The summary of power consumption of the specific electronics components is listed in table \ref[tab:componentsConsumption]. Overall power consumption of specific \glref{OBC}'s submodules is listed in table \ref[tab:subsystemConsumption]. It sums power consumptions of all \glref{IC}s and losses on pull-down/pull-up resistors in the particular subsystem. For the computation, two assumptions were made: i) full functionality of the subsystem, ii) input voltage range as described in table \ref[tab:powerManagement].

\midinsert \clabel[tab:subsystemConsumption]{Subsystems consumption}
    \ctable{ccccccc}{
        \vspan2{Subsystem}  & \mspan3[c]{$3.3\rm{[V]}$ current $\rm{[mA]}$} & \mspan3[c]{$5\rm{[V]}$ current $\rm{[mA]}$} \cr
                            & Min.  & Typ.  & Max.  & Min.  & Typ.  & Max. \crl
        Processing unit     & -     & -     & -     & -     & -     & -    \cr
        Power management    & -     & -     & -     & -     & -     & -   \cr
        Header isolation    & -     & -     & -     & -     & -     & -    \cr
        Flash memories      & -     & -     & -     & -     & -     & -     \cr
        F-ram memories      & -     & -     & -     & -     & -     & -     \cr
        Dual CAN bus        & -     & -     & -     & -     & -     & -     \cr
        Temp. monitoring    & -     & -     & -     & -     & -     & -
    }
    \caption/t A rough estimation of particular subsystems' power consumption. Legend: min. - subsystem in standby mode, typ. - subsystem in typical operation, max. - subsystem peak performance. This table should be used for orientation purposes only. A potential user is encouraged to conduct a proper measurement for each Board Sierra application.
\endinsert



%%%%% ADDITIONAL PARAMETERS

\label[chap:additionalParams]
\secc Additional parameters

\quad A selection of particular electronic components is a vital part of designing circuitry. Usually, various electronic components share the same functionality and are available from different manufacturers. After filtering the available components by specific functionality (with respect to \ref[chap:componentsCertification] and \ref[chap:powerConsumption]), it was common to find various suitable candidates. Therefore, additional criteriums had to be set to resolve the selection:

\begitems
    * {\sbf Price:} As this \glref{OBC} is not primarily designed for an actual space flight (as explained in \ref[chap:introduction]), an individual component's price is not negligible. With a lower overall cost of the \glref{OBC}, a broader project expansion in the LibreCube community can be achieved. Thus components with sufficient attributes but lower price were favored.
    
    * {\sbf Footprint:} As a result of maximizing the PC104 payload sector, the actual \glref{OBC} area was significantly decreased. Therefore components of smaller dimensions available in fine-pitch packages (e.g. \glref{SSOP} or \glref{QFN}) were preferred. The same logic applies to passive components such as resistors and capacitors, resulting in a 0201 package (0603 in metric) being the most widely used in this design. After researching the technical capabilities and related costs of \glref{PCB} manufacturers, we decided not to use the \glref{BGA}-packaged components. Their significantly smaller footprints would require more precise fanout, resulting in increased manufacturing difficulty and price.
    
    * {\sbf Distributor:} We considered it important to use only typically stocked components available from the global distributors. In our case, Mouser electronic\fnote{http://www.mouser.com/}.This rule should repeal any future problems in components sourcing or logistics. Therefore, the availability of a specific component in this distributor was a selection factor.
\enditems



%%%%% PCB FEATURES AND DESIGN

\sec PCB features and design

\quad Designing the Board Sierra's \glref{PCB} was probably the most demanding part of the VST104 project. Restricted surface available for the complex \glref{OBC}'s circuitry, compact footprint of the PC104 main header with a not ideal location, and limited capabilities of the considered \glref{PCB}s manufacturers made the routing and fanout challenging. Visualization of the designed module rendered in KiCad is shown in figures \ref[app_vis:sierraTop] and \ref[app_vis:sierraBottom].



%%%%% PCB REQUIREMENTS

\secc PCB requirements

\quad The benefit of the Board Sierra's lower price regarding the particular electronic components was described in chapter \ref[chap:additionalParams]. The same logic applies to the decision-making during the \glref{PCB} design. It is a common feature between \glref{PCB}s manufacturers that the less advanced design, the cheaper \glref{PCB}. This includes many parameters such as copper layer count, presence of buried vias and their size, copper clearances, tracks widths, and much more. On the other hand, fine features such as buried vias simplify the process of \glref{PCB} design and allow more compact layouts. Therefore, it was crucial to find a balance between the manufacturing price and complexity of the \glref{OBC}'s layout and routing.

Although, following the proper design rules and standards created for automotive and especially space applications is much more important than lowering the \glref{PCB}'s manufacturing price. We addressed various requirements and suggestions listed in \glref{ECSS} standards ECSS-Q-ST-70-12C \cite[sta:ECSS-12C] and ECSS-Q-ST-70-60C \cite[sta:ECSS-60C]. Other precious recommendations were found and implemented from the TEC-ED IoD Board Specification \cite[sta:TEC-ED]. These specifications refer, for example, to track width and spacing, pad design, copper planes, or thermal rules. However, these documents are aimed at state-of-the-art spacecrafts designed and operated by the \glref{ESA}. Thus a punctual following of all of the requirements and suggestions was not necessary for our CubeSat application.



%%%%% OBC AND PAYLOAD SECTOR

\label[chap:payloadSector]
\secc OBC and payload sector

\quad As mentioned in chapter \ref[chap:capabilitesAndFeatures], a high ratio between the payload sector area and OBC area was set as one of the Board Sierra's key features. This requirement substantially affected the \glref{OBC}'s circuitry layout and fanout. After multiple iterations of components' placements and alignments, we have significantly reduced the required \glref{OBC} area. The payload sector covers roughly three-fourths of the available PC104 module surface, leaving only one-fourth to accommodate the OBC. Visual perception can be obtained from the Board Sierra 3D renders in figures \ref[app_vis:sierraTop] and \ref[app_vis:sierraBottom]. The \glref{OBC} sector is situated at the northwest quartal of the PC104 module, surrounded by the payload sector.



%%%%% PCB CHARACTERISTICS

\secc PCB characteristics

\quad The\glref{PCB}'s dimensions, geometry, and layout of the main connector and mounting holes are fully specified by the PC104 standard, as described in chapter \ref[chap:PC104standard]. The only modification added to this template are $1.9$ x $20.3\rm{[mm]}$ cutouts on the module's four edges. These cutouts were introduced in the LibreCube board specification and are designed to accommodate CubeSat's auxiliary power and data cables \cite[sta:libreBoard].

Regarding the manufacturing price and complexity, the\glref{PCB} would ideally be a four-layer one. This means two copper layers for signal traces, a power distribution layer, and a ground plane. Unfortunately, the two signal layers would not be enough to accommodate complex\glref{OBC}'s circuitry. Therefore, we had to choose a six-layer\glref{PCB} with the following setup: i) signal layers: top, second inner, bottom; ii) ground plane layers: first inner, fourth inner; iii) power distribution layers: third inner. \"An additional benefit of the added layers is improved thermal dissipation, as adding a layer of copper to the circuit board can significantly decrease the resulting temperatures" \cite[pap:tempModeling].

Additional parameters of the \glref{PCB} manufacturing such as material, isolation and copper thickness, surface finish and solder mask types are not specified in this thesis nor the VST104 project. It is the responsibility of the potential user to customize these parameters accordingly to the requirements of the specific application. In such a case, we encourage the user to follow relevant sections of the \glref{ECCS} standard \cite[sta:ECSS-12C].



%%%%% TRACES, VIAS AND ROUTING

\secc Traces, vias and routing

\quad Electronic components are placed on both sides of the \glref{PCB} due to the restricted OBC area. Corresponding top and bottom copper layers cover as much routing as possible. The second inside layer accommodates traces that do not fit into the two surface layers. The general width of a signal trace is $0.2\rm{[mm]}$, respectively $0.173\rm{[mm]}$ for traces near the main PC104 header. Clearance between traces themselves and other copper planes is set to $0.127\rm{[mm]}$. All of these parameters satisfy the minimal specifications in the \glref{ECSS} standard \cite[sta:ECSS-12C]. No blind, buried or micro vias are used in the design. The diameter of general via is $0.45\rm{[mm]}$ with a $0.24\rm{[mm]}$ drill. This parameters result in a $6.7$ aspect ratio for a standard $1.6\rm{[mm]}$ thick \glref{PCB}, which recognizes the \glref{ECSS} $\leq7$ rule \cite[sta:ECSS-12C].

The first and fourth inner layers are ground plane layers. Their purpose is to connect all of the component's \glref{GND} pins, provide shielding against the \glref{EMI} and act as heatsink dissipating components heat. Despite the \glref{ECSS} suggestion, both the ground planes have a solid fill instead of additional openings in a grid format \cite[sta:ECSS-12C]. The reason is, the current fifth version of KiCad does not support this option. This design flaw will be improved shortly as the checked fill option is included in the upcoming KiCad v6.

A copper plane with a solid fill is also located in the third inside layer. This plane is attached to a $3.3\rm{[V]}$ power source, and its task is to connect all required components to this power bus. This layer also accommodates a $5\rm{[V]}$ power distribution routing. Its traces are $0.45\rm{[mm]}$ wide with multiple vias to minimize the potential voltage drop.

The final \glref{PCB} routing highlighted by the particular \glref{OBC}'s subsystems is shown throughout multiple figures in chapter \ref[chap:BoardSierraSubmodules]. General principles of good routing were implemented throughout the \glref{PCB} design. These include steep turns avoiding, differential pairs matching, or sliver and peelable prevention \cite[sta:ECSS-60C]. If available, fanout and routing suggestions included in electronic components datasheets were followed. As suggested by the \glref{ECSS} \cite[sta:ECSS-12C], a teardrop finish was applied to all of the pads using the KiCad Teardrops extension listed in chapter \ref[chap:openSource]. It is important to mention that the current KiCad v5 does not support advanced design settings common for professional software such as Altium Designer or CadSoft EAGLE. Our approach to handling this problem was adding extra margins to the few parameters listed in KiCad, which increased the overall tolerance. The final \glref{PCB} design passed a design rule check (\glref{DRC}) with various parameters set according to manufacturers' standards and \glref{ECSS} suggestions \cite[sta:ECSS-60C].
