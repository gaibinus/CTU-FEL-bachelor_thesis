
%%%%% INTRODUCTION

\label[chap:intro]
\chap Introduction

\quad An ongoing revolution in the space sector is known under the name of New Space. Innovative entrepreneurs are entering the field traditionally occupied by institutions to exploit new opportunities. One of the promising domains is the space data \cite[pap:currentSpace]. Precise navigation and planning, Earth environment monitoring, security and surveillance, internet of things communications. These are only a few examples of possible services that could be offered to various terrestrial companies and even the general public.

\"Access to space is now broadening thanks to technology miniaturization. With a tremendous forecasted increase in the launch rate for small satellites, constellations are getting attention again from sustainable businesses." \cite[pap:constellations] Although the small physical size of spacecrafts makes the space more affordable, it is also a very limiting factor in terms of those systems performance \cite[pap:currentSpace]. A possible solution could be the development of new algorithms and processing techniques tailored for these specific space applications.

VisionSpace Technologies (\glref{VST}) is a New Space company located in the German city of Darmstadt, developing and integrating enterprise-level solutions for satellite missions. For some time now, the company has been working on algorithms for telemetry data compression and constellations mission planning. In 2020, the company decided to create a small hardware platform for testing mission control systems and the further development of the mentioned algorithms \cite[abs:openSourceWorkshop]. This platform was meant to consist of an open-source family of electronics boards in the Cubesat PC/104 format, including series of onboard computers and various development auxiliaries. The \glref{VST} did not intend to create launchable products. Although complying with the nuances of space engineering was expected in order to develop a hardware with appropriate characteristics.

In the same year, we have joined the company as a summer intern to proceed with this hardware development project. In this thesis, we present our commitment to the project while providing documentation for the developed boards. Chapter two briefly introduces the related space background and the project's motivation, goals, and workflow. Chapters three, four, and five are devoted to the developed Cubesat modules, describing the particular development steps and provide documentation for the modules' designs. Chapter six presents the design of an auxiliary FlatSat test bench. The results of a radiation testing of one of the modules are shown in the last chapter.