% On-board computer for PC104 format CubeSats
% ---------------------------------------------------------------------
% Filip Geib Apr. 2021

% export in Long-term Archiving (PDF/A)
% conversion needed: https://pdfrecover.herokuapp.com/pdfaconvert/
\input glyphtounicode.tex
\input glyphtounicode-cmr.tex
\input glyphtounicode-ntx.tex
\pdfgentounicode=1

% define English quotes
\enquotes

% include template
\input ctustyle3

\worktype [B/EN]

\faculty    {F3}
\department {Department of Measurement}
\title      {On-board computer for PC104 format CubeSats}
\subtitle   {}

\author     {Filip  Geib}
\date       {May 2021}
\supervisor {Ing. Vojtěch Petrucha, Ph.D.}
\studyinfo  {Cybernetics and Robotics}
\workname   {}

\workinfo   {} %\url{https://github.com/visionspacetec/VST104}
\titleSK    {Palubný počítač pre CubeSaty formátu PC104}
\subtitleSK {}

\pagetwo    {}



%%%%% ABSTRACT - skontrolovane

\abstractEN {
    This thesis is focused on the development of PC/104 format electronic boards for CubeSat applications. Particular attention was given to a single on-board computer (OBC) module. This universal ARM-based OBC is driven by an STM32L4 microcontroller supporting a wide range of interfaces. Its additional features include robust power management, separate peripheral isolation, triple-redundant FLASH and F-RAM memories, two CAN bus transceivers, built-in temperature monitoring, and a significant payload sector. A dedicated radiation testing was conducted under a gamma radiation source. Moreover, three additional boards were developed, including a double redundant version of the OBC, a universal PC/104 module, and a FlatSat test bench. All of these boards were designed with open-source principles in a KiCad environment. This thesis contributes to the VisionSpace Technologies VST104 project by introducing a hardware platform for mission control systems testing and compression algorithms development.
    \bigskip
}
\abstractSK {
    Táto práca je zameraná na vývoj elektronických dosiek formátu PC/104 pre CubeSat aplikácie. Osobitná pozornosť bola venovaná modulu palubného počítača (OBC). Tento univerzálny OBC založený na ARM technológii je riadený mikrokontrolérom STM32L4 podporujúcim širokú škálu rozhraní. Medzi jeho ďalšie funkcie patrí robustná správa napájania, samostatná izolácia periférii, trojito redundantné pamäte FLASH a F-RAM, dva komunikátory zbernice CAN, zabudované monitorovanie teploty a rozsiahly sektor užitočného nákladu. Cielené testovanie odolnosti voči žiareniu bolo uskutočnené pod zdrojom gama radiácie. Ďalej boli vyvinuté tri dosky, vrátane dvojito redundantnej verzie OBC, univerzálneho modulu PC/104 a testovacej platformy FlatSat. Všetky tieto dosky boli navrhnuté podľa princípov open-source v prostredí KiCad. Táto práca prispieva k projektu VisionSpace Technologies VST104 zavedením hardvérovej platformy určeneje pre testovanie systémov riadenia misií a vývoja kompresných algoritmov.
    \bigskip
}



%%%%% KEYWORDS

\keywordsEN {
    CubeSat; FlatSat; OBC; PC104; radiation testing.
}
\keywordsSK {
    CubeSat; FlatSat; OBC; PC104; radiačné testovanie.
}



%%%%% ACKNOWLEDGEMENT

\thanks {
    I would like to express my gratitude to VisionSpace Technologies, namely to José Feiteirinha, for conducting and sponsoring this project. I also thank my supervisor Vojtěch Petrucha for his guidance and support.
}



%%%%% DECLARATION

\declaration {
    I hereby declare that the presented work was developed independently and that I have listed all sources of information used within it in accordance with the methodical instructions for observing the ethical principles in the preparation of university theses. 
    \bigskip
    In Prague on May 21, 2021
    \signature % makes dots
}

%%%%% MACROS

%\draft     % Uncomment this if the version of your document is working only.
%\linespacing=1.7  % uncomment this if you need more spaces between lines
                   % Warning: this works only when \draft is activated!
%\savetoner        % Turns off the lightBlue backround of tables and
                   % verbatims, only for \draft version.
%\blackwhite       % Use this if you need really Black+White thesis.
%\onesideprinting  % Use this if you really don't use duplex printing. 
\input database/glosdata % Include glossary

%%%%% MAIN TEXT

% make title page, acknowledgment, contents etc.
\makefront

% introduction chapter

\label[chap:intro]
\chap Introduction % skontrolovane

% main body

%%%%% VST104 PROJECT

\chap VST104 project

\quad This project is being conducted and sponsored by the company VisionSpace Technologies (\glref{VST}), as mentioned in chapter \ref[chap:intro]. The project name, VST104, is a compilation of the company abbreviation \glref{VST} and standard CubeSat board format PC/104 (section \ref[chap:PC104standard]). In this chapter, we present a brief introduction into related space industry background together with the project's motivation, goals and workflow.



%%%%% PROJECT BACKGROUND

\sec Project background

\quad The VST104 project is about designing electronics hardware for space applications. Therefore a fundamental orientation in a part of the space industry jargon is required. In this section, we provide a short introduction to the corresponding space industry concepts. We briefly explain what CubeSats, PC/104 boards, Fltasats, and onboard computers (\glref{OBC}s) are. A summary of recently available OBC modules is also listed.



%%%%% CUBESAT CONCEPT

\secc CubeSat concept

\quad CubeSats are modular spacecraft from a picosatellite class, usually constructed of similar components and limited to specific dimensions and materials. CubeSats are being developed in several sizes, which are based on a standard 1U unit \cite[boo:nasa101]. This unit is defined as being $100.0 ± 0.1\rm{[mm]}$ wide and $113.5 ± 0.1\rm{[mm]}$ tall, with a limited mass of $1.33\rm{[kg]}$ \cite[app:cubeSat]. Numerous CubeSats also include deployable subsystems, such as antennas, probes, or solar panels that exceed the normative dimensions when deployed \cite[pap:tempModeling]. For illustration, some of the already launched CubeSats are shown in figure \ref[img:cubesats].

\midinsert
    \clabel[img:cubesats]{Existing CubeSats}
    \picw=0.75\hsize \cinspic figures/illustrations/cubesats.pdf
    \caption/f Photos of already launched CubeSats of various sizes. 1U SkCube (left), 2U Antelsat (center), 3U Grifex (right). Sources: TASR, GAUSS, NASA.
\endinsert

\"CubeSats are very popular among universities and other non-commercials groups globally. Larger space companies are developing CubeSat missions in-house to train new employees and assess the possibilities of new technologies." \cite[boo:cubesatTemp] \"Several types of CubeSats have been developed and deployed for a specific mission, such as research and development satellites, earth remote sensing satellites, and space tethers satellites." \cite[pap:vibration]

\"A CubeSat must conform to specific criteria that control factors such as its shape, size, and weight. The standardized aspects of CubeSats make it possible for companies to mass-produce components and offer off-the-shelf parts. As a result, the engineering and development of CubeSats become less costly than highly customized small satellites. The standardized shape and size also reduces costs associated with transporting them to, and deploying them into, space." \cite[boo:nasa101] Some of the standards were introduced by the concept's originators in the CubeSat Design Specification \cite[app:cubeSat]. A team from the \glref{JPL} has compiled the CubeSat principles in The CubeSat Approach to Space Access \cite[pap:approachToSpace].



%%%%% PC/104 STANDARD

\label[chap:PC104standard]
\secc PC/104 standard

\quad \"The CubeSat industry has adopted the PC/104 specifications \cite[app:pc104] as a de-facto standard for electronic boards. Moreover, such specifications provide mechanical and electrical benefits towards CubeSat fabrication beyond the compatibility with different structure and electronics suppliers. Following the PC/104 specifications, all electronic boards must measure $90.17$ x $95.89\rm{[mm]}$, and the electric bus must allocate four rows with 26 contacts of standard $2.54\rm{[mm]}$ spacing through-hole (\glref{THT}) headers." \cite[pap:fromDesignToOperation]

The PC/104 boards are meant to be stacked on top of each other, forming a rigid structure. The 104 pin headers provide a electrical connection between the individual boards, creating one electrical system. Excluding the headers, the PC/104 boards are firmly attached together with M3 standoffs placed in the corner mounting holes. This combination of the shared electrical bus and M3 bolts improves the stiffness provided by the CubeSat’s structure and simplifies the internal harnessing \cite[pap:fromDesignToOperation]. 

\midinsert
    \clabel[img:PC104]{PC/104 board dimensions}
    \picw=1\hsize \cinspic figures/illustrations/PC104Drawing.pdf
    \caption/f Technical drawings of the LibreCube PC/104 board. Overall geometry (left) and edge cutouts (right). All dimensions are in $\rm{[mm]}$. Source: LibreCube.
\endinsert

As the PC/104 standard allows some freedom for changes, a slightly modified PC/104 board template was used in this project. The template was designed by the LibreCube initiative \cite[sta:libreBoard] and its drawings are shown in figure \ref[img:PC104]. The only modification done to the original PC/104 board are $1.9\rm$ x $20.3\rm{[mm]}$ cutouts located on the board's four edges. These cutouts are designed to accommodate CubeSat's auxiliary cables.



%%%%% FLATSAT TEST BENCH

\secc Flatsat test bench

\quad During the hardware and software debugging and development, physical access to a particular CubeSat module is usually required. As the spacing between the stacked PC/104 boards is only $15.24\rm{[mm]}$ \cite[app:pc104], it is not easy nor practical to work with these already stacked modules. Therefore, a test bench called Flatsat is often used to mount all of the CubeSats modules next to each other on a plain surface. This bench should substitute an electrical bus and connect all of the PC/104 headers together. Some of the available Flatsats accommodate additional features such as an inbuild power supply, sun power emulator, or \glref{ESD} protection. A modular Flatsat is shown in figure \ref[img:aalto_flatsat].

\midinsert
    \clabel[img:aalto_flatsat]{PC/104 board dimensions}
    \picw=0.80\hsize \cinspic figures/illustrations/flatsat_aalto.pdf
    \caption/f The first version of the Aalto-3 Flatsat. Source: AaltoSatellites.
\endinsert



%%%%% OBC

\secc On-Board Computer

\quad \"The onboard computer (\glref{OBC}) in CubeSat is the module that acts as a bridge connecting the other modules with each other. It supervises many of the tasks done by the different modules of a satellite and performs housekeeping and monitoring to ensure the health and status of those modules. The hardware and software design of the OBC mainly depends on the mission of the CubeSat." \cite[pap:obc] The three main design parameters for \glref{OBC}'s electronic subsystems are power consumption, physical dimensions, redundancy, and radiation environment behavior \cite[boo:framInSpace].



%%%%% EXISTING OBC MODULES

\label[chap:existingOBCmodules]
\secc Existing OBC modules

\quad \"Despite growing interest from industry in CubeSats as means of technology demonstration, such platforms are still primarily considered as an educational tool." \cite[pap:fromDesignToOperation] Various universities and research institutions around the globe are developing and launching their own CubeSats. The failure rate of these university-build spacecrafts was estimated slightly below 50\% in the first six months of their operation, as stated by two surveys from 2013 \cite[pap:failRate2012] and 2016 \cite[pap:failRate2014]. Since the \glref{OBC} is a critical module and has caused 20\% of these failures \cite[pap:failRate2014], it might be counterproductive to consider these designs as reference ones. Some of these CubeSats were although highly successful, and their \glref{OBC} modules are worth mentioning. For example, the Atmel's \glref{ARM} based \glref{OBC} of StudSat \cite[obc:studsat], or the 16-bit MSP430 microprocessor based \glref{OBC}s of Libertad-1 \cite[obc:libertad] and RAX-2 \cite[obc:rax2].

A popular option for research institutions is to buy a professionally designed and already flight-proven CubeSat. Multiple space companies have therefore developed their own designs, including PC/104 \glref{OBC} modules. Some of the recently available \glref{OBC}s on the global market are: a rugged DP-OBC-0402 by Data Patterns \cite[obc:dataPatterns], a high-performance iOBC by ISIS \cite[obc:isis] (figure \ref[img:existing_obcs]), state of the art IMT CubeSat \glref{OBC} by IMT \cite[obc:imt], a lightweight and cost-saving CubeSat \glref{OBC} by German Orbital Systems \cite[obc:gos], a motherboard for harsh environment CubeSat Kit Motherboard by Pumpkin \cite[obc:pumpkin], a general-purpose ABACUS by Gauss \cite[obc:gauss], an \'always-on' operation KRYTEN-M3 by AAC Clyde Space \cite[obc:aac], an telemetry, tracking \& command and data processing unit Antelope \glref{OBC} by Antelope \cite[obc:anteleope], a space-qualified processor unit NANOSATPRO by STM \cite[obc:nanosatpro], an \glref{OBC} for mission-critical space application NanoMind A712D by GomSpace \cite[obc:gom] and a highly integrated main bus unit SatBus 3C2 by NanoAvionics \cite[obc:satbus] (figure \ref[img:existing_obcs]). Regarding numerous successful flights and continuous development in a highly competitive industry, we can consider these \glref{OBC}s as state-of-the-art designs. Multiple references to these modules could be found through the thesis.

%\cite[obc:dataPatterns, obc:isis, obc:imt, obc:gos, obc:pumpkin, obc:gauss, obc:aac, obc:anteleope, obc:nanosatpro, obc:gom]

\midinsert
    \clabel[img:existing_obcs]{Existing OBCs and Flatsat}
    \picw=0.95\hsize \cinspic figures/illustrations/obcs.pdf
    \caption/f Photos of existing professionally developed OBCs. iOBC by ISIS (left) and SatBus 3C2 by NanoAvionics (right). Sources: ISIS, NanoAvionics.
\endinsert



%%%%% VST104 MOTIVATION

\sec Project motivation

\quad \"Increasing the amount of science and housekeeping data increases a mission's value, but this comes with extended costs and challenges, especially for CubeSat missions. The \glref{VST} focuses on telemetry data compression, mainly the \glref{CCSDS} standard \'Robust compression of fixed-length housekeeping data' (POCKET+) to tackle this problem. The algorithm performance was verified, and the \glref{VST} developed a concept supporting multiple frame sizes." \cite[abs:cubesatIndustryDays] This includes the very first implementation of the POCKET+ in a hardware description language and synthesis for a radiation-hardened \glref{FPGA}.

To enable the use and future development of this concept on CubeSats, the VST104 project was established by the company. The project's main goal was set to develop a series of onboard computer (\glref{OBC}) CubeSat modules serving as a software-defined platform. These boards should host a single or redundant \glref{OBC} paired with an \glref{FPGA} for hardware acceleration. It is essential to state that the VST104 project is still ongoing. Our contribution (thus the content of this thesis) was limited to developing early versions of the \glref{OBC} modules and their auxiliaries. The motivation and expectation behind every one of them, together with the project's open-source principles and contribution to CubeSat communities, are explained in the rest of this section.



%%%% VST104 BOARDS FAMILY

\secc VST104 boards family

\quad The family of boards developed under the VST104 project has four members at the time of writing this thesis. All of them were designed by ourselves as our contribution to this project. The role of each board in the project and its potential use cases are listed below. A detailed description of the boards is provided in the remaining chapters.

\begitems
    * {\sbf Board Zero:} A prototyping module in the PC/104 format. This board accommodates an array of THT pads with multiple $3.3\rm{[V]}$, $5\rm{[V]}$ and \glref{GND} power rails, and four universal SO-24 footprints. The Board Zero should provide an efficient tool for rapid-prototyping temporary circuits with \glref{THT} or \glref{SMD} technology. Since the design of this board is very basic, no specific description is provided in this thesis. Nevertheless, the same design was implemented to a payload sector of the Board Sierra and is described in section \ref[chap:sierra_payloadSector]. A 3D render of the Board Zero is shown in figure 100.
    * {\sbf Board Sierra:} A single onboard compute (\glref{OBC}) module in the PC/104 format. This board is the most important part of our contribution. The goal was to design a universal \glref{OBC} while fulfilling most of the space industry requirements and keeping the manufacturing cost down \cite[abs:openSourceWorkshop]. This module should serve as a template for future VST140 \glref{OBC} variants. Its significant payload sector is designed to be replaced with additional circuitry, particularly with an FPGA. The \glref{VST} proposed a development of this board to obtain a CubeSat computer module for future development and testing of various algorithms. Chapters \ref[chap:boardSierra] and \ref[chap:boardSierraSubsystems] are fully devoted to this board.
    * {\sbf Board Delta:} A double redundant PC/104 \glref{OBC} module. \"This board implements the Board Sierra in a double redundant configuration. The potential of this board is in user scenarios where reliability is essential. In case of system fraud, a supreme logic should simply switch between the identical \glref{OBC}s sharing (almost) the same software and electrical characteristics." \cite[abs:openSourceWorkshop] More details are provided in chapter \ref[chap:boardDelta].
    * {\sbf Element Foxtrot:} A FlatSat test bench with an inbuilt power supply. This auxiliary board should serve as a development and show-off test bench, capable of hosting and powering up multiple PC/104 modules.  \"Element Foxtrot is an ideal tool for testing and developing different CubeSat modules, temporary replacement for power distribution unit (\glref{PDU}), and a nice way to present already developed modules." \cite[abs:openSourceWorkshop] Features and design of this FlatSat are fully described in chapter \ref[chap:ElementFoxtrot].
\enditems

Future expansion of the VST104 board family is planned, and the \glref{VST} team is currently working on new boards. The previously mentioned extension of the Board Sierra with an \glref{FPGA} and its supporting circuitry should be the first one of them \cite[abs:cubesatIndustryDays].



%%%%% COMMUNITY AND OPEN SOURCE

\label[chap:openSource]
\secc Community \& open-source

\quad Concerning the previously explained motivation behind this project, there is no interest by the \glref{VST} in developing their own CubeSat. However, some future software tests may require additional CubeSat systems. It should also be possible to easily integrate the VST104 modules into a third-party CubeSat during a flight opportunity. Therefore cooperation with existing organizations developing CubeSats is essential.

At the beginning of the project, some of the VST employees were already members of the LibreCube initiative. This organization aims to develop ready-for-use CubeSat elements for space and earth exploration missions. The VST supervisors decided to follow the LibreCube PC/104 template and their header pins assignment (as described in sections \ref[chap:PC104standard] and \ref[chap:mainHeaderPinout]).  This opened the possibility of combining the VST104 boards with any LibreCube element while contributing to such an exciting initiative.

Another organization that the \glref{VST} supervisors chose to cooperate with is named TUDSaT. It is a research group formed by students at the local technical university of Darmstadt. These students are interested in space exploration, and one of their projects is developing a 1U CubeSat. Throughout our contribution to the VST104 family, we were in touch with their project leaders. We discussed some issues, and we both slightly tweaked our project for better compatibility. At the end of our internship, the TUDSaT received a fully assembled Board Sierra module together with an Element Foxtrot.

The \glref{VST} supervisors intended the project to be fully open-source from its beginning. This decision was determined by multiple reasons. As the simultaneously developed \glref{VST} software stack is free and open-source \cite[abs:cubesatIndustryDays], proprietary hardware would make no sense. Being open-source is also required to be able to support and contribute to the two previously mentioned organizations. Both the LibreCube and TUDSaT share strong open-source philosophies. Lastly, the open-source concept goes hand in hand with CubeSats' educational and research principles. The entire VST104 project is licensed under the \'Strongly Reciprocal CERN Open Hardware Licence Version 2' (CERN-OHL-S) \cite[lic:cern] and is publicly accessible on a company GitHub repository.



%%%%% PROJECT WORKFLOW

\sec Project workflow

\quad The VST104 project turned out to be quite complex, and we had to undergo multiple design steps. Many of them were related to a specific VST104 board, but some were more general. In this section, we address the design steps common for the entire project. We discuss the KiCad environment, the \glref{VST} libraries, the project's GitHub repositories, the PC/104 pin assignment, and the assembly of the VST104 boards.



%%%%% KICAD AND ITS PLUGINS

\label[chap:kicadPlugins]
\secc KiCad and its plugins

\quad Since the VST104 project is mainly about designing electronics hardware, proper PCB design software had to be selected. There are multiple professional software packages available, such as Altium Designer or CadSoft EAGLE. Although these design tools come with powerful features, they do not fit with the open-source policy of the project (section \ref[chap:openSource]). The VST supervisors chose the KiCad EDA instead. This tool is probably the most popular one in the open hardware and makers community. On top of that, both LibreCube and TUDSaT use KiCad for their projects.

As the KiCad is an open-source project itself, various extensions were created by the community. These action plugins are not included in the official KiCad distribution but are generally well-behaved and helpful. We have also used some of them during the development of the VST104 boards to compensate for missing features. Particularly:

\begitems
    * {\sbf Interactive HTML BOM}\urlnote{https://github.com/openscopeproject/InteractiveHtmlBom} for more convenient \glref{PCB} assembly.
    * {\sbf Teardrops}\urlnote{https://github.com/NilujePerchut/kicad_scripts/tree/master/teardrops} to generate teardrops patterns for traces and pads.
    * {\sbf Replicate layout}\urlnote{https://github.com/MitjaNemec/Kicad_action_plugins/tree/master/replicate_layout} to copy and paste chunks of traced circuitry.
    * {\sbf RF-Tools for KiCAD}\urlnote{https://github.com/easyw/RF-tools-KiCAD} to measure and tune differential pairs.
\enditems



%%%%% VST104 KICAD LIBRARIES

\label[chap:vst104Libraries]
\secc VST104 KiCad libraries

\quad Although the KiCad official symbol libraries are pretty large, they do not include some of the electronic components used in the VST104 boards. The majority of these missing symbols correspond to specialized integrated circuits (\glref{IC}s). To resolve this problem, we created a project-related symbol library containing all of the missing symbols. We designed these symbols accordingly to components’ datasheets and the KiCad library conventions \cite[app:kiCadLib].  If a symbol was available in the official library, we have used it.

Every electronic component needs to be properly attached to the \glref{PCB}'s surface. An arrangement of pads required to solder the component on the \glref{PCB} is called a footprint. Components with different standardized packages require corresponding footprints. These footprints are usually available for download on various websites or are included in the KiCad official libraries. Although it is comfortable to use these premade footprints, this approach brings inconsistency and dependency to the \glref{PCB} design. Therefore we decided to create each of the used footprints by ourselves, following reference designs in components' datasheets and the KiCad library conventions \cite[app:kiCadLib]. 

It is a common feature in modern PCB design software such as KiCad to support a 3D visualization. Besides pads, copper traces, or drill holes, the 3D render can visualize particular electronic components. We consider this tool crucial for creating documentation or presenting the VST104 project. However, to enable this feature, WRL or STEP files containing the components' 3D models are required. Thus, while creating an individual component's footprint, we also searched the web for its 3D model files. The WRL provides support for material properties, allowing superior 3D rendering. Unfortunately, it is not so common for component manufacturers to offer this type of model. In such a case, a fake WRL was created from the available STEP file.

The {\tt VST104-Libraries} are structured into three separate folders: {\tt VST140_symbols}, {\tt VST104_footprints}, and {\tt VST104_logos}. At the time of writing this thesis, the {\tt VST104_sybols} contain 16 symbols. The total number of 84 individual footprints and their associated WRL and STEP 3D models are included in the {\tt VST_footprints}. The {\tt VST104_logos} was not described in this section by now. This folder contains 14 silkscreen graphics of various logos in different sizes (\glref{VST}, open-source Hardware, and LibreCube logo). These logos are used through all of the VST104 boards. 



%%%%% GITHUB REPOSITOT

\secc GitHub repository

\quad The entire VST104 project is available on the company's GitHub. The main repository is called {\tt VST104}\urlnote{https://github.com/visionspacetec/VST104} and acts as a project crossroad. It lists connections to other repositories and provides a brief description of the project itself. Each of the VST104 boards has its own repository. This GitHub structure was suggested by the LibreCube community. Its main benefit is the possibility to push and review changes to the different boards separately. The \glref{VST} libraries also have their separate GitHub repository linked as a submodule to each of the boards. With this approach, only one shared library exists instead of multiple board-limited libraries with duplicity content.



%%%% MAIN HEADER PINOUT

\label[chap:mainHeaderPinout]
\secc Main header pinout

\quad Despite the international effort to achieve modularity of the CubeSats, the pinout of the main PC/104 header changes between different manufacturers and missions. \"The allocation and distribution topology for power are not taken over, nor standardized for CubeSats, leading to compatibility issues. Therefore, when PC/104 is mentioned as standard in relation to CubeSats, this refers to a fixed physical wiring harness and the mechanical layout, and not the data bus or pin allocation." \cite[pap:reliability] Therefore, as our first contribution to the VST104 project, we had to come with a reasonable pin assignment.

After researching publicly available pinouts of different CubeSat manufacturers and related organizations (NanoAvionics, ISIS, Gomspace, Endurosta, TUDSaT, LibreCubeSat, etc.) we proposed and implemented PC/104 pin assignment as shown in figure \ref[app_vst:vst104_pinout]. Some of the signals with not so self-explaining names are explained in table \ref[tab:vst104_pinout]. This so-called \glref{VTS} pinout is present in all of the VST104 boards. Our main goal during its creation was to provide the highest compatibility with other CubeSats as possible while accommodating various data buses and system maintenance signals.

It is essential to state that the currently presented \glref{VTS} pinout is a subject of change in the nearest future. Our presentation of the VST104 project at the Open Source CubeSat Workshop 2020 \cite[abs:openSourceWorkshop] started a discussion about the PC/104 header pin assignment. As a follow-up to this workshop, multiple online meetings regarding the pinout hold place. Organizations developing open-source CubeSats modules, such as \glref{VTS}, LibreCube, AcubeSAT, or SatNOGS, are currently consolidating their pin assignments in order to become more compatible and possibly even uniformed.



%%%%% PCB ASSEMBLY

\secc PCB assembly

\quad A manual assembly of an electronics board is a time-consuming and, therefore, costly operation. This applies especially to double-sided boards with as many fine-pitch components as the Board Sierra and Delta have. To address this problem, we designed all VST104 boards accordingly to \glref{PCB} industry requirements so that the boards can be populated with electronics components by an assembly machine. The required design properties were, for example, respecting the minimal fabrication clearance of a component or strictly following the suggested solder paste distribution areas. We also addressed these requirements during the previously mentioned creation of \glref{VST} footprints (described in section \ref[chap:vst104Libraries]) while creating the documentation, courtyard, and paste layers (named as {\tt .Fab}, {\tt .CrtYd}, {\tt .Paste} in the KiCad jargon).

Exported Gerber files containing all of the fabrication and documentation layers are available for the VST104 boards in their GitHub repositories. Multiple versions of the bill of materials (\glref{BOM}) files are also available for each board. This includes files generated with i) a standard KiCad BOM export, ii) the Interactive HTML BOM plugin (described in section \ref[chap:kicadPlugins]), and iii) a project list on Mouser Electronics.

Despite the mentioned possibility of an automated assembly, we have manually populated a couple of the boards by ourselves. As the service of machine assembly is worthy only for higher quantities, it was not suitable for our development and testing purposes. By the end of the internship, we assembled four Board Sierra modules, together with four Element Foxtrot flatsats. Our approach can be summarized in the following steps: i) spread a solder paste to one side of the PCB using a stencil sheet, ii) place all of the electronics components to the \glref{PCB} with the help of an optical microscope and precise tweezers, iii) preheat the \glref{PCB} and sequentially solder the components with a hot air soldering station, iv) repeat the process with the other side of the board (if any). This method, especially step number ii, was tricky due to a small size of the components (most of the Board Sierra resistors footprints are 0201). However, all of the assembled boards turned out good and passed a visual and electrical inspection. Photographs of the assembled VST104 boards are listed in figures 100, 200, and 300.
 % skontrolovane

\chap Board Sierra - description

%%%%% CUBESATS %%%%%%%%%%%%%%%%%%%%%%%%%%%%%%%%%%%%%%%%%%%%%%%%%%%%%%%%%%%%%%%%%%%%%%%%%%%%%%%%%%%%%%%%

\sec CubeSat concept

%%%%% OBC REQUIREMENTS %%%%%%%%%%%%%%%%%%%%%%%%%%%%%%%%%%%%%%%%%%%%%%%%%%%%%%%%%%%%%%%%%%%%%%%%%%%%%%%%

\sec OBC requirements

%%%%% RADIATION

\secc Radiation and redundancy

\quad Occasionally traveling through weaker parts of the Earth's magnetic field and not shielded by the Earth's atmosphere, the CubeSats have to operate in an environment full of radiation. A direct hit of a high-energy particle might have serious consequences for \glref{OBC} functionality. These include transistor gate ruptures, memory bit flips, software upsets, or latch-ups. A proper strategy must be taken to increase the \glref{OBC} durability and ability to handle such an error, resulting in maximizing a possible mission life.

One approach is to use special radiation-hardened components. This strategy is typical for professional and more expensive satellites than the CubeSats. These components are usually bigger, more costly, and have less functionality than ordinary COTS.

Considering the size and budget requirements of our \glref{OBC}, we chose to implement another option. Instead of the previously mentioned physical hardening technique, a logical one was realized. The \glref{OBC} hosts many schematic design-related features ensuring the proper handling of any radiation-related event. These include: i) over-current sensing power management, ii) separate peripheral isolators, iii) full high-impedance mode requested by the higher logic, iv) triple-redundant memories, v) multiple temperature sensors. A fully double-redundant \glref{OBC} is presented in chapter \ref[chap3:delta].

%%%%% FEATURES

\label[chap2:features]
\secc Capabilities and features

\quad Having in mind the expectations of the \glref{VST} supervisors, features common for \glref{OBC}s by different professional manufacturers, and design requirements implied by the radiation, we created and implemented the following list of desired \glref{OBC}'s features:

\begitems
    * {\sbf Microcontroller:} 
    * {\sbf External clock sources:} The \glref{MCU} has two internal resistor-capacitor (\glref{RC}) oscillators that can be used to drive a master system clock and auxiliary clocks \cite[dat:mcu]. These internal oscillators have a significantly lower frequency stability and a higher temperature dependency than the external ones \cite[app:oscInt]. Therefore, to ensure the clock reliability in the harsh space environment, we had to implement external clock sources.
    * {\sbf Robust power management:} The electric power subsystem (\glref{EPS}) is known to be the most vital subsystem of a spacecraft. Its reliability and error handling should be ensured by the power control and distribution unit (\glref{PDCU}). However, it is a good engineering practice by professional \glref{OBC} manufacturers to include an additional power control to their designs \cite[obc:dataPatterns, obc:isis, obc:imt, obc:pumpkin, obc:gauss, obc:aac, obc:anteleope, obc:nanosatpro, obc:gos]. Our \glref{OBC} requires $3.3\rm{[V]}$ and $5\rm{[V]}$ power inputs from the main power buses. In the case of their malfunction, it is our responsibility to sense it and power down the \glref{OBC}. This feature is also important for some of the \glref{OBC} on-earth user cases. During a hardware development or a system presentation, the user might misconnect the power line or use an unsupported power source.
    * {\sbf Isolation of the peripherals:} The spacecraft \glref{OBC} is connected to many data buses shared among all other subsystems. In some scenarios, the \glref{OBC} must be able to isolate itself from a specific or multiple data buses. For example, to switch between \glref{OBC}s in a redundant configuration, handle the failure on a data bus, or prevent unintentional interferences. Standard approaches to addressing this isolation feature are analog switches \cite[the:wurzburg], optocouplers \cite[pap:satelliteLeo], or \glref{FPGA}s \cite[obc:dataPatterns]. These isolators should also guarantee that all data lines are high impedance when the \glref{OBC} is powered off.
    * {\sbf Redundant external memory:}
    * {\sbf CAN bus peripherals:}
    * {\sbf Temperature monitoring:}
    * {\sbf Maximal payload sector:}
\enditems

%%%%% CERTIFICATION

\secc Components certification

%%%%% SELECTION

\label[chap:componentsSelection]
\secc Components selection 

\quad After listing all of the technical requirements for a specific electronic part, we had to chose a particular component. As there are usually multiple similar components from various manufacturers, we had to use additional criteriums for the selection process.

As this \glref{OBC} module is not primarily designed for an actual space flight (explained in chapter \ref[chap1:introduction]), an individual component's price is not negligible. With a lower overall cost of the \glref{OBC}, a broader project expansion in the LibreCube community can be achieved. Thus components with sufficient attributes but lower price were favored.

Another criterium in the component selection was the available PCB surface. As a result of maximizing the payload sector of the PC104 module, the actual \glref{OBC} subsystem area was significantly decreased. Therefore the components of smaller dimensions available in fine-pitch packages (e.g. \glref{SSOP} or \glref{QFN}) were preferred. After researching the technical capabilities and related costs of PCB manufacturers, we decided not to use the \glref{BGA} package components. Their significantly small footprints would require more precise fanout, resulting in increased manufacturing difficulty and price.

Different components are available from various distributors, which can prolong the assembling process. We chose to use Mouser Electronics for purchasing the components. Therefore the availability of a specific component in this store was also a selection factor. This decision was influenced by the \glref{VST} supervisors and previous experiences.

%%%%% PCI104 %%%%%%%%%%%%%%%%%%%%%%%%%%%%%%%%%%%%%%%%%%%%%%%%%%%%%%%%%%%%%%%%%%%%%%%%%%%%%%%%%%%%%%%%%%

\sec PCI104 standard

%%%%% MECHANICAL

\secc Mechanical specification

%%%%% PINOUT

\secc Main header pinout

\sec PCB design and assembly
\secc PCB specifications
\secc Routing and fanout


 % skontrolovane

%%%%% BOARD SIERRA - SUBMODULES

\label[chap:boardSierraSubsystems]
\chap Board Sierra - subsystems

\quad In this chapter, we provide all details about the individual \glref{OBC} subsystems. We address one of them in each section. Firstly, we explain the importance of the subsystem in theory, and then we describe our approach and the design decisions we made. We also present and comment on each subsystem's electronics scheme design together with the design of its layout and routing on the Board Sierra \glref{PCB}. For a better image of the overall location and layout of these OBC subsystems, refer to figure \ref[img:sierra_subsystems]. 

\circleparams={\ratio=1 \fcolor=\White \lcolor=\Black \hhkern=0.9pt \vvkern=0.9pt}

\midinsert
    \clabel[img:sierra_subsystems]{Sierra subsystems summary}
    \picw=1\hsize \cinspic figures/model/sierra_subsystems.pdf
    \caption/f Location and layout of the Board Sierra \glref{OBC} subsystems. The top side of the \glref{OBC} is displayed on the left, whereas the bottom side is on the right. Legend: {\typosize[8/]\incircle{⟨1⟩}} microcontroller, {\typosize[8/]\incircle{⟨2⟩}} external clock sources, {\typosize[8/]\incircle{⟨3⟩}} power management, {\typosize[8/]\incircle{⟨4⟩}} peripheral isolators, {\typosize[8/]\incircle{⟨5⟩}} triple-redundant FLASH memory, {\typosize[8/]\incircle{⟨6⟩}} triple-redundant F-ram memory, {\typosize[8/]\incircle{⟨7⟩}} \glref{CAN} bus transceivers, {\typosize[8/]\incircle{⟨8⟩}} onboard temperature sensors, and {\typosize[8/]\incircle{⟨9⟩}} \glref{SWD} connector and  circuitry.
\endinsert



%%%%% MICROCONTROLLER

\sec Microcontroller

\quad The processing unit is the most important part of the \glref{OBC}. Professional designs use \quad{MCU}s with a wide variety of instruction set architectures \cite[obc:dataPatterns, obc:imt, obc:pumpkin, obc:gauss, obc:nanosatpro]. However, the \glref{ARM} architecture seems to be the most popular \cite[obc:isis, obc:gos, obc:aac, obc:anteleope, obc:gom, obc:satbus]. This architecture is known for its good multiprocessing support, low power consumption, affordable pricing, and existing applications in various fields. Considering these benefits and influenced by the LibreCube initiative, existing TUDSaT's \glos{OBC} project, and most importantly the \glref{VST} supervisors, we decided to pick an STM32 microprocessor.

Our task was to choose a particular model of this 32-bit, Arm and Cortex-M based \glref{MCU}. Aiming rather for a low-power than high-performance characteristics, we decided to select an L series. Particularly the L4 series as it combines the largest flash memory size with the highest number of general-purpose input/output pins (\glref{GPIO}s) \cite[dat:mcuSeries]. Although, only one device from the L4 series is equipped with two CAN bus channels. As the presence of a second bus is crucial for our double-redundant approach, the STM32L496xx option was selected. This family comes in six different packages. Avoiding all of the \glref{BGA}-like ones (as described in section \ref[chap:additionalParams]) narrows the selection to Zx, Vx, and Rx variants. The Zx is the most capable one in terms of flash size and \glref{GPIO} count. Therefore the STM32L496ZG (where G stands for extended operating temperature range) was our final choice \cite[dat:mcu]. From now on, we will refer to this particular device as the \glref{MCU}. Some of its key characteristics are listed in table \ref[tab:microcontroller].

\midinsert \clabel[tab:microcontroller]{Microcontroller parameters}
    \ctable{lcc|lcc}{
        Max. frequency      & $80\;\rm{[MHz]}$            &&& \glref{SPI}     & $3$     \cr
        Flash memory        & $1\;\rm{[MB]}$              &&& \glref{$\rm{I^2C}$}     & $4$     \cr
        Static RAM          & $320\;\rm{[kB]}$            &&& \glref{UART}    & $5$     \cr
        Comparators         & $2$                         &&& \glref{CAN}     & $2$     \cr
        Op. amplifiers      & $2$                         &&& \glref{GPIO}    & $115$   \cr
        Operation temp.     & $-40$ to $125\;\rm{[°C]}$   &&& \glref{DAC}     & $2$
    }
    \caption/t Highlighted characteristics of the STM32L469ZG microcontroller \cite[dat:mcu].
\endinsert

\secc Schematic design

\quad The \glref{MCU} pin assignment was continuously changing during the entire process of \glref{OBC} schematic and PCB design. Its final state is presented in figure \ref[app_sch:microcontroller]. We attempted to maximize the number of user-free \glref{GPIO}s with added functionalities such as \glref{ADC} or \glref{PWM} while keeping the fanout manageable. A significant help during this process was the CubeMX tool of the STM32CubeIde, visualizing all of the pinout combinations with a specific functionality. Each of the $3.3\rm{[V]}$ tolerant pins was used only for the internal circuitry, resulting in a fully $5\rm{[V]}$ tolerant main PC104 header connection.

A significant number of filtering and reservoir capacitors is needed to ensure the proper \glref{MCU} functionality. The assignment of correct capacitors to required \glref{MCU} pins was pretty straightforward, following the device datasheet \cite[dat:mcu] and STM32L4 hardware development application note \cite[app:getStart]. Furthermore, a $10\rm{[\mu H]}$ choke and $120\rm{[\Omega]}$ at $100\rm{[MHz]}$ ferrite bead were placed in series with the analog power input. This \glref{LC} filter supported by the bead should effectively eliminate both low and high-frequency interference.

The clock and data lines of both \glref{$\rm{I^2C}$} busses are equipped with standard $4.7\rm{[k\Omega]}$ resistors. Multiple $22\rm{[\Omega]}$ resistors were placed in series with high-speed clock signals of the \glref{SPI} interface. Two $0\rm{[\Omega]}$ resistors are present on {\it FAULT} and {\it MODE} lines to facilitate an optional hardware isolation. A $10\rm[k\Omega]$ resitor at {\it PH3} is suggested by \cite[app:getStart].

\secc PCB design

\quad Location and fanout of the \glref{MCU} are shown in figure \ref[pcb:microcontroller]. The \glref{MCU} is placed on the \glref{PCB} top side, covering most of the non-payload area. This big footprint of a LQFP144 package (approximately $2$ x $2\rm{[cm]}$) is a consequence of avoiding the \glref{BGA} packages while still trying to keep the pin count high. All of the capacitors were placed as close to their assigned pins as possible. In some cases, it was necessary to use the bottom side of the \glref{PCB}. The same approach was applied to all of the resistors. A hole of $1.2\rm{[mm]}$ diameter is placed roughly at the middle of the \glref{MCU} footprint. The diameter was set to fit a 19 gauge needle attached to a syringe with epoxy. This epoxy glue should be filled through this hole into the void space between the \glref{PCB} and the \glref{MCU}. This procedure is typical for space applications. Its purpose is to add some mechanical robustness to the LQFP device and create a thermal bridge between the \glref{PCB} and the devices.

\midinsert
    \clabel[pcb:microcontroller]{Microcontroller circuitry}
    \picw=0.85\hsize \cinspic figures/pcbs/microcontroller.pdf
    \caption/f Highlighted location of microcontroller circuitry.
\endinsert



%%%%% EXTERNAL CLOCK SOURCES

\sec External clock sources

\quad Proper timing and synchronization are the key features while dealing with high-speed data busses, \glref{ADC}s, or other precise applications. The \glref{MCU} is equipped with two internal \glref{RC} oscillators that can be used to drive a master system clock and other auxiliary clocks \cite[dat:mcu]. These internal oscillators generally have a significantly lower frequency stability,  a higher temperature dependency, and smaller overall accuracy than their external equivalents \cite[app:oscInt]. Therefore, to increase the clock precision and reliability in the harsh space environment, we had to implement external clock sources.

A $4-48\rm{[MHz]}$ high speed external oscillator (\glref{HSE}) can drive the system clock. Supported types are crystal, ceramic resonator, or silicone oscillator \cite[dat:mcu]. The last option seems to be the best as it is insensitive to electromagnetic interference (\glref{EMI}) and vibration. The only downside is its slightly lower temperature rejection \cite[app:oscComp]. We chose the SiT8924B, a $26\rm{[MHz]}$ silicon microelectromechanical system (\glref{MEMS}) oscillator \cite[dat:hse]. Accordingly to the clock configuration tool of the stm32cube software, we can reach various system clock and auxiliary clocks frequencies up to $78\rm{[MHz]}$ (maximum is $80\rm{[MHz]}$ \cite[dat:mcu]). This is done using the phase-locked loop (\glref{PLL}) clock generation.

A $32.768\rm{[kHz]}$ low speed external oscillator (\glref{LSE}) can drive the real time clock (\glref{RTC}), hardware auto calibration, or other timing functions \cite[dat:mcu]. Table 7 in \cite[app:oscDes] recommends individual crystal resonators for combination with STM32 \glref{MCU}s. After a consideration of these options, we decided to pick the ABS07AIG ceramic base crystal \cite[dat:lse].

\secc Schematic design

\quad The \glref{HSE} circuitry follows the oscillator datasheet \cite[dat:hse]  and is shown on the right side of figure \ref[sch:clockSources]. Only a decoupling capacitor and a terminator resistor are required. The clock output {\it OSE_IN} can be enabled or disabled by the binary {\it OSC_EN} signal.

 The \glref{LSE} circuitry is based on a reference design in the oscillator design guide \cite[app:oscDes]. To achieve a stable frequency of this Pierce oscillator, it is required to find the values of load capacitors $C_{L1}$, $C_{L2}$ and external resistor $R_{E}$. This can be done using equations
$$ C_L = {{C_{L1} C_{L2}}\over{C_{L1}} + C_{L2}} + C_S \quad \wedge \quad C_{L1} = C_{L2}, \eqmark $$
$$ R_{E} = {1\over{2\pi f C_{L2}}}, \eqmark $$
where $C_L$ is the crystal load capacitance, $f$ is the crystal oscillation frequency and $C_S$ is the stray capacitance \cite[app:oscDes]. Values of $C_L$ and $f$ are listed in the crystal datasheet \cite[dat:lse]. We can assume as a rule of thumb, that $C_S=4\rm{[pF]}$. The final \glref{LSE} circuitry with computed values of the components is shown on the left side of figure \ref[sch:clockSources].

\midinsert
    \clabel[sch:clockSources]{Clock sources schematic}
    \picw=0.5\hsize \cinspic figures/schemes/clockSources.pdf
    \caption/f Schematic diagram of external clock sources.
\endinsert

\secc PCB design

\quad Location and fanout of the external clock circuitry are shown in figure \ref[pcb:clockSources]. All of the components are placed on the bottom side of the \glref{PCB}. The circuitry layout follows multiple tips presented in the oscillator design guide \cite[app:oscDes]. Separate \glref{GND} planes are assigned to both the \glref{HSE} and \glref{LSE} circuitry. These planes are bounded by guard rings, formed by series of vias. Each of the planes is connected to a common \glref{GND} only at one point. This approach provides proper \glref{EMI} shielding while reducing a ground loop effect. We also minimized the distance between the \glref{MCU} pins and both oscillators. All these measures combined should improve the clock generation stability and robustness.

\midinsert
    \clabel[pcb:clockSources]{Clock sources circuitry}
    \picw=0.85\hsize \cinspic figures/pcbs/clockSources.pdf
    \caption/f Highlighted location of external clock sources circuitry.
\endinsert



%%%%% POWER MANAGEMENT

\sec Power management

\quad The electric power subsystem (\glref{EPS}) is known to be the most vital subsystem of a spacecraft \cite[pap:failRate2012, pap:failRate2016]. Its reliability and error handling should be ensured by the power control and distribution unit (\glref{PDCU}). However, it is a good practice by professional manufacturers to include an additional power monitoring and control to their \glref{OBC} design \cite[obc:dataPatterns, obc:isis, obc:imt, obc:gos, obc:pumpkin, obc:gauss, obc:aac, obc:anteleope, obc:nanosatpro, obc:gom, obc:satbus]. We have to implement circuitry that can sense a power bus malfunction and, as a response, power down the \glref{OBC}. This feature is also beneficial for some of the \glref{OBC} on-earth user cases. During a hardware development or a presentation, the user might misconnect the power line or use an unsupported power source by a mistake.

\midinsert
    \clabel[dia:powerManagement]{Power management diagram}
    \picw=0.75\hsize \cinspic figures/diagrams/powerManagement.pdf
    \caption/f Functional diagram of power management circuitry.
\endinsert

A functional diagram of the implemented power management is shown in figure \ref[dia:powerManagement]. To increase the overall efficiency and decrease complexity, we decided to avoid any voltage conversion independent from the \glref{PDCU}. Therefore, our \glref{OBC} requires two separate inputs from the main power buss ($3.3\rm{[V]}$ and $5\rm{[V]}$). The \glref{OBC} is connected to each of these inputs through an electronic fuse (\glref{eFuse}). This device continuously monitors the bus for events of under-voltage, over-voltage, and over-current\fnote{This is a crucial feature in handling and resolving a latch-up event.}. As a response to such an event, the \glref{eFuse} will switch into high impedance and pull down specific input of an AND logic gate. This gate simultaneously controls two load switches, one for each power line. This approach ensures that a fault on one power bus will result in a high impedance of both \glref{OBC} power inputs. It also eliminates the risk of a death loop, in which a reset of \glref{eFuse}s is not possible as they are switching each other off. Added benefits of this design are an inbuild current measurement and a Kill Switch integration into the logic gate. A summary of the final power management rating is listed in table \ref[tab:powerManagement]. These values were chosen considering the power requirements of the remaining \glref{OBC} components and are a subject of change by a future user.

\midinsert \clabel[tab:powerManagement]{Power management rating}
    \ctable{lccccc}{
        Power input                 & Parameter & Min & Typ & Max & Unit        \crl
        \vspan2{$3.3\rm{[V]}$ BUS}  & voltage   & 2.9 & 3.3 & 3.5 & $\rm{V}$    \cr
                                    & current   & 0.0 & -   & 1.2 & $\rm{A}$    \crl
        \vspan2{$5\rm{[V]}$ BUS}    & voltage   & 4.6 & 5.0 & 5.4 & $\rm{V}$    \cr
                                    & current   & 0.0 & -   & 1.2 & $\rm{A}$
    }
    \caption/t \glref{OBC} power rating. Value out of range will cause a protective shutdown.
\endinsert

\label[secc:powerManagementSch]
\secc Schematic design

\quad An actual schematic diagram of power management circuitry is shown in figure \ref[sch:powerManagement]. The most important part of this design is the \glref{eFuse}, as it covers all of the power control features. We decided to use the TPS25940-Q1 device \cite[dat:efuse]. Custom  threshold values can be set following the typical application schematic in the datasheet \cite[dat:efuse]. This is done by connecting specific resistors, with values calculated using the TPS2594x design calculation tool \cite[app:tps2594Calc]. As the logic AND gate, we chose the 74LVC1G11-Q100 \cite[dat:gate]. This device is designed to operate in a mixed logic level environment, what corresponds with our application. The last important component is the load switch. In our case, the TPS22965W-Q1 with an inbuilt output discharge function \cite[dat:switch]. For a correct operation of the switch, the {\it VBIAS} pin should stay saturated for a while after disconnecting the {\it VIN} voltage. We achieved this behavior by charging a capacitor connected to the {\it VBIAS} from the {\it VIN} through a Schottky diode. Four reservoir capacitors are placed on both sides of the load switches, following the suggestions in both datasheets \cite[dat:efuse, dat:switch]. Nominal logic values of all switching signals are set by pull-up or pull-down resistors.

\midinsert
    \clabel[sch:powerManagement]{Power management schematic}
    \picw=0.95\hsize \cinspic figures/schemes/powerManagement.pdf
    \caption/f Schematic diagram of power management circuitry.
\endinsert

\secc PCB design

\quad Location and fanout of the power management circuitry are shown in figure \ref[pcb:powerManagement].  All of the power management components are placed on the top side of the \glref{PCB}. Similarly, all of the power tracks are on the top copper layer. Only a few signal traces are running within the second or the bottom layer. Both the $3.3\rm{[V]}$ and $5\rm{[V]}$ control circuitry share the same layout and routing regarding recommendations presented in the \glref{eFuse} and load switch datasheets \cite[dat:efuse, dat:switch]. The $3.3\rm{[V]}$ part is situated closer to the main PC104 header, while the $5\rm{[V]}$ part is right below it. Both \glref{eFuse}s are connected to the main PC104 header power pins through strengthened $0.77\rm{[mm]}$ traces in the third copper layer. Considering a standard copper thickness of $35\rm{[\mu m]}$, each of the input traces is rated to deliver up to $2\rm{[A]}$ of current. The logic gate with associated pull-down resistors is located above all of the remaining circuitry, closest to the main PC104 header.

\midinsert
    \clabel[pcb:powerManagement]{Power management circuitry}
    \picw=0.85\hsize \cinspic figures/pcbs/powerManagement.pdf
    \caption/f Highlighted location of power management circuitry.
\endinsert



%%%%% PERIPHERAL ISOLATORS

\sec Peripheral isolators

\quad The spacecraft \glref{OBC} is connected to many data buses shared among all other CubeSat submodules. In some scenarios, the \glref{OBC} must be able to isolate itself from a specific or multiple data buses. For example, to switch between \glref{OBC}s in a redundant configuration, to handle a failure on a data bus, or to prevent unintentional interference. Standard approaches to address this feature are based on using analog switches \cite[the:wurzburg], optocouplers \cite[pap:satelliteLeo], or \glref{FPGA}s \cite[obc:dataPatterns]. Furthermore, these isolators should also guarantee that all data lines are in a high impedance state when the \glref{OBC} is powered off.

After a brief survey, we decided to implement the design using robust analog switches. This approach is more straightforward, less expensive, and requires a smaller \glref{PCB} area than the optocoupler or the \glref{FPGA}-based ones. A functional diagram of the implemented circuity is shown in figure \ref[dia:peripheralIsolators]. All of the \glref{OBC} data lines are connected to the rest of the spacecraft through a series of analog switches. These data lines are grouped by particular data buses and are assigned to a separate switch. The \glref{OBC} can enable or disable a specific switch and therefore isolate a particular data bus from the remaining spacecraft subsystems. Pulling low the Kill Switch will result in high impedance of all switches and completely isolating the \glref{OBC} data lines.

\midinsert
    \clabel[dia:peripheralIsolators]{Peripheral isolators diagram}
    \picw=0.8\hsize \cinspic figures/diagrams/peripheralIsolators.pdf
    \caption/f Functional diagram of peripheral isolators circuitry.
\endinsert

\secc Schematic design

\quad Schematic diagrams of two isolators are shown in figure \ref[sch:peripheralIsolators]. We decided to use the DGQ2788A device \cite[dat:isolator]. The \glref{OBC} hosts fifteen of these analog switches in a dual double pole double throw (\glref{DPDT}) configuration. The \glref{OBC} data lines are connected to common terminals ({\it COM}). Normally closed terminals ({\it NC}) are left floating, whereas normally open terminals ({\it NO}) terminals are connected to the spacecraft data lines. The important Kill Switch functionality is implemented using the device's power down protection. If the switch loses power, it will enter the normal state. This approach simplifies the circuitry a lot as it substitutes an otherwise necessary system of multiple logic gates. Other beneficial features of this analog switch are a high latch-up current of $300\rm{[mA]}$ and inbuild signal clamping. The device will clamp all of the signals exceeding its supply voltage by internal diodes. As the \glref{MCU} pins connected to these analog switches are $5\rm{[V]}$ tolerant, we chose to power the switches from the $5\rm{[V]}$ power bus. A malfunction could be caused by the switch's enable terminals ({\it EN}), as they do not include internal pull-up or pull-down resistors. We decided to use external pull-down ones to avoid the enable signals' floating state and avoid unintentional device switching. The decoupling capacitors were placed accordingly to the device datasheet. 

\midinsert
    \clabel[sch:peripheralIsolators]{Peripheral isolators schematic}
    \picw=0.92\hsize \cinspic figures/schemes/peripheralIsolators.pdf
    \caption/f Schematic diagram of two peripheral isolators.
\endinsert

\label[secc:peripheralIsolatorsPCBDesign]
\secc PCB design

\quad Location and fanout of the isolators circuitry are shown in figure \ref[pcb:peripheralIsolators]. Sixty separate signals are running from the \glref{MCU} pins through analog switches up to assigned pins in the PC104 header. Hence this part of the \glref{PCB} was the most challenging to design. The switches are placed in two main rows, each on one side of the \glref{PCB} (only two switches are not aligned). The position of every switch was determined by its assigned \glref{MCU} and PC104 header pins. It took several iterations to find out the current layout. The routing network is quite dense, using all three copper signal layers. To accommodate all of the signal traces, the standard signal trace width was decreased from $200\rm{[\mu m]}$ to $173\rm{[\mu m]}$. This new value was acquired as a maximal width possible to squeeze three traces between two pins of the PC104 header. To ensure \glref{CAN} buses signal integrity, we addressed a length matching of its differential pairs. As a finishing step of the routing, lengths of separate \glref{CAN} traces were measured and tuned with serpentine patterns.

\midinsert
    \clabel[pcb:peripheralIsolators]{Peripheral isolators circuitry}
    \picw=0.85\hsize \cinspic figures/pcbs/peripheralIsolators.pdf
    \caption/f Highlighted location of peripheral isolators circuitry.
\endinsert



%%%%% EXTERNAL MEMORY

\sec External memory

\quad One of the \glref{OBC}'s important tasks is collecting and processing data from other submodules present in the spacecraft. As this payload and housekeeping communication might generate a wide data flow, external memory storage is usually dedicated to the \glref{MCU} \cite[obc:dataPatterns, obc:isis, obc:imt, obc:gos, obc:pumpkin, obc:gauss, obc:aac, obc:anteleope, obc:nanosatpro, obc:gom, obc:satbus]. This secondary memory should be writable, preferably fast, and non-volatile (can retain the stored information after power off) to support quick buffering and permanent data storage. Commonly used semiconductor memory technologies matching the criteria are EPROM, EEPROM, FLASH, and various NVRAMs. 

A great threat to memory storage is space radiation, as presented in section \ref[chap:radiationAndRedundancy]. In a single event upset (SEU), the storage node charge in memory is upset due to particle strike and change its logic level \cite[pap:radiationSurvey]. This effect results in random bit-flips throughout the memory address space, corrupting the stored information. Approaches handling this damage are based on i) more durable or even rad-hard memory \glref{IC}s, ii) triple modular redundancy, and iii) error detecting and correcting codes. As the last approach is a software implementation, only the first two could be used in our design.

The EPROM and EEPROM memories are not a good candidate for the \glref{OBC}, as they proved to be more sensitive to gamma radiation \cite[pap:memoryDamage]. The same applies to MRAM, as these memories are generally sensitive to single event effects (SEE) \cite[pap:mramSurvey]. On the other hand, the cell of fairly popular FLASH devices proved to be robust against SEU because more energy is required to change the state of a bit. FLASH memories are more tolerant to radiation and are a good candidate for vital data and code \cite[boo:framInSpace]. F-RAM is a technology that combines the best of Flash and SRAM. It offers faster writes, high read-write cycle endurance, and very low power consumption \cite[pap:ramSurvey]. Moreover, the F-RAM memories have excellent tolerance to radiation \cite[boo:framInSpace]. Considering the listed aspects of radiation effects rejection, together with speed and power consumption, we decided to include both FLASH and F-RAM memories in our \glref{OBC} design. The final parameters of the external memory subsystems are listed in table \ref[tab:externalMemory].

\midinsert \clabel[tab:externalMemory]{External memory parameters}
    \ctable{llll}{
        Subsystem   & Size              & Max. clock        & SPI connection        \crl
        FLASH       & 3x $32\rm{[MB]}$  & $133\rm{[MHz]}$   & SIO, DIO, QIO, QPI    \cr
        F-RAM       & 3x $2\rm{[Mb]}$   & $25\rm{[MHz]}$    & SIO, QIO
    }
    \caption/t Parameters of the external memory subsystems.
\endinsert

\secc Schematic design

\quad Schematic diagrams of FLASH (left) and F-RAM (right) memory subsystems are shown in figure \ref[sch:externalMemory]. Triple modular redundancy is achieved in both of them by connecting three of the same memory \glref{IC}s to one shared data bus. The \glref{MCU} can separately enable or disable a particular \glref{IC} throughout its chip select (\textit{CS}) pin. External pull-up resistors added to these selection lines set the disable mode as a default one. 

As a FLASH memory, the S25FL256L device was selected \cite[dat:flash]. The \glref{SPI} interface of this $32\rm{[MB]}$ NOR memory (also available in $16\rm{[MB]}$ version) can operate in single, dual, or four-bit wide signal lines and also supports the quad peripheral interface (\glref{QPI}) commands. For the F-RAM device, we chose the CY15B102Q \cite[dat:fram]. This $2\rm{[Mb]}$ memory supports a dual or four-bit wide \glref{SPI} interface. The $100\rm{[nF]}$ decoupling capacitors connected to the power inputs of the memory \glref{IC}s were suggested by both devices datasheets. Additionally, a test point was placed on each signal line of both \glref{SPI} data buses. This addition may improve the potential debugging of these subsystems.

\midinsert
    \clabel[sch:externalMemory]{External memory schematic}
    \picw=1\hsize \cinspic figures/schemes/externalMemory.pdf
    \caption/f Schematic diagram of external memory circuitry. 
\endinsert

\secc PCB design

\quad Locations and fanout of external memory circuitry are shown in figure \ref[pcb:ExternalMemory]. Devices of the FLASH subsystem are located at the top side of the \glref{PCB}, below the \glref{MCU}. Devices of the F-RAM subsystem are located at the bottom side of the \glref{PCB}, right from the \glref{MCU}. This device is the only one in the entire \glref{OBC} design with a non-fine pitch package. All of the \glref{SPI} test pads are placed from the bottom side of the \glref{PCB}.

\midinsert
    \clabel[pcb:ExternalMemory]{External memory circuitry}
    \picw=0.85\hsize \cinspic figures/pcbs/externalMemory.pdf
    \caption/f Highlighted location of external memory circuitry.
\endinsert



%%%%% CAN BUS DRIVERS

\sec CAN bus drivers

\quad The high-speed SpaceCAN is considered a primary control and monitoring bus of a LibreCube spacecraft. Therefore it was required to ensure its full support by the \glref{OBC}. An external \glref{CAN} transceiver is usually added to a microcontroller, as its internal physical layer has only limited properties or does not even exists. The separate transceiver provides a stable and reliable physical environment. In our case, the \glref{MCU}'s BxCAN is compatible with both 2.0A and 2.0B \glref{CAN} specifications with a bit rate up to $1\rm{[MB\,s^{-1}]}$ \cite[dat:mcu]. As the \glref{MCU}'s \glref{CAN} drivers are equipped with the 2nd network layer only, an external transceiver implementation to our design was required.

\secc Schematic design

\quad A schematic diagram of implemented \glref{CAN} driving circuitry is shown in figure \ref[sch:canBusDrivers]. As the \glref{MCU} supports two independent \glref{CAN} busses, we had to accommodate each of them. For the \glref{CAN} transceiver, we decided to use a TCAN1051V-Q1 device \cite[dat:canDriver]. The \glref{Rx/Tx} lines of the \glref{MCU} are connected to the device with a $22\rm{[\Omega]}$ terminating resistors. A test point is also present on each of these lines to assist potential debugging. This version of the device comes with a level shifting feature. Different voltage levels on \glref{CAN} and \glref{Rx/Tx} sides are supported. Since the SpaceCAN is a $5\rm{[V]}$ bus and the \glref{MCU} operates at $3.3\rm{[V]}$, we set the device's power levels accordingly. As recommended in the device datasheet, decoupling capacitors were added to power pins. Considering the \glref{CAN} bus's importance, we expect it to stay active straight from powering on the \glref{OBC}. To save some complexity and \glref{MCU} pins,  we decided to ignore the option of controlling the device standby mode. A pull-down resistor on the {\it STBY} pin forces an active mode.

A well-designed \glref{CAN} network usually contains a terminating resistor, filtering, and a transient \& \glref{ESD} protection. To correctly implement these optional features, we followed application reports \cite[app:canDesign, app:canChokes]. Instead of a simple $120\rm{[\Omega]}$ terminating resistor, we chose a more advanced terminating node. A difference is in added filtering as the node consists of two $60\rm{[\Omega]}$ series resistors connected to a \glref{GND} through a $4.7\rm{[nF]}$ capacitor. Furthermore, $100\rm{[pF]}$ filtering capacitors were added to signal lines. Recognizing the suggestions in the reports, we did not include any common-mode chokes or \glref{ESD} protection in our design. Since our \glref{OBC} is not intended to operate near heavy machinery, no extra improvement of susceptibility to electromagnetic disturbance or \glref{EMC} is required.

\midinsert
    \clabel[sch:canBusDrivers]{CAN drivers schematic}
    \picw=0.62\hsize \cinspic figures/schemes/canDrivers.pdf
    \caption/f Schematic diagram of CAN bus driving circuitry.
\endinsert

\secc PCB design

\quad Location and fanout of the \glref{CAN} bus driving circuitry are shown in figure \ref[pcb:canDrivers]. The transceivers are placed on the \glref{PCB} top side in two different locations. Prioritizing the \glref{PCB} surface's optimal usage,  we were unable to keep the devices in a mutual area. The transceiver circuitry layout follows suggestions in the datasheet \cite[dat:canDriver] and the application report \cite[app:canDesign]. This layout is the same for both devices. As mentioned in section \ref[secc:peripheralIsolatorsPCBDesign], serpentine patterns were used to match trace lengths of particular differential pairs.

\midinsert
    \clabel[pcb:canDrivers]{CAN drivers circuitry}
    \picw=0.85\hsize \cinspic figures/pcbs/canDrivers.pdf
    \caption/f Highlighted location of CAN bus drivers circuitry.
\endinsert



%%%%% TEMPERATURE MONITORING

\label[chap:temperatureMonitoring]
\sec Temperature monitoring

\quad Temperature is a critical parameter in the space environment, and all electronic components are sensitive to its variation. Any increase in temperature may reduce their lifespan and even result in irreversible damage. Such a temperature increase can be evoked by an ambient temperature or by the component's activity. Most electronic components are designed to dissipate heat into the ambient air. This is problematic in the space due to its lack of an atmosphere. Instead, a component transfers heat into the \glref{PCB}'s thermal capacity.\fnote{And then dissipated into the spacecraft environment by the thermal radiation \cite[pap:tempModeling].} This conduction can produce temperature gradients throughout the \glref{PCB} and influence other components. Furthermore, a change in a component's temperature can indicate its malfunction. Sudden temperature increase is a common sign of a latch-up event \cite[app:badBoys]. Accordingly, the \glref{OBC} temperature is a valuable part of a spacecraft telemetry and worthy of monitoring. Professional manufacturers have also included temperature sensors into their \glref{OBC} \cite[obc:dataPatterns, obc:isis, obc:imt, obc:gauss, obc:gom, obc:satbus].

Standard temperature sensor technologies include integrated circuit (\glref{IC}) sensors, thermistors, resistance temperature detectors (\glref{RTD}s), and thermocouples. Their key features are compared in the guide to temperature sensing \cite[boo:temperature]. The \glref{IC} sensors appeared to be the best choice for our design challenge. These sensors are typical for their good accuracy, small footprint, easy complexity, and excellent linearity. A processing unit can usually communicate with the \glref{IC} sensors using one shared data bus and receive ready-to-use temperature data. In contrast, the implementation of the other sensor technologies requires extra analog components and circuitry. For example, amplifiers and \glref{ADC}s for thermistors and thermocouples or precise current sources and \glref{ADC}s for \glref{RTD}s. These non-\glref{IC} technologies would also use multiple \glref{MCU} pins and require additional calibration and shielding. As our goal was to design a compact PCB and a simple system, we decided to implement the \glref{IC} sensors approach.

\secc Schematic design

\quad A schematic diagram of a temperature sensor network is shown in figure \ref[sch:temperatureMonitoring]. After a brief survey, we selected a MCP9804 device \cite[dat:temperature]. This temperature sensor has an accuracy of $\pm0.25\rm{[{^\circ C}]}$ and communicates through an $\rm{I^2C}$ interface. The device offers eight different $\rm{I^2C}$ addresses, selected by different logic levels on three slave address pins ({\it A0-A2}). Thanks to this feature, we were able to use only one $\rm{I^2C}$ bus for all of the devices. The final address assignment to particular sensors is listed in table \ref[tab:temperatureMonitoring]. As the temperature monitoring is continuous, we decided not to use the device's inbuild alter function. Placement of a decoupling capacitor and $10\rm{[k\Omega]}$ pull-up resistors on $\rm{I^2C}$ lines were suggested by the device datasheet. It is worth mentioning that this sensor supports a low-power standby mode, accessible through a special register.

\midinsert \clabel[tab:temperatureMonitoring]{Temp. sensors description}
    \ctable{cclcc}{
        Designator  & \mspan2[l]{Targeted component}    & Slave addr.   & \glref{$\rm{I^2C}$} addr. \crl
        TS1         & M2  & - Flash memory              & $000$         & $0\rm{x}18$       \cr
        TS2         & U1  & - \glref{MCU}, central west & $100$         & $0\rm{x}1C$       \cr
        TS3         & EF2 & - $5\rm{[V]}$ eFuse         & $001$         & $0\rm{x}19$       \cr
        TS4         & U1  & - \glref{MCU}, north east   & $110$         & $0\rm{x}1E$       \cr
        TS5         & U3  & - \glref{CAN}2 driver       & $010$         & $0\rm{x}1A$       \cr
        TS6         & EF1 & - $3.3\rm{[V]}$ eFuse       & $011$         & $0\rm{x}1B$       \cr
        TS7         & U2  & - \glref{CAN}1 driver       & $111$         & $0\rm{x}1F$       \cr
    }
    \caption/t List of temperature sensors location and addresses.
\endinsert

Accordingly to the survey on CubeSat electrical bus reliability \cite[pap:reliability], the $\rm{I^2C}$ interface is the most likely to fail. Over half of investigated spacecrafts experienced at least one $\rm{I^2C}$ lockup\fnote{A continuous busy state of the $\rm{I^2C}$ bus, where is the master prevented from starting a new transaction.}. Hence it was essential to apply measures assuring the proper functionality of our $\rm{I^2C}$ based temperature monitoring network. We had to implement a mechanism that can either prevent the lockup from occurring or is capable of resolving it. We chose the second option, as we consider it as a more robust and reliable. A simple but efficient approach to resolve an $\rm{I^2C}$ lockup is to reset the power of all its slave devices. For this purpose, we implemented the TPS22965W-Q1 load switch in the very same configuration as we have used in the power management circuitry (subsection \ref[secc:powerManagementSch]).

\midinsert
    \clabel[sch:temperatureMonitoring]{Temp. monitoring schematic}
    \picw=1.0\hsize \cinspic figures/schemes/temperatureMonitoring.pdf
    \caption/f Schematic diagram of temperature monitoring circuitry.
\endinsert

\secc PCB design

\quad The layout of the temperature monitoring circuitry is dependent on the other subsystems. Therefore, it had to be implemented as the very last one. Each of the \glref{IC} sensors was assigned to monitor a temperature of a particular device from another subsystem. These monitored devices were carefully chosen to cover all of the \glref{OBC}'s most critical components. These are various \glref{IC}s used in the other subsystems as they execute multiple tasks, produce most of the heat, and are vulnerable to latch-up events. A listing of the monitored devices and their assigned temperature sensors is stated in table \ref[tab:temperatureMonitoring].

As we cannot exceed a total number of eight \glref{IC} temperature sensors (MCP9804 devices) at one $\rm{I^2C}$ bus, we decided to monitor only one of the three FLASH devices. We selected the one in the middle as a good approximation of the two remaining FLASH devices. None of the three F-ram devices is monitored directly as the opposite side of the \glref{PCB} in their location contains a dense layout of the power management circuitry. However, the F-ram devices are neighbored by three temperature sensors: TS5, TS6, and TS3. Measurements obtained from these sensors should be sufficient to monitor the F-ram devices. The \glref{MCU} contains an inbuilt temperature sensor suitable only for applications that detect temperature changes only \cite[dat:mcu]. We decided to monitor the \glref{MCU} with a pair of \glref{IC} sensors as we prefer to measure the absolute temperature.

To ensure correct temperature measurement, we have to create a sufficient thermal bridge between the temperature sensor and its targeted device. A common approach is to place the sensor on the other side of the \glref{PCB}, right opposite the device. A thermal bridge is then created using a set of \glref{PCB} vias. This method is recommended and described in the temperature sensors guideline for \glref{SMD}s \cite[app:temperature]. Its illustration for our use case is shown in figure \ref[dia:temperatureMonitoring]. It is worth mentioning that the added vias help dissipate the heat into the \glref{PCB} and are usually required by the device's datasheet.

\circleparams={\ratio=1 \fcolor=\White \lcolor=\Black \hhkern=0.9pt \vvkern=0.9pt}

\midinsert
    \clabel[dia:temperatureMonitoring]{Thermal bridge illustration}
    \picw=0.8\hsize \cinspic figures/diagrams/temperatureMonitoring.pdf
    \caption/f Illustration of a thermal bridge between the temperature sensor and WSON package (left) or LQFP package (right). Legend: {\typosize[8/]\incircle{⟨1⟩}} 6-layer \glref{PCB}, {\typosize[8/]\incircle{⟨2⟩}} targeted device, {\typosize[8/]\incircle{⟨3⟩}} \glref{IC} temperature sensor, {\typosize[8/]\incircle{⟨4⟩}} measurement die, {\typosize[8/]\incircle{⟨5⟩}} via, {\typosize[8/]\incircle{⟨6⟩}} thermal pad, {\typosize[8/]\incircle{⟨7⟩}} epoxy resin.
\endinsert

Location and fanout of the temperature monitoring circuitry are shown in figure \ref[pcb:temperatureMonitoring]. All \glref{IC} sensors are placed at the bottom side of the \glref{PCB}, directly under their targeted devices. The load switch is located at the \glref{PCB}'s west-central part. The switch itself and most of its auxiliary components are placed on the top side of the \glref{PCB}. The \glref{$\rm{I^2C}$} bus signal traces and separate power bus runs between all of the sensors.

\midinsert
    \clabel[pcb:temperatureMonitoring]{Temp. monitoring circuitry}
    \picw=0.85\hsize \cinspic figures/pcbs/temperatureMonitoring.pdf
    \caption/f Highlighted location of temperature monitoring circuitry.
\endinsert



%%%%% DEBUG CONNECTOR

\sec Debug connector

\quad A host/target interface is required to program or debug the \glref{MCU}. This interface is made of three components. A hardware debug tool, an \glref{SWJ} connector and a cable connecting the host to the debug tool \cite[app:getStart]. As the \glref{SWJ} connector is a part of the target board, we had to implement it into our \glref{PCB} design. The MCU has an embedded \glref{SWJ-DP} interface that enables either a serial wire debug (\glref{SWD}) or a \glref{JTAG} probe to be connected to the target \cite[dat:mcu]. Considering that the \glref{SWD} is required by the ST-LINK, is performed using two pins (five needed for the \glref{JTAG}), and is the preferred debug port \cite[app:debugToolbox], we decided to implement the \glref{SWD} only. This decision slightly decreased the \glref{PCB} complexity and freed some \glref{MCU} pins for further usage. Also, the Sierra Board was not created with a mass production indent, the primary use case of the \glref{JTAG} debug.

The implemented \glref{SWD} connector includes not only the two mandatory SWDIO and SWCLK pins. An NRST signal is also presented, enabling a connection under reset. This signal is required to take back the control of the board when it is not responding \cite[app:debugToolbox]. Also, a serial wire output (SWO) is available, providing an asynchronous serial communication channel. It has to be used in combination with a serial wire viewer (SWV) \cite[app:debugToolbox]. Additionally, one UART channel is also presented, providing a convenient communication option for the development and testing process. Pin assignment of the ST-LINK compatible connector of the Board Sierra is shown in figure \ref[mod:programPort].

\midinsert
    \clabel[mod:programPort]{Debug connector pinout}
    \picw=0.45\hsize \cinspic figures/model/programPort.pdf
    \caption/f Debug connector pinout.
\endinsert

\secc Schematic design

\quad The programing connector is intended to be in direct touch with external hardware and human manipulation. As suggested in the application report \cite[app:badBoys], we decided to protect the associated signals against the \glref{ESD}. A \glref{TVS} diode array SP3012-06 \cite[dat:esd] was selected, as it provides the exact pin count as needed. All of the debugging \glref{MCU} signals were routed through this low-capacitance \glref{ESD} protection device. A schematic diagram of the circuitry is shown in picture \ref[sch:programPort]. Additionally, a $22\rm{[\Omega]}$ terminating resistor is connected in series to each signal. A $100\rm{[\Omega]}$ resistor and a ferrite bead are connected to the connector's power pin. Their purpose is to limit the transient current and suppresses high-frequency electronic noise that may affect the \glref{OBC} from an attached debugger.

\midinsert
    \clabel[sch:programPort]{Debug connector schematic}
    \picw=0.5\hsize \cinspic figures/schemes/programPort.pdf
    \caption/f Schematic diagram of debug connector circuitry.
\endinsert

\secc PCB design

\quad Location and fanout of the debug connector circuitry are shown in figure \ref[pcb:programPort]. A small $4$x$2$ female header with a $1.27\rm{[mm]}$ pitch was selected as a debug connector. Its placement near the edge of the PCB in a cutout area should improve its accessibility and cable management. This \glref{SMD} connector is located on the \glref{PCB}'s top side, while the remaining electronics parts are placed below it, from the bottom side of the \glref{PCB}.

\midinsert
    \clabel[pcb:programPort]{Debug connector circuitry}
    \picw=0.85\hsize \cinspic figures/pcbs/programPort.pdf
    \caption/f Highlighted location of debug connector circuitry.
\endinsert



%%%%% PAYLOAD SECTOR

\label[chap:sierra_payloadSector]
\sec Payload sector

\quad As mentioned in section \ref[chap:payloadSector], only one-fourth of the entire PCB surface is occupied by the OBC circuitry. The remaining space is meant to be used as a user-defined payload sector. After discussing with the VST supervisors, we utilized this sector as a universal prototyping board. We implemented this feature to illustrate the purpose of this sector - to accommodate any user-specific circuitry. We expect that a future user of the Board Sierra would replace the universal board with his mission-specific circuitry.

The design of the universal prototyping board is visible in figures \ref[app_vis:sierraTop] and \ref[app_vis:sierraBottom]. Its main feature is a grid of $0.85\rm{[mm]}$ wide THT holes with a plating utilized for an SMD soldering. This THT array is surrounded and divided by power bus rails, providing $5\rm{[V]}$, $3.3\rm{[V]}$, and GND connections. This power delivery is sourced directly from the main PC104 header, bypassing the power management circuitry. However, each of the rails is fused by a 1206 packaged ceramic fuse. In the lower part of the prototyping board, two universal SO-24 footprints are available on each PCB side for rapid prototyping. Every pad of these footprints is routed to a small circular pad for easier soldering.
 % skontrolovane

\label[chap:boardDelta]
\chap Board Delta

\quad In this chapter, we address the Board Delta - a double redundant \glref{OBC} module in the PC/104 format. We explain the motivation and expectation behind the redundant systems while providing description and comments on the schematic and \glref{PCB} module design. A 3D visualization of the Board Delta is shown in figure \ref[img:delta_miniature].

\midinsert
    \clabel[img:delta_miniature]{Board Delta 3D render}
    \picw=0.95\hsize \cinspic figures/model/delta_miniature.pdf
    \caption/f 3D render of the Board Delta from the top (left) and the bottom (right) side.
\endinsert



%%%%% MOTIVATION

\sec Motivation and expectation

\quad During their launch and operation, Cubesats have to perform under extreme conditions of the surrounding environment. The most critical impacts were briefly described in the previous chapters, namely: i) damage of semiconductor devices caused by radiation (section \ref[chap:radiationAndRedundancy]), ii) exposure to wide temperature gradients and problems related to limited heat dissipation, and iii) significant mechanical stress generated by a launcher vehicle (both in section \ref[chap:componentsCertification]). Together with the practically non-existing possibility of maintenance, these problems are the key factors of a high failure rate of the already executed missions. As we have listed in section \ref[chap:existingOBCmodules], every second launched Cubesat has experienced a fatal failure by 2016, from which 20\% were caused by the \glref{OBC}s.

Designing the particular modules with multiple layers of redundancy is a common method of increasing the spacecraft's overall reliability \cite[pap:failRate2016]. \"By implementing onboard redundancy, CubeSats can meet or exceed their mission life, providing additional science data and post-mission payload testing." \cite[pap:approachToSpace] Together with telecommunications, attitude determination, and electrical power, the \glref{OBC} is one of the critical modules with high requirements for redundancy \cite[pap:satelliteLeo]. Some of the professional \glref{OBC} manufacturers have also included elements of redundancy into their designs \cite[obc:dataPatterns, obc:isis, obc:imt, obc:gauss, obc:nanosatpro]. Interestingly, the fairly popular CubeSat \glref{OBC} Kit from Pumpkin \cite[obc:pumpkin] is not one of them and its non-redundant architecture was pointed out as one of the two major weaknesses \cite[pap:reliability].

Although the Board Sierra implements some features of redundancy (triple external memories and dual peripheral data busses), it is still a single OBC module. In a case of permanent damage to the power management or the \glref{MCU}, the Cubesat's mission would be severely jeopardized. However, this scenario can be eliminated by including another \glref{OBC} into the module design. Each of these two \glref{OBC}s is fully independent and has its own circuitry. Such a double redundant \glref{OBC} module can simply switch between the two \glref{OBC}s if one of them undergoes a serious malfunction. This approach was, for example, implemented in the professional DP-OBC-0402 by Data Patterns \cite[obc:dataPatterns].



%%%% DOUBLE REDUNDANT DESIGN

\sec Double redundant design

\quad From the beginning of the VST104 project, we thought about the Board Delta as a hardware merge of the two Board Sierras. The intention was to share the same OBC design between the two redundant \glref{OBC}s at the Board Delta and also with the \glref{OBC} on the Board Sierra. This approach of the only one \glref{OBC} design has several advantages. Firstly, the two identical \glref{OBC}s on the Board Delta can run almost the same software and provide the same functionality to the spacecraft. After an emergency switch to the other \glref{OBC}, the Cubesat can continue to operate normally without changing the mission plan. Secondly, it allows an easy project migration between the boards. A potential user can develop and test his/her setup on the cheaper Board Sierra and then move to the more expensive Board Delta. Lastly, the approach also simplifies the development process and future maintenance of the VST104 project. It is faster and cheaper to develop and test out only one \glref{OBC} design and then integrate it into different modules. Also, if an \glref{OBC} design flaw or a possible improvement is found on one of the modules, it can be automatically implemented to the remaining modules.



%%%%% SCHEMATIC DESIGN

\secc Schematic design

\quad As the idea was to use the already existing OBC design (described in chapters \ref[chap:boardSierra] and \ref[chap:boardSierraSubsystems]), the creation of the Board Delta schematic was straightforward. The KiCad project of the Board Delta consists of one primary sheet and two sub-sheets. The primary sheet contains the PC/104 header with assigned global signals, while each sub-sheet includes one \glref{OBC} design copied from the Board Sierra. Power and peripheral input/outputs of these so-called Left and Right \glref{OBC}s are attached to the same global signals as the PC/104 header. The resulting configuration of the module is simple: peripheral isolators and inputs of the power managements are connected directly to the PC/104 header. The individual peripheral isolators and Kill Switch function of the power management are then used to isolate and power down one of the \glref{OBC}s.

Although we have stated multiple times that the \glref{OBC}s share the same design, changes to a few PC/104 header pin assignments were required. The watchdog signal {\tt CPU_WD_1} and the Kill Switch signal {\tt GLO_KS_1} were assigned to the Left \glref{OBC}, whereas their {\tt CPU_WD_2} and {\tt GLO_KS_2} variants were assigned to the Right \glref{OBC}.

It is also important to note that our design does not exclude the possibility of running both of the \glref{OBC}s simultaneously. In such a scenario, both \glref{OBC}s should be powered on and negotiate the use of the shared PC/104 busses. The OBC not using a particular bus should isolate itself from it. The simultaneous operation can be beneficial in missions where significant amounts of data need to be processed quickly. Also, some Cubesats may require one primary \glref{OBC}, handling the housekeeping and mission control, together with a secondary \glref{OBC}, serving payloads such as scientific instruments or cameras.



%%%%% PCB DESIGN

\secc PCB design

\quad The decision to use only one \glref{OBC} design does not apply exclusively to the schematic design but also to the \glref{PCB} design. The same (or as similar as possible) \glref{PCB} layout and routing of the \glref{OBC}s would guarantee comparable mechanical, electrical, and thermal characteristics. As the desire to create the Board Delta was clear from the beginning, we count it into the design of the Board Sierra \glref{PCB}. Its \glref{OBC} was designed exclusively on the PC/104 module northwest quarter, leaving the northeast quarter empty.

We used several tricks to ensure the best possible similarity between the \glref{OBC}s layouts. As a building template of the Board Delta \glref{PCB}, we copied the final design of the Board Sierra. The \glref{OBC} design on the template was assigned to the Left \glref{OBC}. The layout for the Right \glref{OBC} was generated using the Replicate layout plugin (mentioned in section \ref[chap:kicadPlugins]). This plugin was capable of copying the entire \glref{OBC} design, including all footprints, traces, and vias. However, to match the alignment and pinout of the PC/104 header, we had to flip this copied layout. This changed the previous top and bottom sides of the \glref{OBC} and is why the \glref{MCU}s are each on a different \glref{PCB} side. 

Some manual adjustments and routing were although required to finish the \glref{PCB} design. The second and third inner copper layers were manually reversed back to meet the copper layers assignment from section \ref[chap:pcbCharacteristics]. Also, the debug connector and its circuitry were mirrored back and rerouted. Thanks to this adjustment, both debug connectors are located on the \glref{PCB} top side with the same orientation and pinout. However, the most challenging part was to manually trace all peripheral isolators to their required PC/104 pins. Multiple attempts and sacrifice of the northern edge cutout were required to accomplish so. The complex layout and routing of both \glref{OBC}s are shown in figure \ref[img:delta_kicad_cut]. The Left \glref{OBC} is located on the left side (northwest quarter), while the Right \glref{OBC} is located on the right side (northeast quarter) of the Board Delta.

\midinsert
    \clabel[img:delta_kicad_cut]{Board Delta KiCad}
    \picw=1\hsize \cinspic figures/pcbs/delta_kicad_cut.pdf
    \caption/f PCB design of the Board Delta captured directly in the KiCad environment. Visible is only the upper part with the PC/104 header and both \glref{OBC}s.
\endinsert
 % skontrolovane

%%%%% ELEMENT FOXTROT

\label[chap:elementFoxtrot]
\chap Element Foxtrot

\quad This chapter addresses the design and characteristics of the Element Foxtrot - a FlatSat test bench for the PC/104 format modules with an built-in power supply. Its layout, embedded features, and functional principles are explained in the following sections. A 3D visualization of the Element Foxtrot is shown in figure \ref[app_vis:foxtrotTop].


%%%%% FLATSAT DESIGN

\sec FlatSat design

\quad According to the \glref{VST} desires, the Element Foxtrot fits two use-case scenarios. Like any other FlatSat, it supports the testing and development of new PC/104 modules. Also, it is a tool used for presenting their work. Developed modules can be attached to the FlatSat and then displayed at workshops or conferences. For example, running presented algorithms or software packages in real-time. Thus, we attempted to design the Element Foxtrot to accommodate handy features and have an appealing and professional look. After a discussion with the \glref{VST} supervisors, we came up with its final design. The FlatSat has a rectangular shape and is divided into two parts.

The main part has dimensions of $20.25$ x $26.10\rm{[cm]}$ and hosts four slots for the PC/104 format modules. Silkscreen contours mark the position and orientation of these slots with respect to the LibreCube template (section \ref[chap:PC104standard]). Individual slots headers are electrically connected together, concerning the VST104 pinout (section \ref[chap:mainHeaderPinout]). A large \glref{VST} company logo is displayed below the slots, giving the FlatSat a nice advertisement feature. The purpose of this part is to provide a connection between the mounted PC/104 modules while keeping them next to each other and easily accessible.

The second part of the FlatSat is an built-in power supply. The main task of this $20.25$ x $4.55\rm{[cm]}$ strip is to deliver power to the mounted PC/104 modules. This power supply can be connected to either a $6.4\rm{[mm]}$ barrel jack or \glref{USB-C} power inputs. Then, it delivers a $3.3\rm{[V]}$, $5\rm{[V]}$, and unregulated power lines. We decided to integrate a power source into the FlatSat to have an accessible power-up possibility. During a software testing or a conference, it might be convenient just to plug in an ordinary charger rather than setting up a laboratory power source or a battery-powered \glref{PCDU} module.

The overall dimensions of the FlatSat are $20.25$ x $30.75\rm{[cm]}$, which should be easily fabricable by ordinary \glref{PCB} manufacturers. If desired, the two FlatSat parts can be broken apart from each other and it is possible to use them separately. As the design of the built-in power source is nontrivial, it is addressed in the rest of this chapter.



%%%% POWER SUPPLY FEATURES

\label[chap:powerFeatures]
\sec Power supply features

\quad Besides its main functionality, the power supply also offers multiple extra features. These small additions were designed to create the usage of the Element Foxtrot more convenient and user-friendly. The entities and locations of these features are shown and labeled in figure \ref[img:foxtrot_functions]. We list and describe their functionality in the rest of this section.

\circleparams={\ratio=1 \fcolor=\White \lcolor=\Black \hhkern=0.9pt \vvkern=0.9pt}
\ovalparams={\roundness=5pt \fcolor=\White \lcolor=\Black \hhkern=-4pt \vvkern=-2.5pt}

\midinsert
    \clabel[img:foxtrot_functions]{Foxtrot functionality}
    \picw=1\hsize \cinspic figures/model/foxtrot_functions.pdf
    \caption/f Visualization of the Element Foxtrot's power supply with its features labeled. Legend: {\typosize[8/]\incircle{⟨1⟩}} input selection switch, {\typosize[8/]\incircle{⟨2⟩}} $6.4\rm{[mm]}$ barrel jack connector, {\typosize[8/]\inoval{⟨3a⟩}} \glref{\glref{USB-C}} connector, {\typosize[8/]\inoval{⟨3b⟩}} \glref{USB} data bus header, {\typosize[8/]\incircle{⟨4⟩}} current sensing resistors: {\typosize[8/]\inoval{⟨4a⟩}} $5\rm{[V]}$, {\typosize[8/]\inoval{⟨4b⟩}} $3.3\rm{[V]}$, {\typosize[8/]\inoval{⟨4c⟩}} unregulated, {\typosize[8/]\incircle{⟨5⟩}} disconnection terminal jumpers: {\typosize[8/]\inoval{⟨5a⟩}} $5\rm{[V]}$, {\typosize[8/]\inoval{⟨5b⟩}} $3.3\rm{[V]}$, {\typosize[8/]\inoval{⟨5c⟩}} unregulated, {\typosize[8/]\incircle{⟨6⟩}} breakable connection, {\typosize[8/]\incircle{⟨7⟩}} $4\rm{[mm]}$ banana sockets: {\typosize[8/]\inoval{⟨7a⟩}} $5\rm{[V]}$, {\typosize[8/]\inoval{⟨7b⟩}} $3.3\rm{[V]}$, {\typosize[8/]\inoval{⟨7c⟩}} unregulated, {\typosize[8/]\inoval{⟨7d⟩}} \glref{GND}, and {\typosize[8/]\incircle{⟨8⟩}} indication \& status \glref{LED}s (detailed explanation in section \ref[chap:powerFeatures]).
\endinsert

\begitems
    * {\sbf User control of inputs:} The built-in power supply supports two separate power inputs: a $6.4\rm{[mm]}$ barrel jack or a \glref{\glref{USB-C}} connector. A three-position slide switch (on/off/on) between these two connectors is used to enable one or none of them. When its slider is located in the middle position, the power supply is turned off. When slid to one side, the corresponding power input enables, and the power supply turns on.
    * {\sbf 6.4[mm] barrel jack input:} This power input expects a positive barrel jack polarity. The supported input voltage range is $7$-$14\rm{[V]}$, and the maximum current flow is $8\rm{[A]}$. The power input is embedded with a reversed polarity and overvoltage protection.
    * {\sbf \glref{USB-C} handshake input:} A standard \glref{USB-C} connector is combined with a power delivery controller \glref{IC}, allowing so-called handshaking. The power input can be used in two modes: i) it can handle an ordinary $5\rm{[V]}$ \glref{USB-C} charger, or ii) negotiate a power contract with an intelligent \glref{USB-C} charger. Therefore, the supported input voltage range is $5$ \& $7$-$14\rm{[V]}$, and the maximum current flow is $6\rm{[A]}$.  The differential pair of the \glref{USB} data bus is led to a separate header close to the connector. An independent $5\rm{[V]}$ power bus is also available at this header. This feature allows the use of the \glref{USB-C} connection also for communication purposes.
    * {\sbf Current sensing resistors:} Before leaving the power supply and entering the other part of the FlatSat, the power lines are interrupted with shunt resistors. A 1206 sized \glref{SMD} resistor's footprints are placed in series with those lines, accommodating an built-in current sensing feature. Two test pads are then connected to each resistor, providing a connection terminal for the voltage drop measurement. If not intent to use, the footprints can be populated with $0\rm{[\Omega]}$ resistors or even a blob of solder.
    * {\sbf Disconnection terminal:} A set of removable jumpers can be used to temporarily separate the power supply circuitry from the rest of the FlatSat. Each of them is rated for currents of up to $6\rm{[A]}$. Although removing the jumpers would interrupt the power lines, the ground plane is shared and \glref{GND} would remain connected.
    * {\sbf Breakable connection:} Under some circumstances, it might be needed to remove the power supply from the remaining FlatSat entirely. In such a case, these two parts can be separated by breaking a connection between them. This line consists of multiple long cutoffs and can be broken with a reasonable amount of applied force.
    * {\sbf 4[mm] banana sockets:} A laboratory power supply with its standard banana cables can also be connected to the FlatSat. The $4\rm{[mm]}$ female sockets are placed behind the power supply circuitry. These sockets will remain directly attached to the PC/104 modules power lines even after temporary or permanent separation of the power source. This can be used to bypass the power supply or for voltage measurement purposes. The sockets are available for all power lines together with the \glref{GND}.
    * {\sbf Indication \& status \glref{LED}s:} To increase the user experience, a bunch of colorful light emitting diodes (\glref{LED}s) is spread through the power supply circuitry. Their monitoring purposes (following the labeling in figure \ref[img:foxtrot_functions]) are: 8a - power at the barrel jack input detected (red), 8b - power at the \glref{USB-C} input detected (red), 8c - a power delivery error (yellow), 8d - the power level negotiated successfully (blue), 8f - $5\rm{[V]}$ bus for the \glref{USB-C} data header active (green), and 8e - the power supply active (green).
    * {\sbf Link to another Foxtrot:} In applications where more than four PC/104 modules are required, the PC/104 bus can be connected to another Element Foxtrot. This extension is made through two 45-conductor \glref{FFC} cables. Their two \glref{SMD} connectors are located under the third module of the FlatSat. It is important to note that these cables connect only the PC/104 data busses. The power delivery has to be connected separately, preferably using the $4\rm{[mm]}$ banana sockets.
\enditems



%%%%% POWER SUPPLY CIRCUITRY

\sec Power supply circuitry

\quad A schematic diagram of the power supply circuitry is available at the end of the appendix section. This electronics scheme is rather complex, and it might not be easy to orientate in it quickly. Therefore we explain the circuitry design and functional principles with the help of its hardware diagram, presented in figure \ref[dia:foxtrotArchitecture], instead.

\midinsert
    \clabel[dia:foxtrotArchitecture]{Foxtrot HW architecture}
    \picw=0.924\hsize \cinspic figures/diagrams/foxtrotArchitecture.pdf
    \caption/f Diagram of the Element Foxtrot's power supply hardware architecture.
\endinsert

The central element of the circuitry is its main power bus, labeled as the {\it VSINK}. This bus provides electric energy for the rest of the FlatSat either directly or through the associated \glref{DCDC} converters. This energy can be sourced in three different ways. In other words, the {\it VSINK} can be powered in three modes: i) from the $6.5\rm{[mm]}$ barrel jack connector, ii) from the \glref{USB-C} connector with power delivery, or iii) from the \glref{USB-C} connector without power delivery. These modes are selected by a combined logic of the user switch position and the power delivery controller's handshake status.  

In the first mode, a closed state of switch {\it A} and an open state of switch {\it B} are forced by the user switch position. In the real circuitry, these two switches ({\it A}, {\it B}) are implemented as bi-directional P-channel \glref{MOSFET} switches. The first power mode also accommodates two safety features: overvoltage and reverse polarity protections. The overvoltage protection is designed to compare the input voltage and open switch {\it A} if it exceeds $15\rm{[V]}$. Its circuitry uses discrete components (such as TL432 voltage reference) and follows a reference design presented in \cite[app:reversePolarity]. The reverse polarity protection is based on a well-known design using a single P-channel \glref{MOSFET} combined with a Zener diode.

The second and third modes share the same setup while powering the {\it VSINK} bus. The user switch forces an open state of switch {\it A} and permits a closed state of switch {\it B}. The associated power delivery controller makes the final decision whether to close or leave switch {\it B} open. In the circuitry design, we have implemented a standalone STUSB4500 controller accordingly to its typical application given in its datasheet \cite[dat:pdController] and the evaluation data-brief \cite[app:steval]. This device attempts to negotiate an appropriate voltage from the $7-14\rm{[V]}$ range with the attached \glref{USB-C} charger. If the charger supports power delivery and the negotiation is successful (mode ii) or if the negotiation is unsuccessful\fnote{The power delivery negotiation (handshake) can fail because of two reasons: the charger i) does not support power delivery, or ii) cannot provide voltage (current) from the required voltage (current) range.} although standard $5\rm{[V]}$ is provided (mode iii), switch {\it B} is closed by a {\it USB OK} signal. The STUSB4500 device can be initialized through the \glref {$\rm{I^2C}$} connection (using the associated header) as described in its programming guide \cite[app:usbcProramm]. We found the approach using the STSW-STUSB003 software library \cite[app:stsw] to be the most convenient.

Once is the {\it VSINK} bus correctly powered, it is required to distribute the power to the FlatSat's separate power lines. All of the PC/104 slots are connected to the required $3.3\rm{[V]}$, $5\rm{[V]}$, and unregulated lines. The unregulated line is attached to the {\it VSINK} bus directly. However,  step-down \glref{DCDC} converters are necessary to decrease the {\it VSINK} voltage for the $3.3\rm{[V]}$ and $5\rm{[V]}$ power lines. For this purpose, the OKL-T/6-W12 devices were selected. Output voltages of both converters are set by multi-rotation trimmers, enabling a fine tuning. Running the converters in the power modes i and ii is straightforward as the {\it VSINK} voltage is guaranteed to be at least $7\rm{[V]}$. This is above the $6.5\rm{[V]}$ minimum required by the $5\rm{[V]}$ converter. Yet, the mode iii is problematic as its {\it VSINK} voltage is only $5\rm{[V]}$ and therefore insufficient. Hence, bypassing the $5\rm{[V]}$ \glref{DCDC} converter in the power mode iii had to be implemented. We accomplished this by including two more P-channel \glref{MOSFET} switches (switches {\it C} and {\it D}) controlled by a {\it bypass} logic. Inputs of this logic are the user switch position and the {\it handshake status} signal from the power delivery controller. If the power from the barrel jack was selected (mode i) or the \glref{USB-C} negotiation was successful (mode ii), the {\it bypass} logic closes switch {\it C} and opens switch {\it D}. However, if the power from the \glref{USB-C} was selected and the negotiation failed (mode iii), the logic opens switch {\it C} and closes switch {\it D}. As mentioned in section \ref[chap:powerFeatures], current sensing resistors (marked as {\it R}) and {\it jumper} terminals are placed between the power supply circuitry and the PC/104 slots.

Inspired by the data-brief of the STUSB4500 evaluation board \cite[app:steval], we decided to include a separate step-down switching regulator. The L7985 device is connected directly to the {\it VSINK} bus and provides a stable $5\rm{[V]}$ output for the \glref{USB} data connection.

Various \glref{ESD} protection diodes, 2AG cartridge fuses, and reservoir capacitors are distributed through the circuitry. For their exact location, refer to the circuitry electronics schematic. A \glref{SPICE}-based analog electronic circuit simulator LTspice was used to simulate and check for the desired functionality of the presented circuitry. This functionality was afterward properly tested on multiple assembled Element Foxtrots. 



%%%%% PCB DESIGN
\sec PCB design

\quad The Element Foxtrot is designed on a four-layer \glref{PCB} with a slightly unconventional copper layer assignment. The two previously mentioned parts of the FlatSat (the power supply part and the main part with four PC/104 slots) are routed with different approaches, sharing only the common ground plane. The final layout and routing of the whole FlatSat captured from the KiCad environment is shown in figure \ref[app_vis:foxtrotKicad]. The location and layout of the power supply's circuitry are shown in figure \ref[img:foxtrot_subsystems].

\circleparams={\ratio=1 \fcolor=\White \lcolor=\Black \hhkern=0.9pt \vvkern=0.9pt}

\midinsert
    \clabel[img:foxtrot_subsystems]{Foxtrot circuitry summary}
    \picw=1\hsize \cinspic figures/model/foxtrot_subsystems.pdf
    \caption/f Location and layout of the Element Foxtrot's power supply circuitry. Legend: {\typosize[8/]\incircle{⟨1⟩}} step-down switching regulator for the \glref{USB} communication, {\typosize[8/]\incircle{⟨2⟩}} barrel jack overvoltage and reverse polarity protection, {\typosize[8/]\incircle{⟨3⟩}}, power delivery controller, {\typosize[8/]\incircle{⟨4⟩}} power source and bypass switching, {\typosize[8/]\incircle{⟨5⟩}} power lines 2AG cartridge fuses, {\typosize[8/]\incircle{⟨6⟩}} tunable \glref{DCDC} converters.
\endinsert

In the power supply part, the top and the first inner copper layers are used for the power routing. Power buses and other power connections are formed from polygons of these layers. A dense rectangular net of vias is used to connect these layers within an individual bus. These power traces (polygons) are scaled to easily withstand a total current delivery of $6\rm{[A]}$. The second inner layer accommodates the common ground plane. All of the power supply electronic components are placed on the \glref{PCB} top side for easier assembly. Therefore the top layer is also used for a majority of the signal traces. If it was not possible to route a trace there, then the bottom layer was used.

The main FlatSat part contains just a few power polygons but a very dense network of signal traces. These traces are placed on the top and the bottom copper layers. The first inner layer supports the power delivery, whereas the second inner layer is the common ground plane. The PC/104 slots are connected in a U shape, providing one continuous data bus avoiding any loops. The silkscreen numbers marking the individual slots are enumerated with respect to this connection. All of the signal traces connecting the slots had to be routed manually as all auto-routing attempts have failed.
 % skontrolovane

%%%%% BOARD SIERRA TESTING

\chap Board Sierra - testing

%%%%% TESTING SOFTWARE

\sec Testing software




%%%%% RADIATION TESTING

\sec Radiation testing




%%%%% EXPERIMENT SETUP

\secc Experiment setup

\quad For the purpose of the radiation testing, the Board Sierra was extended with a couple of electronic sensors: i) an always-on 3D accelerometer and 3D gyroscope LSM6DS3, ii) a high-performance 3-axis magnetic sensors MMC5983MA (two devices), and iii) an integrated 6-axis motion processor with gyroscope and accelerometer MPU6050. All of these sensors were connected to the \glref{OBC} using an \glref{$\rm{I^2C}$} connection.  Each sensor was powered on through a separate NPN transistor controlled by the \glref{MCU}. A proper power reset of each sensor could have been achieved by turning off this transistor and isolating the corresponding \glref{$\rm{I^2C}$} peripheral isolator. For detailed information about the experiment wiring, please refer to the hardware architecture diagram in figure 100.


%%%%% EXPERIMENT RESULTS

\secc Experiment results

Approximately at 5.20[hour] after the beginning of the experiment, the OBC started to malfunction. Until this point, it performed normally without a single error. The total radiation dose at this point sums up to 24.96[krad]. Then, the OBC entered a continuous restart loop of about 100[Hz], managing to send only one UART log per restart. It booted successfully again at 5.27[hod], after 4.2[min] from the first malfunction. After some initial errors and a couple of reboots, it operated well without a single problem until its last log at 5.30[hour]. After that, the UART received dozen of broken characters, and the OBC stopped responding at 25.44[krad].

The first and only sensor that stopped performing normally was the LSM6DS3. After 4.02[hour] from the experiment start, the sensor stopped answering measurement requests. This corresponds roughly to a 19.29[krad] total radiation dose. However, after multiple power resets and few faulty readings, the sensor started to behave well again at 5.30[hour] of the experiment.

 % skontrolovane

% conclusion chapters

%%%%% CONCLUSION

\chap Conclusion

\quad In this thesis, we have contributed to the VST104 project of CubeSat hardware development established by the company VisionSpace Technologies. For this project, we have designed a family of PC/104 format electronics boards, including a universal board, a single onboard computer board, a double redundant onboard computer board, and a FlatSat test bench. We have also managed to properly test out our primary design - the single onboard computer. On top of that, we have successfully conducted its testing under a gamma radiation source and acquire valuable experiment results. In order to provide helpful and accurate documentation of the developed boards, we have included many design details and illustrations into the main body of this thesis.

It is safe to say that our work has already been noticed by the open-source CubeSat community. We have presented our work at an Open Source CubeSat Workshop 2020, igniting a broad discussion about a united PC/104 pinout. At the time of submitting this thesis, an abstract including the VST104 project was accepted for the 5th \glref{ESA} CubeSat Industry Days. Similarly, another abstract regarding the radiation testing was accepted for the 1st Students Conference on Sensors, Systems and Measurement.

We are also pleased that our work is already being used not only by the \glref{VST} but also by other organizations. Romanian InSpace Engineering S.R.L. started the development of CubeSat subsystems based on the provided Board Sierra and Element Foxtrot. TU Darmstadt Space Technology e.V. has also received both of these boards for their development purposes. The \glref{VST} is currently using the VST104 platform for developing their Rust implementation of the telemetry and telecommand packet utilization. Future expansion of our work is planned as \glref{VST} is interested in integrating a NanoXplore NG-Medium \glref{FPGA} to the Board Sierra for the development of the POCKET+.
 % skontrolovane

% bibliography
\bibchap
\usebib/c (iso690) database/bibliography

% appendencies

%%%%% THESIS ASSIGNMENT

\app Thesis assignment

\midinsert
    \clabel[app:thesisAssignment]{Thesis assignment}
    \picw=0.9\hsize \cinspic database/thesisAssignment.pdf
    \caption/f Assignment of this bachelor's thesis.
\endinsert



%%%%% GLOSSARY

\app Glossary
\makeglos



%%%%% VST104 PINOUT

\app VST104 pinout

\midinsert
    \clabel[app_vst:vst104_pinout]{VST104 pinout}
    \picw=0.95\hsize \cinspic figures/diagrams/vst104_pinout.pdf
    \caption/f VST104 pinout: assignment of PC/104 header pins used in the VST104 project. Legend: red - mandatory, orange - optional, green - user defined, blue - legacy pins.
\endinsert

\midinsert \clabel[tab:vst104_pinout]{VST104 pinout}
    \ctable{cl}{
        Signal name &   Signal purpouse \crl
        {\tt UART_RCS_} & \glref{UART} used explicitly for the \glref{RCS}$^*$\cr
        {\tt GLO_SYNC}  & clock signal of global synchronization \cr
        {\tt GLO_FAULT} & signal setting a global fault flag \cr
        {\tt GLO_KS_}   & standard kill switch signal (active high) \cr
        {\tt CPU_WD_}   & watchdog signal for each of the \glref{OBC}s (\glref{CPU}s) \cr
        {\tt CPU_MODE}  & \glref{OBC} selection signal used in Board Delta \cr
    }
    \caption/t A brief explanation of selected signals from the VST104 pinout. A more specific definition will be proved ed by the \glref{VST} in the future. $*$radio communication subsystem
\endinsert



%%%%% PHOTO DOCUMENTATION

\app Photo documentation

\midinsert
    \clabel[app_photo:sierra_angle]{Board Sierra photo}
    \picw=1\hsize \cinspic figures/photos/sierra_angle.pdf
    \caption/f Photograph of the assembled Board Sierra.
\endinsert

\midinsert
    \clabel[app_photo:sierra_top]{Board Sierra photo top}
    \picw=0.685\hsize \cinspic figures/photos/sierra_top.pdf
    \caption/f Photograph of the assembled Board Sierra captured from its top side.
\endinsert

\midinsert
    \clabel[app_photo:sierra_bottom]{Board Sierra photo bottom}
    \picw=0.685\hsize \cinspic figures/photos/sierra_bottom.pdf
    \caption/f Photograph of the assembled Board Sierra captured from its bottom side.
\endinsert

\midinsert
    \clabel[app_photo:sierra_macro]{Board Sierra macro photo}
    \picw=0.85\hsize \cinspic figures/photos/sierra_macro.pdf
    \caption/f Macro photograph of the assembled Board Sierra's \glref{OBC}. A match is used to give a notion of the scale of the electronic components.
\endinsert

\midinsert
    \clabel[app_photo:watchdog]{Radiation: watchdog module}
    \picw=0.89\hsize \cinspic figures/photos/watchdog.pdf
    \caption/f Photograph of the watchdog module prepared for the radiation experiment.
\endinsert

\midinsert
    \clabel[app_photo:foxtrot]{Element Foxtrot photo}
    \picw=14.5cm \cinspic figures/photos/foxtrot.pdf
    \caption/f Photograph of the assembled Element Foxtrot with attached PC/104 modules.
\endinsert



%%%%% OVERSIZED FIGURES

\app Over-sized figures

\midinsert
    \clabel[app_vis:zeroTop]{Board Zero render top}
    \picw=90.17mm \cinspic figures/model/zero_top.pdf
    \caption/f Visualization of the Board Zero captured from its top side. The dimensions of this 3D render match the actual size of the board 1:1.
\endinsert

\midinsert
    \clabel[app_vis:sierraTop]{Board Sierra render top}
    \picw=90.17mm \cinspic figures/model/sierra_top.pdf
    \caption/f Visualization of the Board Sierra captured from its top side. The dimensions of this 3D render match the actual size of the board 1:1.
\endinsert

\midinsert
    \clabel[app_vis:sierraBottom]{Board Sierra render bottom}
    \picw=90.17mm \cinspic figures/model/sierra_bottom.pdf
    \caption/f Visualization of the Board Sierra captured from its bottom side. The dimensions of this 3D render match the actual size of the board 1:1.
\endinsert

\midinsert
    \clabel[app_sch:microcontroller]{Microcontroller schematic}
    \picw=0.88\hsize \cinspic figures/schemes/microcontroller.pdf
    \caption/f Schematic diagram of the microcontroller and its auxiliaries.
\endinsert

\midinsert
    \clabel[app_vis:foxtrotTop]{Element Foxtrot render}
    \picw=14.5cm \cinspic figures/model/foxtrot_top.pdf
    \caption/f Visualization of the Element Foxtrot captured from its top side. The dimensions of this 3D render match the actual size of the board 1:1.4.
\endinsert

\midinsert
    \clabel[app_vis:foxtrotKicad]{Element Foxtrot KiCad}
    \picw=14.5cm \cinspic figures/pcbs/foxtrot_kicad.pdf
    \caption/f PCB design of the Element Foxtrot captured directly in the KiCad environment. The dimensions of this render match the actual size of the board 1:1.4.
\endinsert



%%%%% ADDITIONAL MATERIALS

\app Additional materials

\midinsert \clabel[tab:componentsConsumption]{Components consumption}
    \ctable{ccccc}{
        \vspan2{Designator} & \vspan2{Power bus  $\rm{[V]}$} & \mspan3[c]{Current consumption $\rm{[mA]}$} \cr
                    &       & Min.  & Typ.  & Max. \crl
        AS[1-15]    & 5     & -     & 0.02  & 0.06 \cr %DGQ2788A  
        EF[1,2]     & 3.3, 5 & 0.14 & 0.21  & 0.30 \cr %TPS25940-Q1
        LG1         & 3.3   & -     & 0.1   & 4    \cr %74LVC1G11-Q100
        M[1-3]      & 3.3   & 10    & 25    & 40   \cr %S25FL256L
        M[4-6]      & 3.3   & -     & 5     & 5    \cr %CY15B102Q
        TS[1-7]     & 3.3   & -     & 0.20  & 0.40 \cr %MCP9804
        U[2,3]      & 5     & -     & 40    & 70   \cr %TCAN1051V-Q1
        Y2          & 3.3   & -     & 4.0   & 4.8      %SiT8924B
    }
    \caption/t Listing of electronic components power consumption. The values were obtained from the datasheets, and correspond to the normal operation at room temperature.
\endinsert

\midinsert \clabel[tab:uncertifiedParts]{Uncertified parts variants}
    \ctable{clll}{
        Designator      & Certified part no.      & Uncertified part no.  & Difference \crl
        LG1             & 74LVC1G11GW-Q100        & 74LVC1G11GW           & auto. \crl
        \vspan2{M[1-3]} & \vspan2{S25FL256LAGNFN} & S25FL256LAGNFI        & temp. \cr
                        &                         & S25FL256LDPNFI        & temp., speed \crl
        TS[1-7]         & MCP9804x-E/MC           & MCP9808x-E/MC         & accuracy \crl
        \vspan2{Q[1-3]} & \vspan2{TPS22965W-Q1 }  & TPS22965-Q1           & temp. \cr
                        &                         & TPS22975              & temp., auto.\cr
    }
    \caption/t List of available uncertified variants to some of the OBCs electronic components. Legend: auto. - missing \glref{AEC} certification, temp. - shrink operational temperature range, speed - decreased frequency, accuracy - decreased accuracy of measurements.
\endinsert


\bye
