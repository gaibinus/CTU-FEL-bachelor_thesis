
\chap Board Sierra - submodules

%%%%% MICROCONTROLLER

\sec Microcontroller

\quad The processing unit is the most important part of the \glref{OBC}. Professional designs use a wide variety of different instruction set architectures \cite[obc:dataPatterns, obc:imt, obc:pumpkin, obc:gauss, obc:nanosatpro]. However, the \glref{ARM} architecture seems to be more popular \cite[obc:isis, obc:aac, obc:anteleope, obc:gos]. This architecture is known for its good multiprocessing support, low power consumption, affordable pricing, and broad spectrum of existing applications. Considering these benefits and influenced by the LibreCube and TUDSaT, our \glref{VST} supervisors decided to pick an STM32 \glref{MCU}.

Our task was to choose a particular model of this 32-bit, Arm and Cortex-M based \glref{MCU}. Aiming rather for a low-power than high-performance characteristics, we decided to select an L series. Particularly the L4 series as it combines the largest flash memory size with the highest number of general-purpose input/output pins (\glref{GPIO}s) \cite[dat:mcuSeries]. Although, only one device from the L4 series is equipped with two CAN bus channels. As the presence of a second bus is crucial for our double-redundant approach, the STM32L496xx option was selected. This family comes in six different packages. Avoiding all of the \glref{BGA}-like ones (as described in chapter \ref[chap:componentsSelection]) narrows the selection to Zx, Vx, and Rx variants. The Zx is the most capable one in terms of flash size and \glref{GPIO} count. Therefore the STM32L496ZG (G stands for extended operating temperature range) was our final choice \cite[dat:mcu]. Some of its key characteristics are listed in table \ref[tab:microcontroller].

\midinsert \clabel[tab:microcontroller]{Microcontroller characteristics}
    \ctable{lcc|lcc}{
        Max. frequency      & $80\;\rm{[MHz]}$            &&& \glref{SPI}     & $3$     \cr
        Flash memory        & $1\;\rm{[MB]}$              &&& \glref{I2C}     & $4$     \cr
        Static RAM          & $320\;\rm{[kB]}$            &&& \glref{UART}    & $5$     \cr
        Comparators         & $2$                         &&& \glref{CAN}     & $2$     \cr
        Op. amplifiers      & $2$                         &&& \glref{GPIO}    & $115$   \cr
        Temperature         & $-40$ to $125\;\rm{[°C]}$   &&& \glref{DAC}     & $2$
    }
    \caption/t Highlighted characteristics of the STM32L469ZG microcontroller \cite[dat:mcu].
\endinsert

\secc Schematic design

\quad The \glref{MCU} pin assignment was continuously changing during the entire process of \glref{OBC} schematic and PCB design. Its final state is presented in figure \ref[app_sch:microcontroller]. We attempted to maximize the number of user-free \glref{GPIO}s with added functionalities such as \glref{ADC} or \glref{PWM} while keeping the fanout manageable. A significant help during this process was the CubeMX tool of the STM32CubeIde, visualizing all of the pinout combinations with a specific functionality. Each of the $3.3\rm{[V]}$ tolerant pins was used only for the internal circuitry, resulting in a fully $5\rm{[V]}$ tolerant main PC104 header connection.

A significant number of filtering and reservoir capacitors is needed to ensure the proper \glref{MCU} functionality. The assignment of correct capacitors to required \glref{MCU} pins was pretty straightforward, following the device datasheet \cite[dat:mcu] and STM32L4 hardware development application note \cite[app:getStart]. Furthermore, a $10\rm{[\mu H]}$ choke and $120\rm{[\Omega]}$ at $100\rm{[MHz]}$ ferrite bead were placed in series with the analog power input. This \glref{LC} filter supported by the bead should effectively eliminate both low and high-frequency interference.

The clock and data lines of both \glref{I2C} busses are equipped with standard $4.7\rm{[k\Omega]}$ resistors. Multiple $22\rm{[\Omega]}$ resistors were placed in series with high-speed clock signals of the \glref{SPI} interface. Two $0\rm{[\Omega]}$ resistors are present on {\it FAULT} and {\it MODE} lines to facilitate an optional hardware isolation. A $10\rm[k\Omega]$ resitor at {\it PH3} is suggested by \cite[app:getStart].

\secc PCB design

\quad The location and fanout of the \glref{MCU} are shown in figure \ref[pcb:microcontroller]. The \glref{MCU} is placed on the \glref{PCB} top side, covering most of the non-payload area. This big footprint of a LQFP144 package (approximately $2\rm{x}2\rm{[cm]}$) is a consequence of avoiding the \glref{BGA} packages while still trying to keep the pin count high. All of the capacitors were placed as close to their assigned pins as possible. In some cases, it was necessary to use the bottom side of the \glref{PCB}. The same approach was applied to the resistors.

\midinsert
    \clabel[pcb:microcontroller]{Microcontroller circuitry}
    \picw=0.75\hsize \cinspic figures/pcbs/microcontroller.png
    \caption/f Highlighted location of microcontroller circuitry.
\endinsert

%%%%% EXTERNAL CLOCK SOURCES

\sec External clock sources

\quad A $4-48\rm{[MHz]}$ high speed external oscillator (\glref{HSE}) can drive the system clock. Supported types are crystal, ceramic resonator, or silicone oscillator \cite[dat:mcu]. The last option seems to be the best as it is insensitive to electromagnetic interference (\glref{EMI}) and vibration. The only downside is its slightly lower temperature rejection \cite[app:oscComp]. We chose the SiT8924B, a $26\rm{[MHz]}$ silicon microelectromechanical system (\glref{MEMS}) oscillator \cite[dat:hse]. Accordingly to the clock configuration tool of the stm32cube software, we can reach various system clock frequencies up to $78\rm{[MHz]}$ (the max. is $80\rm{[MHz]}$ \cite[dat:mcu]). The circuitry follows the \glref{HSE} datasheet \cite[dat:hse]  and is shown on the right side of figure \ref[fig:clockSource]. The \glref{HSE} output OSE_IN can be enabled or disabled by the binary OSC_EN signal.

\midinsert
    \clabel[fig:clockSource]{Clock sources schematic}
    \picw=0.502\hsize \cinspic figures/schemes/clockSources.pdf
    \caption/f Schematic diagram of external clock sources.
\endinsert

A $32.768\rm{[kHz]}$ low speed external oscillator (\glref{LSE}) can drive the real time clock (\glref{RTC}), hardware auto calibration, or other timing functions. Table 7 in \cite[app:oscDes] recommends individual crystal resonators for this specific purpose with STM32 \glref{MCU}s. After a consideration of these options, we decided to use the ABS07AIG ceramic base crystal \cite[dat:lse]. The \glref{LSE} circuitry is based on the reference design in figure 5 in \cite[app:oscDes]. To achieve a stable frequency of this Pierce oscillator, it is required to determine the values of load capacitors $C_{L1}$, $C_{L2}$ and an external resistor $R_{Ext}$. This can be done using equations

$$ C_L = {{C_{L1} C_{L2}}\over{C_{L1}} + C_{L2}} + C_S \quad \wedge \quad C_{L1} = C_{L2}, \eqmark $$

$$ R_{Ext} = {1\over{2\pi f C_{L2}}}, \eqmark $$

where $C_L$ is the crystal load capacitance, $f$ is the crystal oscillation frequency and $C_S$ is the stray capacitance \cite[app:oscDes]. Values of $C_L$ and $f$ are listed in the crystal datasheet \cite[dat:lse]. We can assume as a rule of thumb, that $C_S=4\rm{[pF]}$. The final \glref{LSE} circuitry with computed values of the components is shown on the left side of figure \ref[fig:clockSource]. 

%%%%% POWER MANAGEMENT

\sec Power management

\quad A functional diagram of the implemented power management is shown in figure \ref[fig:powerManagement]. The \glref{OBC} is connected to each power bus through an electronic fuse (\glref{eFuse}). This device continuously monitors the bus for events of under-voltage, over-voltage, and over-current\fnote{This is a crucial feature in handling and resolving a latch-up event.}. As a response to such an event, the \glref{eFuse} will switch into high impedance and pull down specific input of an AND logic gate. This gate simultaneously controls two load switches, one for each power line. This approach ensures that a fault on one power bus will result in a high impedance of both \glref{OBC} power inputs. It also eliminates the risk of a death loop, a state where a reset of \glref{eFuse}s is not possible as they are switching each other off. Added benefits of this design are simple current measurements (using the \glref{eFuse} analog output) and a Kill Switch integration into the AND logic gate.

\midinsert
    \clabel[fig:powerManagement]{Power management diagram}
    \picw=0.75\hsize \cinspic figures/diagrams/powerManagement.pdf
    \caption/f Functional diagram of power management circuitry.
\endinsert

The final schematic diagram of power management circuitry is shown in figure \ref[app:powerManagement]. The most important part of this design is the \glref{eFuse}, as it covers all of the power control features. We decided to use the TPS25940-Q1 device \cite[dat:efuse]. Custom monitoring thresholds values can be set following the typical application schematic in figure $10-1$ in \cite[dat:efuse]. This was achieved by connecting specific resistors to the device, with values calculated using the TPS2594x design calculation tool \cite[app:tps2594Calc]. As the logic AND gate, we chose the 74LVC1G11-Q100 \cite[dat:gate]. This device is designed to operate in a mixed 3.3V and 5V environment, what corresponds with our application. The last important component is the load switch. In our case, the TPS22965W-Q1 with an inbuilt output discharge function \cite[dat:switch]. For a correct operation of the switch, the VBIAS pin should stay saturated for a while after disconnecting the VIN voltage. We achieved this behavior by charging a capacitor connected to the VBIAS from the VIN through a Schottky diode. Four reservoir capacitors are placed on both sides of the load switches, following the suggestions in \cite[dat:efuse, dat:switch]. Nominal logic values of all switching signals are set by pull-up or pull-down resistors. The summary of the final power management ratings is listed in table \ref[tab:powerManagement]. These values were chosen considering the power requirements of the remaining \glref{OBC} components and are a subject of change by a potential user.

\midinsert \clabel[tab:powerManagement]{Power management rating}
    \ctable{lc|ccc|c}{
        Power input      & Parameter & Min & Typ & Max & Unit \crll
        \vspan2{3V3 BUS} & voltage   & 2.9 & 3.3 & 3.5 & $\rm{V}$ \cr
                         & current   & 0.0 & -   & 1.2 & $\rm{A}$ \crl
        \vspan2{5V BUS}  & voltage   & 4.6 & 5.0 & 5.4 & $\rm{V}$ \cr
                         & current   & 0.0 & -   & 1.2 & $\rm{A}$
    }
    \caption/t \glref{OBC} power rating. Value out of range will cause a protective shutdown.
\endinsert

%%%%% PERIPHERAL ISOLATORS

\sec Peripheral isolators

\quad We chose to implement the design using robust analog switches as isolators of data lines. This approach is more straightforward, less expensive, and requires a smaller \glref{PCB} area than an optocoupler-based or an \glref{FPGA}-based ones (referred in chapter \ref[chap2:features]). A functional diagram of the implementation is shown in figure \ref[fig:peripheralIsolatorsDiag]. \glref{OBC} data lines are connected to the rest of the spacecraft through a series of analog switches. These lines are grouped by particular data buses and are assigned to a separate switch. The \glref{OBC} can enable or disable a specific switch and therefore isolate a particular data bus from the remaining spacecraft subsystems. Pulling low the Kill Switch will result in high impedance of all switches and completely isolating the \glref{OBC} data lines.

\midinsert
    \clabel[fig:peripheralIsolatorsDiag]{Peripheral isolators diagram}
    \picw=0.8\hsize \cinspic figures/diagrams/peripheralIsolators.pdf
    \caption/f Functional diagram of peripheral isolators circuitry.
\endinsert

Schematic diagrams of two isolators are shown in figure \ref[fig:peripheralIsolatorsSchem]. We decided to use the DGQ2788A device \cite[dat:isolator]. To cover all of the data buses, the \glref{OBC} hosts fifteen of these analog switches in a dual double pole double throw (\glref{DPDT}) configuration. The \glref{OBC} data lines are connected to common (\glref{COM}) terminals. Normally closed (\glref{NC}) terminals are left floating, whereas normally open (\glref{NO}) terminals are connected to the spacecraft data lines. The important Kill Switch functionality is implemented using the device's power down protection. If the switch loses power, it will enter the normal state. This approach simplifies the circuitry a lot as it substitutes an otherwise necessary system of multiple logic gates. The other beneficial features of this analog switch are inbuild signal clamping diodes and a high latch-up current of $300\rm{[mA]}$. The presence of pull-down resistors and decoupling capacitors in the schematic follows the device datasheet.

\midinsert
    \clabel[fig:peripheralIsolatorsSchem]{Peripheral isolators schematic}
    \picw=\hsize \cinspic figures/schemes/peripheralIsolators.pdf
    \caption/f Schematic diagram of two peripheral isolators.
\endinsert

%%%%% EXTERNAL MEMORY

\sec External memory

%%%%% CAN BUS DRIVERS

\sec CAN bus drivers

\midinsert
    \clabel[fig:canBusDrivers]{CAN drivers schematic}
    \picw=0.75\hsize \cinspic figures/schemes/canBusDrivers.pdf
    \caption/f Schematic diagram of CAN buses driving circuitry.
\endinsert

%%%%% TEMPERATURE MONITORING

\sec Temperature monitoring
