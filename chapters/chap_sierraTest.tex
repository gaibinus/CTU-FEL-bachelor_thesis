
%%%%% BOARD SIERRA TESTING

\chap Board Sierra - testing

%%%%% TESTING SOFTWARE

\sec Testing software




%%%%% RADIATION TESTING

\sec Radiation testing




%%%%% EXPERIMENT SETUP

\secc Experiment setup

\quad For the purpose of the radiation testing, the Board Sierra was extended with a couple of electronic sensors: i) an always-on 3D accelerometer and 3D gyroscope LSM6DS3, ii) a high-performance 3-axis magnetic sensors MMC5983MA (two devices), and iii) an integrated 6-axis motion processor with gyroscope and accelerometer MPU6050. All of these sensors were connected with the \glref{OBC} using an \glref{$\rm{I^2C}$} data bus. Each sensor was powered through a separate high-base current PNP transistor controlled by the \glref{MCU}. A proper power reset of each sensor could have been achieved by turning off this transistor and isolating the corresponding \glref{$\rm{I^2C}$} peripheral isolator. This feature was implemented in order to resolve a potential \glref{$\rm{I^2C}$} bus lockup or the sensor's latch-up.


\midinsert
    \clabel[pho:rez_setup]{}
    \picw=0.85\hsize \cinspic figures/photos/rez_setup.pdf
    \caption/f 
\endinsert


%%%%% EXPERIMENT RESULTS

\secc Experiment results

\quad Being a complex system of multiple semiconductor devices, the \glref{OBC} was not expected to withstand the whole experiment. At $5.19\rm{[hour]}$ after the start, the first unintentional reboot was logged. Until this point, the \glref{OBC} performed normally without a single malfunction. The radiation dose at the time of this event was $254.32\rm{[Gy]}$. This reboot was the opening of $6.8$ minutes long \glref{OBC}'s decay. Figure \ref[plt:reboot] shows timestamps of data and reboot logs received in this period. It is visible that the \glref{OBC} entered a loop reaching up to $110$ reboots per second. Despite this enormous frequency, some windows of inactivity and even a few data logs can be observed. At $5.27\rm{[hour]}$, the \glref{OBC} managed to start functioning again. Although, with visible delays between the acquired data. The very last log was received from the \glref{OBC} at $5.30\rm{[hour]}$, marking the end of its functionality. The overall radiation dose at this point was $259.90\rm{[Gy]}$.

\midinsert
    \clabel[plt:reboot]{Radiation: OBC decay}
    \picw=0.95\hsize \cinspic figures/plots/reboot.pdf
    \caption/f Timestamps of logs received from the \glref{OBC} during its last moments of activity.
\endinsert

The \glref{OBC}'s and the sensors' current consumptions, measured through the external shunt resistors, are shown in figure \ref[plt:currents]. The periodically repeating pattern on both $3.3\rm{[V]}$ power busses was generated by the prevention power reset. As this feature has powered off the sensors and suspended some of the \glref{OBC}'s activity every ten minutes, the corresponding decrease in required power is visible. Another noticeable trend is the increasing current consumption by the \glref{OBC} and slightly decreasing current consumption by the sensor board. However, this time we cannot provide it with an explanation. A significant current drop and oscillations follow the already described \glref{OBC} failure. Some changes in the current consumption are also visible closely before this event.

\midinsert
    \clabel[plt:currents]{Radiation: current consump.}
    \picw=0.95\hsize \cinspic figures/plots/currents.pdf
    \caption/f Current consumption measured throughout the experiment.
\endinsert

Temperature readings obtained from the \glref{OBC}'s inbuilt temperature sensors are shown in figure \ref[plt:mcp9884]. No errors or failures of these sensors were recorded throughout the experiment. The acquired data also doesn't show any abnormality in the \glref{OBC}'s temperature or heat distribution. The slightly increased readings of the sensor T1 could be explained by its assignment to the Flash memory subsystem. Its continuous activity might have easily resulted in an increase of its temperature by roughly a $0.5\rm{[°C]}$.

\midinsert
    \clabel[plt:mcp9884]{Radiation: OBC temperature}
    \picw=0.95\hsize \cinspic figures/plots/mcp9884.pdf
    \caption/f Readings of the OBC's inbuilt temperature sensors. The location and purpose of each sensor were described in section \ref[chap:temperatureMonitoring].
\endinsert

The LSM6DS3 was the only sensor that has experienced a total failure (actually seven of them). After being requested by the OBC, no response was obtained from the sensor, and the \glref{$\rm{I^2C}$} connection timeout was reached. The first failure occurred after exposure to a radiation dose of $196.90\rm{[Gy]}$. Timestamps of these failures, together with the obtained angular velocity, are shown in figure \ref[plt:lsm6ds3g_gyr]. A continuous malfunction/degradation of the sensor is visible from these data, as the Y and Z-axis readings are decreasing over time. A real motion could not have caused these trends due to the sensor's stationary mount. Interestingly, the acceleration data acquired from this sensor are perfectly reasonable.

\midinsert
    \clabel[plt:lsm6ds3g_gyr]{Radiation: LSM6DS3G}
    \picw=0.95\hsize \cinspic figures/plots/lsm6ds3g_gyr.pdf
    \caption/f Angular velocity readings and occurred malfunctions of the LSM6DS3 sensor.
\endinsert



\midinsert
    \clabel[plt:mmc5983]{Radiation: MMC5983}
    \picw=0.95\hsize \cinspic figures/plots/mmc5983.pdf
    \caption/f 
\endinsert

\midinsert
    \clabel[plt:mpu6050_acc]{Radiation: MPU6050}
    \picw=0.95\hsize \cinspic figures/plots/mpu6050_acc.pdf
    \caption/f 
\endinsert