% On-board computer for PC104 format CubeSats
% ---------------------------------------------------------------------
% Filip Geib Jan. 2021

% export in Long-term Archiving (PDF/A)
% conversion needed: https://pdfrecover.herokuapp.com/pdfaconvert/
\input glyphtounicode.tex
\input glyphtounicode-cmr.tex
\input glyphtounicode-ntx.tex
\pdfgentounicode=1

\input ctustyle3  % The template (in version 3, for OpTeX) is included here.

\worktype [B/EN]

\faculty    {F3}
\department {Department of Measurement}
\title      {On-board computer for PC104 format CubeSats}
\subtitle   {}

\author     {Filip  Geib}
\date       {February 2021}
\supervisor {Ing. Vojtěch Petrucha, Ph.D.}
\studyinfo  {Cybernetics and Robotics}
\workname   {}

\workinfo   {\url{https://github.com/visionspacetec/VST104}}
\titleSK    {Palubný počítač pre CubeSaty formátu PC104}
\subtitleSK {}

\pagetwo    {}

\abstractEN {
   \lorem[1-2]
}
\abstractSK {
   \lorem[3-4]
}

\keywordsEN {
    CubeSat; PC104; OBC; hardware; PCB design; schematics.
}
\keywordsSK {
    CubeSat; PC104; OBC; hardvér; PCB dizajn; schémy.
}
\thanks {
   \lorem[5]
}
\declaration {
    I hereby declare that the presented thesis is my own work and that I have
    cited all sources of information in accordance with the Guideline for
    adhering to ethical principles when elaborating an academic final thesis.
                
    I acknowledge that my thesis is subject to the rights and obligations
    stipulated by the Act No.\,121/2000~Coll., the Copyright Act, as amended. In
    accordance with Article~46~(6) of the Act, I hereby grant a nonexclusive
    authorization (license) to utilize this thesis, including any and all
    computer programs incorporated therein or attached thereto and all
    corresponding documentation (hereinafter collectively referred to as the
    “Work”), to any and all persons that wish to utilize the Work. Such
    persons are entitled to use the Work for non-profit purposes only, in any
    way that does not detract from its value. This authorization is not limited
    in terms of time, location and quantity.
    
    \bigskip
    In Prague on February 25, 1999 % !! Attention, you have to change this item.
    \signature % makes dots
}

%%%%% <--   % The place for your own macros is here.

\draft     % Uncomment this if the version of your document is working only.
%\linespacing=1.7  % uncomment this if you need more spaces between lines
                   % Warning: this works only when \draft is activated!
%\savetoner        % Turns off the lightBlue backround of tables and
                   % verbatims, only for \draft version.
%\blackwhite       % Use this if you need really Black+White thesis.
%\onesideprinting  % Use this if you really don't use duplex printing. 

\makefront  % Mandatory command. Makes title page, acknowledgment, contents etc.

% only for temporary content generation, one chapter => one TeX file
\chap Introduction
\chap Related works
% main body
\chap PC104 format CubeSats
\sec CubeSat concept
\sec PC104 standard
\secc Mechanical specification
\secc Main header pinout
\sec OBC requirements
\secc Capabilities and features 
\secc Components certification
%
\chap Board Sierra - single OBC
\sec Submodules and circuit design
\secc Microcontroller
\secc Peripheral isolators
\secc Power management
\secc External memory
\secc CAN bus drivers
\secc Temperature sensing
\sec PCB design and assembly
\secc PCB specifications
\secc Submodles placement
\secc Routing and fanout
\secc Assembly and debug
\sec Board Delta - double OBC
\secc Circuit modification
\secc PCB modification
%
\chap Element Foxtrot - FlatSat
\sec Submodules and circuit design
\secc Ordinary power source
\secc USB-C power source
\secc Voltage handling
\secc PC104 modules slots
\sec PCB design and assembly
\secc PCB specifications
\secc Submodles placement
\secc Routing and fanout
\secc Assembly and debug
%
\chap Board Sierra testing
\sec Testing software
\sec Radiation testing
\sec Environmental testing
%
\chap Conclusion

\bye
