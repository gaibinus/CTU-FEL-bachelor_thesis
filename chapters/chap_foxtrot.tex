
%%%%% ELEMENT FOXTROT

\label[chap:elementFoxtrot]
\chap Element Foxtrot

\quad This chapter addresses the design and characteristics of the Element Foxtrot - a FlatSat test bench for the PC/104 format modules with an inbuilt power supply. Its layout, embedded features, and functional principles are explained in the following sections. A 3D visualization of the Element Foxtrot is shown in figure \ref[app_vis:foxtrotTop].


%%%%% FLATSAT DESIGN

\sec FlatSat design

\quad According to the \glref{VST} desires, the Element Foxtrot fits two use-case scenarios. Like any other FlatSat, it supports the testing and development of new PC/104 modules. Also, it is a tool of their show-off and presentation. Developed modules can be attached to the FlatSat and then displayed at workshops or conferences. For example, running presented algorithms or software packages in real-time. Thus, we attempted to design the Element Foxtrot to accommodate handy features and have an appealing and professional look. After a discussion with the \glref{VST} supervisors, we came up with its final design. The FlatSat has a rectangular shape and is divided into two parts.

The main part has dimensions of $20.25$ x $26.10\rm{[cm]}$ and hosts four slots for the PC/104 format modules. Silkscreen contours mark the position and orientation of these slots with respect to the LibreCube template (section \ref[chap:PC104standard]). Individual slots headers are electrically connected together, concerning the VST104 pinout (section \ref[chap:mainHeaderPinout]). A large \glref{VST} company logo is displayed below the slots, giving the FlatSat a nice advertisement feature. The purpose of this part is to provide a connection between the mounted PC/104 modules while keeping them next to each other and easily accessible.

The second part of the FlatSat is an inbuilt power supply. The main task of this $20.25$ x $4.55\rm{[cm]}$ strip is to deliver power to the mounted PC/104 modules. This power supply can be connected to either a $6.4\rm{[mm]}$ barrel jack or \glref{USB-C} power inputs. Then, it delivers a $3.3\rm{[V]}$, $5\rm{[V]}$, and unregulated power lines. We decided to integrate a power source into the FlatSat to have an accessible power-up possibility. During a software testing or a conference, it might be convenient just to plug in an ordinary charger rather than setting up a laboratory power source or a battery-powered \glref{PDCU} module.

The overall dimensions of the FlatSat are $20.25$ x $30.75\rm{[cm]}$, which should be easily fabricable by ordinary \glref{PCB} manufacturers. If desired, the two FlatSat parts can be broken apart from each other and it is possible to use them separately. As the design of the inbuilt power source is nontrivial, it is addressed in the rest of this chapter.



%%%% POWER SUPPLY FEATURES

\label[chap:powerFeatures]
\sec Power supply features

\quad Besides its main functionality, the power supply also offers multiple extra features. These small additions were designed to create the usage of the Element Foxtrot more convenient and user-friendly. The entities and locations of these features are shown and labeled in figure \ref[img:foxtrot_functions]. We list and describe their functionality in the rest of this section.

\circleparams={\ratio=1 \fcolor=\White \lcolor=\Black \hhkern=0.9pt \vvkern=0.9pt}
\ovalparams={\roundness=5pt \fcolor=\White \lcolor=\Black \hhkern=-4pt \vvkern=-2.5pt}

\midinsert
    \clabel[img:foxtrot_functions]{Foxtrot functionality}
    \picw=1\hsize \cinspic figures/model/foxtrot_functions.pdf
    \caption/f Visualization of the Element Foxtrot's power supply with its features labeled. Legend: {\typosize[8/]\incircle{⟨1⟩}} input selection switch, {\typosize[8/]\incircle{⟨2⟩}} $6.4\rm{[mm]}$ barrel jack connector, {\typosize[8/]\inoval{⟨3a⟩}} \glref{\glref{USB-C}} connector, {\typosize[8/]\inoval{⟨3b⟩}} \glref{USB} data bus header, {\typosize[8/]\incircle{⟨4⟩}} current sensing resistors: {\typosize[8/]\inoval{⟨4a⟩}} $5\rm{[V]}$, {\typosize[8/]\inoval{⟨4b⟩}} $3.3\rm{[V]}$, {\typosize[8/]\inoval{⟨4c⟩}} unregulated, {\typosize[8/]\incircle{⟨5⟩}} disconnection terminal jumpers: {\typosize[8/]\inoval{⟨5a⟩}} $5\rm{[V]}$, {\typosize[8/]\inoval{⟨5b⟩}} $3.3\rm{[V]}$, {\typosize[8/]\inoval{⟨5c⟩}} unregulated, {\typosize[8/]\incircle{⟨6⟩}} breakable connection, {\typosize[8/]\incircle{⟨7⟩}} $4\rm{[mm]}$ banana sockets: {\typosize[8/]\inoval{⟨7a⟩}} $5\rm{[V]}$, {\typosize[8/]\inoval{⟨7b⟩}} $3.3\rm{[V]}$, {\typosize[8/]\inoval{⟨7c⟩}} unregulated, {\typosize[8/]\inoval{⟨7d⟩}} \glref{GND}, and {\typosize[8/]\incircle{⟨8⟩}} indication \& status \glref{LED}s (detailed explanation in section \ref[chap:powerFeatures]).
\endinsert

\begitems
    * {\sbf User control of inputs:} The inbuilt power supply supports two separate power inputs: a $6.4\rm{[mm]}$ barrel jack or a \glref{\glref{USB-C}} connector. A three-position slide switch (on/off/on) between these two connectors is used to enable one or none of them. When its slider is located in the middle position, the power supply is turned off. When slid to one side, the corresponding power input enables, and the power supply turns on.
    * {\sbf 6.4[mm] barrel jack input:} This power input expects a positive barrel jack polarity. The supported input voltage range is $7$-$14\rm{[V]}$, and the maximum current flow is $8\rm{[A]}$. The power input is embedded with a reversed polarity and overvoltage protection.
    * {\sbf \glref{USB-C} handshake input:} A standard \glref{USB-C} connector is combined with a power delivery controller \glref{IC}, allowing so-called handshaking. The power input can be used in two modes: i) it can handle an ordinary $5\rm{[V]}$ \glref{USB-C} charger, or ii) negotiate a power contract with an intelligent \glref{USB-C} charger. Therefore, the supported input voltage range is $5$ \& $7$-$14\rm{[V]}$, and the maximum current flow is $6\rm{[A]}$.  The differential pair of the \glref{USB} data bus is led to a separate header close to the connector. An independent $5\rm{[V]}$ power bus is also available at this header. This feature allows the use of the \glref{USB-C} connection also for communication purposes.
    * {\sbf Current sensing resistors:} Before leaving the power supply and entering the other part of the FlatSat, the power lines are interrupted with shunt resistors. A 1206 sized \glref{SMD} resistor's footprints are placed in series with those lines, accommodating an inbuilt current sensing feature. Two test pads are then connected to each resistor, providing a connection terminal for the voltage drop measurement. If not intent to use, the footprints can be populated with $0\rm{[\Omega]}$ resistors or even a blob of solder.
    * {\sbf Disconnection terminal:} A set of removable jumpers can be used to temporarily separate the power supply circuitry from the rest of the FlatSat. Each of them is rated for currents of up to $6\rm{[A]}$. Although removing the jumpers would interrupt the power lines, the ground plane is shared and \glref{GND} would remain connected.
    * {\sbf Breakable connection:} Under some circumstances, it might be needed to remove the power supply from the remaining FlatSat entirely. In such a case, these two parts can be separated by breaking a connection between them. This line consists of multiple long cutoffs and can be broken with a reasonable amount of applied force.
    * {\sbf 4[mm] banana sockets:} A laboratory power supply with its standard banana cables can also be connected to the FlatSat. The $4\rm{[mm]}$ female sockets are placed behind the power supply circuitry. These sockets will remain directly attached to the PC/104 modules power lines even after temporary or permanent separation of the power source. This can be used to bypass the power supply or for voltage measurement purposes. The sockets are available for all power lines together with the \glref{GND}.
    * {\sbf Indication \& status \glref{LED}s:} To increase the user experience, a bunch of colorful light emitting diodes (\glref{LED}s) is spread through the power supply circuitry. Their monitoring purposes (following the labeling in figure \ref[img:foxtrot_functions]) are: 8a - power at the barrel jack input detected (red), 8b - power at the \glref{USB-C} input detected (red), 8c - a power delivery error (yellow), 8d - the power level negotiated successfully (blue), 8f - $5\rm{[V]}$ bus for the \glref{USB-C} data header active (green), and 8e - the power supply active (green).
    * {\sbf Link to another Foxtrot:} In applications where more than four PC/104 modules are required, the PC/104 bus can be connected to another Element Foxtrot. This extension is made through two 45-conductor \glref{FFC} cables. Their two \glref{SMD} connectors are located under the third module of the FlatSat. It is important to note that these cables connect only the PC/104 data busses. The power delivery has to be connected separately, preferably using the $4\rm{[mm]}$ banana sockets.
\enditems



%%%%% POWER SUPPLY CIRCUITRY

\sec Power supply circuitry

\quad A schematic diagram of the power supply circuitry is available at the end of the appendix section. This electronics scheme is rather complex, and it might not be easy to orientate in it quickly. Therefore we explain the circuitry design and functional principles with the help of its hardware diagram, presented in figure \ref[dia:foxtrotArchitecture], instead.

\midinsert
    \clabel[dia:foxtrotArchitecture]{Foxtrot HW architecture}
    \picw=0.924\hsize \cinspic figures/diagrams/foxtrotArchitecture.pdf
    \caption/f Diagram of the Element Foxtrot's power supply hardware architecture.
\endinsert

The central element of the circuitry is its main power bus, labeled as the {\it VSINK}. This bus provides electric energy for the rest of the FlatSat either directly or through the associated \glref{DCDC} converters. This energy can be sourced in three different ways. In other words, the {\it VSINK} can be powered in three modes: i) from the $6.5\rm{[mm]}$ barrel jack connector, ii) from the \glref{USB-C} connector with power delivery, or iii) from the \glref{USB-C} connector without power delivery. These modes are selected by a combined logic of the user switch position and the power delivery controller's handshake status.  

In the first mode, a closed state of switch {\it A} and an open state of switch {\it B} are forced by the user switch position. In the real circuitry, these two switches ({\it A}, {\it B}) are implemented as bi-directional P-channel \glref{MOSFET} switches. The first power mode also accommodates two safety features: overvoltage and reverse polarity protections. The overvoltage protection is designed to compare the input voltage and open switch {\it A} if it exceeds $15\rm{[V]}$. Its circuitry uses discrete components (such as TL432 voltage reference) and follows a reference design presented in \cite[app:reversePolarity]. The reverse polarity protection is based on a well-known design using a single P-channel \glref{MOSFET} combined with a Zener diode.

The second and third modes share the same setup while powering the {\it VSINK} bus. The user switch forces an open state of switch {\it A} and permits a closed state of switch {\it B}. The associated power delivery controller makes the final decision whether to close or leave switch {\it B} open. In the circuitry design, we have implemented a standalone STUSB4500 controller accordingly to its typical application given in its datasheet \cite[dat:pdController] and the evaluation data-brief \cite[app:steval]. This device attempts to negotiate an appropriate voltage from the $7-14\rm{[V]}$ range with the attached \glref{USB-C} charger. If the charger supports power delivery and the negotiation is successful (mode ii) or if the negotiation is unsuccessful\fnote{The power delivery negotiation (handshake) can fail because of two reasons: the charger i) does not support power delivery, or ii) cannot provide voltage (current) from the required voltage (current) range.} although standard $5\rm{[V]}$ is provided (mode iii), switch {\it B} is closed by a {\it USB OK} signal. The STUSB4500 device can be initialized through the \glref {$\rm{I^2C}$} connection (using the associated header) as described in its programming guide \cite[app:usbcProramm]. We found the approach using the STSW-STUSB003 software library \cite[app:stsw] to be the most convenient.

Once is the {\it VSINK} bus correctly powered, it is required to distribute the power to the FlatSat's separate power lines. All of the PC/104 slots are connected to the required $3.3\rm{[V]}$, $5\rm{[V]}$, and unregulated lines. The unregulated line is attached to the {\it VSINK} bus directly. However,  step-down \glref{DCDC} converters are necessary to decrease the {\it VSINK} voltage for the $3.3\rm{[V]}$ and $5\rm{[V]}$ power lines. For this purpose, the OKL-T/6-W12 devices were selected. Output voltages of both converters are set by multi-rotation trimmers, enabling a fine tuning. Running the converters in the power modes i and ii is straightforward as the {\it VSINK} voltage is guaranteed to be at least $7\rm{[V]}$. This is above the $6.5\rm{[V]}$ minimum required by the $5\rm{[V]}$ converter. Yet, the mode iii is problematic as its {\it VSINK} voltage is only $5\rm{[V]}$ and therefore insufficient. Hence, bypassing the $5\rm{[V]}$ \glref{DCDC} converter in the power mode iii had to be implemented. We accomplished this by including two more P-channel \glref{MOSFET} switches (switches {\it C} and {\it D}) controlled by a {\it bypass} logic. Inputs of this logic are the user switch position and the {\it handshake status} signal from the power delivery controller. If the power from the barrel jack was selected (mode i) or the \glref{USB-C} negotiation was successful (mode ii), the {\it bypass} logic closes switch {\it C} and opens switch {\it D}. However, if the power from the \glref{USB-C} was selected and the negotiation failed (mode iii), the logic opens switch {\it C} and closes switch {\it D}. As mentioned in section \ref[chap:powerFeatures], current sensing resistors (marked as {\it R}) and {\it jumper} terminals are placed between the power supply circuitry and the PC/104 slots.

Inspired by the data-brief of the STUSB4500 evaluation board \cite[app:steval], we decided to include a separate step-down switching regulator. The L7985 device is connected directly to the {\it VSINK} bus and provides a stable $5\rm{[V]}$ output for the \glref{USB} data connection.

Various \glref{ESD} protection diodes, 2AG cartridge fuses, and reservoir capacitors are distributed through the circuitry. For their exact location, refer to the circuitry electronics schematic. A \glref{SPICE}-based analog electronic circuit simulator LTspice was used to simulate and check for the desired functionality of the presented circuitry. This functionality was afterward properly tested on multiple assembled Element Foxtrots. 



%%%%% PCB DESIGN
\sec PCB design

\quad The Element Foxtrot is designed on a four-layer \glref{PCB} with a slightly unconventional copper layer assignment. The two previously mentioned parts of the FlatSat (the power supply part and the main part with four PC/104 slots) are routed with different approaches, sharing only the common ground plane. The final layout and routing of the whole FlatSat captured from the KiCad environment is shown in figure \ref[app_vis:foxtrotKicad]. The location and layout of the power supply's circuitry are shown in figure \ref[img:foxtrot_subsystems].

\circleparams={\ratio=1 \fcolor=\White \lcolor=\Black \hhkern=0.9pt \vvkern=0.9pt}

\midinsert
    \clabel[img:foxtrot_subsystems]{Foxtrot circuitry summary}
    \picw=1\hsize \cinspic figures/model/foxtrot_subsystems.pdf
    \caption/f Location and layout of the Element Foxtrot's power supply circuitry. Legend: {\typosize[8/]\incircle{⟨1⟩}} step-down switching regulator for the \glref{USB} communication, {\typosize[8/]\incircle{⟨2⟩}} barrel jack overvoltage and reverse polarity protection, {\typosize[8/]\incircle{⟨3⟩}}, power delivery controller, {\typosize[8/]\incircle{⟨4⟩}} power source and bypass switching, {\typosize[8/]\incircle{⟨5⟩}} power lines 2AG cartridge fuses, {\typosize[8/]\incircle{⟨6⟩}} tunable \glref{DCDC} converters.
\endinsert

In the power supply part, the top and the first inner copper layers are used for the power routing. Power busses and other power connections are formed from polygons of these layers. A dense rectangular net of vias is used to connect these layers within an individual bus. These power traces (polygons) are scaled to easily withstand a total current delivery of $6\rm{[A]}$. The second inner layer accommodates the common ground plane. All of the power supply electronic components are placed on the \glref{PCB} top side for easier assembly. Therefore the top layer is also used for a majority of the signal traces. If it was not possible to route a trace there, then the bottom layer was used.

The main FlatSat part contains just a few power polygons but a very dense network of signal traces. These traces are placed on the top and the bottom copper layers. The first inner layer supports the power delivery, whereas the second inner layer is the common ground plane. The PC/104 slots are connected in a U shape, providing one continuous data bus avoiding any loops. The silkscreen numbers marking the individual slots are enumerated with respect to this connection. All of the signal traces connecting the slots had to be routed manually as all auto-routing attempts have failed.
