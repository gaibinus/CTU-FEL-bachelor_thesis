% On-board computer for PC104 format CubeSats
% ---------------------------------------------------------------------
% Filip Geib Apr. 2021

% export in Long-term Archiving (PDF/A)
% conversion needed: https://pdfrecover.herokuapp.com/pdfaconvert/
\input glyphtounicode.tex
\input glyphtounicode-cmr.tex
\input glyphtounicode-ntx.tex
\pdfgentounicode=1

% define English quotes
\enquotes

% include template
\input ctustyle3

\worktype [B/EN]

\faculty    {F3}
\department {Department of Measurement}
\title      {On-board computer for PC104 format CubeSats}
\subtitle   {}

\author     {Filip  Geib}
\date       {May 2021}
\supervisor {Ing. Vojtěch Petrucha, Ph.D.}
\studyinfo  {Cybernetics and Robotics}
\workname   {}

\workinfo   {} %\url{https://github.com/visionspacetec/VST104}
\titleSK    {Palubný počítač pre CubeSaty formátu PC104}
\subtitleSK {}

\pagetwo    {}



%%%%% ABSTRACT - skontrolovane

\abstractEN {
    This thesis is focused on the development of PC/104 format electronic boards for CubeSat applications. Particular attention was given to a single on-board computer (OBC) module. This universal ARM-based OBC is driven by an STM32L4 microcontroller supporting a wide range of interfaces. Its additional features include robust power management, separate peripheral isolation, triple-redundant FLASH and F-RAM memories, two CAN bus transceivers, built-in temperature monitoring, and a significant payload sector. A dedicated radiation testing was conducted under a gamma radiation source. Moreover, three additional boards were developed, including a double redundant version of the OBC, a universal PC/104 module, and a FlatSat test bench. All of these boards were designed with open-source principles in a KiCad environment. This thesis contributes to the VisionSpace Technologies VST104 project by introducing a hardware platform for mission control systems testing and compression algorithms development.
    \bigskip
}
\abstractSK {
    Táto práca je zameraná na vývoj elektronických dosiek formátu PC/104 pre CubeSat aplikácie. Osobitná pozornosť bola venovaná modulu palubného počítača (OBC). Tento univerzálny OBC založený na ARM technológii je riadený mikrokontrolérom STM32L4 podporujúcim širokú škálu rozhraní. Medzi jeho ďalšie funkcie patrí robustná správa napájania, samostatná izolácia periférii, trojito redundantné pamäte FLASH a F-RAM, dva komunikátory zbernice CAN, zabudované monitorovanie teploty a rozsiahly sektor užitočného nákladu. Cielené testovanie odolnosti voči žiareniu bolo uskutočnené pod zdrojom gama radiácie. Ďalej boli vyvinuté tri dosky, vrátane dvojito redundantnej verzie OBC, univerzálneho modulu PC/104 a testovacej platformy FlatSat. Všetky tieto dosky boli navrhnuté podľa princípov open-source v prostredí KiCad. Táto práca prispieva k projektu VisionSpace Technologies VST104 zavedením hardvérovej platformy určenej pre testovanie systémov riadenia misií a vývoja kompresných algoritmov.
    \bigskip
}



%%%%% KEYWORDS

\keywordsEN {
    CubeSat; FlatSat; OBC; PC104; radiation testing.
}
\keywordsSK {
    CubeSat; FlatSat; OBC; PC104; radiačné testovanie.
}



%%%%% ACKNOWLEDGEMENT

\thanks {
    I would like to express my gratitude to VisionSpace Technologies, namely to José Feiteirinha, for conducting and sponsoring this project. I also thank my supervisor Vojtěch Petrucha for his guidance and support.
}



%%%%% DECLARATION

\declaration {
    I hereby declare that the presented work was developed independently and that I have listed all sources of information used within it in accordance with the methodical instructions for observing the ethical principles in the preparation of university theses. 
    \bigskip
    In Prague on May 21, 2021
    \signature % makes dots
}

%%%%% MACROS

%\draft     % Uncomment this if the version of your document is working only.
%\linespacing=1.7  % uncomment this if you need more spaces between lines
                   % Warning: this works only when \draft is activated!
%\savetoner        % Turns off the lightBlue backround of tables and
                   % verbatims, only for \draft version.
%\blackwhite       % Use this if you need really Black+White thesis.
%\onesideprinting  % Use this if you really don't use duplex printing. 
\input database/glosdata % Include glossary

%%%%% MAIN TEXT

% make title page, acknowledgment, contents etc.
\makefront

% introduction chapter

\label[chap:intro]
\chap Introduction % skontrolovane

% main body

\chap VST104 project

%%%%% CUBESATS

\sec CubeSat concept

%%%%% OBC

\secc On Board Computer

Despite growing interest from industry in CubeSats as proper means of technology
demonstration, such platforms are still primarily considered as an educational tool. \cite[pap:fromDesignToOperation]

%%%%% VST104 MOTIVATION

\sec Project motivation


%%%%% OPEN SOURCE

\secc OPEN SOURCE

% software

%%%%% PCI104

\label[chap:PC104standard]
\sec PCI104 standard

%%%%% MECHANICAL

\secc Mechanical specification

%%%%% PINOUT

\secc Main header pinout

 % skontrolovane

%%%%% BOARD SIERRA

\label[chap:boardSierra]
\chap Board Sierra

\quad This chapter addresses the design process and overall characteristics of the Board Sierra - a single \glref{OBC} module in the PC/104 format. Design challenges, development decisions, and final features are explained in detail. Particular attention is given to the onboard computer, from now referred to as the \glref{OBC}. A 3D visualization of the Board Sierra is shown in figure \ref[img:sierra_miniature]. A description of its subsystems is provided in chapter \ref[chap:boardSierraSubsystems].

\midinsert
    \clabel[img:sierra_miniature]{Board Sierra 3D render}
    \picw=0.95\hsize \cinspic figures/model/sierra_miniature.pdf
    \caption/f 3D render of the Board Sierra from the top (left) and the bottom side (right). A real scaled version of this visualization is available in figures \ref[app_vis:sierraTop] and \ref[app_vis:sierraBottom]. 
\endinsert



%%%%% OBC CHARACTERISTICS

\sec OBC characteristics

\quad An OBC of a CubeSat is a complex subsystem combining various features. Reviewing technical documentation of already existing \glref{OBC}s (chapter \ref[chap:existingOBCmodules]) showed that some features are shared among most \glref{OBC}s, whereas some are unique. In this section, we list all of the features we decided to implement, both typical and VST104 related.



%%%%% RADIATION AND REDUNDANCY

\label[chap:radiationAndRedundancy]
\secc Radiation and redundancy

\quad Occasionally traveling through weaker parts of the Earth's magnetic field and not shielded by the Earth's atmosphere, the CubeSats have to operate in an environment full of radiation. A direct hit of a high-energy particle might have serious consequences for \glref{OBC} functionality. These include transistor gate ruptures, memory bit flips, software upsets, or latch-ups \cite[pap:radiationSurvey, app:badBoys, pap:memoryDamage]. \"Single event latch-ups and upsets can occur even in a low radiation environment like a polar \glref{LEO} orbit." \cite[pap:approachToSpace] Proper measures have to be taken to increase the \glref{OBC} durability and ability to handle errors, resulting in maximizing a mission lifetime. This process is known as radiation hardening (its results are labeled as {\it rad-hard} or {\it radiation-hardened}) and can be achieved by two different techniques.

A physical approach, also known as a radiation hardening by design, relies on special radiation-hardened components. Hardened \glref{IC}s are manufactured on insulating substrates (rather than typical semiconductor wafers) or use some of the many dedicated design principles \cite[pap:radiationSurvey]. Implementation of this strategy is typical for professional and more expensive satellites than the CubeSats. These components are usually bigger, costly, and have reduced performance and functionality compared to ordinary ones.

Considering the size and budget requirements of our \glref{OBC}, we chose to implement another option. Instead of the previously mentioned physical hardening technique, a logical one was realized. The \glref{OBC} hosts multiple schematic design features ensuring the proper handling of any radiation-related event. These include: i) over-current sensing power management, ii) separate peripheral isolators, iii) full high-impedance mode requested by the higher logic, iv) triple-redundant memories, or v) multiple onboard temperature sensors. A fully double-redundant \glref{OBC} is presented in chapter \ref[chap:boardDelta].



%%%%% FEATURES

\label[chap:capabilitesAndFeatures]
\secc Capabilities and features

\quad Having in mind the expectations of the \glref{VST} supervisors, features common for \glref{OBC}s by different manufacturers (chapter \ref[chap:existingOBCmodules]), and design requirements implied by the radiation (chapter \ref[chap:radiationAndRedundancy]), we set and implemented the following \glref{OBC}'s features:

\begitems
    * {\sbf Microcontroller:} An ultra-low-power STM32L496 ARM Cortex-M4 core microcontroller drives the \glref{OBC}. This \glref{MCU} can run up to $78\rm{[MHz]}$, is equipped with a $1\rm{[MB]}$ flash memory and $320\rm{[kB]}$ \glref{RAM}, and provides a wide external connectivity. A user can program and debug this \glref{MCU} using a separate \glref{SWI} or \glref{UART} connection.
    * {\sbf External clock sources:} The \glref{MCU} is equipped with two external clock sources for improved and more reliable clock generation and timing. Low-speed external $32.768\rm{[kHz]}$ oscillator and high-speed external $26\rm{[MHz]}$ oscillator. This oscillator can be temporarily disabled by the \glref{MCU} to achieve better power consumption.
    * {\sbf Robust power management:} Two separate power lines with $3.3\rm{[V]}$ and $5\rm{[V]}$ ratings supply the \glref{OBC}. A robust power management circuitry is present separately on these lines, providing tunable over and under-voltage protection, over-current protection, amplified current monitoring output, kill switching, and simultaneous power down.
    * {\sbf Isolation of the peripherals:} The \glref{OBC} can communicate with other CubeSat modules using a 2x \glref{CAN} bus, 3x \glref{$\rm{I^2C}$}, 4x \glref{UART}, 2x \glref{SPI}, 22 \glref{GPIO} pins, and 5 maintenance signals. All the peripherals are $5\rm{[V]}$ tolerant and connected to the PC/104 header thru separate analog switches. These isolators are independently controlled by the \glref{MCU} and provide a complete or selected disconnection from the rest of the CubeSat.
    * {\sbf Redundant external memory:} Two subsystems of triple redundant external memories are available for \glref{MCU}'s additional data storage. This triple redundancy ensures data validity in the radiation-rich environment, although the single-mode is also supported. The first memory subsystem is a 3x $32\rm{[MB]}$ FLASH connected via the Quad-\glref{SPI} interface, while the second one is a 3x $2\rm{[Mb]}$ F-ram using the \glref{SPI}.
    * {\sbf CAN bus peripherals:} Two \glref{CAN} transceivers provide the \glref{OBC}'s support of a high-speed \glref{CAN} bus.  These transceivers ensure a voltage level conversion ($3.3\rm{[V]}$ for the \glref{MCU}, $5\rm{[V]}$ for the CubeSat) and support communication speeds up to $250\rm{[kB\,s^{-1}]}$.
    * {\sbf Temperature monitoring:} Seven \glref{$\rm{I^2C}$} temperature sensors are spread over the entire \glref{OBC}. These sensors monitor the temperature of essential submodules, 2x \glref{MCU}, 2x power management, 2x \glref{CAN} bus drivers, and 1x external memories. The \glref{MCU} can power on or off these sensors to save some power or resolve a latch-up even.
    * {\sbf Maximal payload sector:} The PCB surface occupied by the \glref{OBC} circuitry is shrank to the minimal size possible. The remaining area is considered as a payload sector that can accommodate any user-defined modules. In our case, we decided to place a universal soldering array with exposed power lines all over this sector.
\enditems



%%%%% HARDWARE ARCHITECTURE

\secc Hardware architecture

\quad In the rest of this thesis, we address each of the previously mentioned features as a separate \glref{OBC} subsystem. A diagram of the \glref{OBC} hardware architecture is shown in figure \ref[dia:hwArchitecture]. This diagram shows relations and used communication busses between all of the subsystems. For their detailed description, refer to the following chapter \ref[chap:boardSierraSubsystems].

\midinsert
    \clabel[dia:hwArchitecture]{Sierra HW architecture}
    \picw=0.75\hsize \cinspic figures/diagrams/hwArchitecture.pdf
    \caption/f Diagram of the Board Sierra's OBC hardware architecture.
\endinsert



%%%%% ELECTRONIC COMPONENTS

\label[chap:componentsSelection]
\sec Electronic components

\quad Electronic components are the building stones of each electronic hardware. Big commercial, military, or research spacecrafts use specialized components designed, modified, or particularly selected for space applications. CubeSats, on the other hand, use in most cases commercial off-the-shelf (\glref{COTS}) components that are not specifically designed for application within the space environment \cite[pap:tempModeling]. \"The use of \glref{COTS} components is certainly an attractive option for the development team of small satellite projects. Lead times are short, and recent developments in personal computing equipment and consumer electronic allow for high performance." \cite[boo:cubesatTemp] Furthermore, these components have proven their low risk to mission success \cite[pap:approachToSpace]. In the VST104 project, we have also decided to follow this trend and to use \glref{COTS} components. In this section, we describe various requirements applied to these components and the criteria of their selection.



%%%%% CERTIFICATION

\label[chap:componentsCertification]
\secc Components certification

\quad Spacecrafts are exposed to a harsh environment and its effects during their lifetime. This problem needs to be also addressed on the electronic components level by selecting components with adequate parameters. The two essential criteria that need to be watched are operational temperature range and mechanical stress resistance.

\"The temperature range that electronic components can encounter in orbit is quite large as the thermal control options are limited" \cite[boo:cubesatTemp]. \"The most crucial impact on the temperatures of a spacecraft is caused by the external radiation from Sun and Earth, as well as internal heat dissipation throughout electronic devices" \cite[pap:tempModeling]. Typical operational temperature for CubeSat components is in the range of $-40$ to $+85\rm{[^\circ C]}$ \cite[pap:tempModeling, boo:cubesatTemp], usually referred to as an industrial temperature range. As we wanted to be on the safe side, we decided to add a safety margin by extending the range. We chose the automotive range of $-40$ to $+125\rm{[^\circ C]}$ as a requirement for electronic components in our design.

\"During the launch, a spacecraft is subjected to various external loads resulting from vibroacoustic noise, booster ignition and burn out, propulsion system engine vibration, steady-state booster acceleration, and much more" \cite[pap:vibration]. Hence, various measures are applied to increase the robustness of the spacecraft. Regarding the \glref{OBC} design, it was necessary to select only the electronic components meeting the \glref{AEC}-Q100 and \glref{AEC}-Q200 standards. \"Components meeting these specifications are suitable for the harsh automotive environment without additional component-level qualification testing" \cite[nor:AEC].

To sum it up, all of the electronic components used in the Sierra Board i) are rated for the automotive temperature range of $-40$ to $+125\rm{[^\circ C]}$, and ii) obtained the \glref{AEC}-Q100 or \glref{AEC}-Q200 qualification. The only exception is the \glref{MCU} with no \glref{AEC} qualification.

Some of the \glref{OBC}'s potential applications may not require temperature and automotive certification of the electronics components. In such a case, some components may be replaced by their uncertified variants to save a bit of the financial budget. All of the electronics components with an official uncertified version are listed in table \ref[tab:uncertifiedParts]. Furthermore, the potential user is encouraged to replace all passive components (resistors, capacitors, ferrite beads, inductors) with unqualified parts with the same parameters. A replacement of oscillators Y[1,2] is possible, although this change is non-negligible.



%%%%% POWER CONSUMPTION

\label[chap:powerConsumption]
\secc Power consumption

\quad CubeSat's power budget is a closely monitored thing. Such a small spacecraft has a limited means of generating and storing electric power. Solar arrays are usually bound in their active surface and lack advanced tools of positioning. Similarly, battery storage systems are restricted by the available space and means of temperature control. It is crucial to minimize the power requirements of the CubeSat, including the \glref{OBC}. 

Reasonable power consumption of the \glref{OBC} can be achieved on the level of electronic components selection by prioritizing the low-power components. A downfall of this approach is usually an implied decrease in their performance, speed, or frequency. Therefore, we preferred components with similar parameters but lower power consumption. This affected mostly the selection process in cases of the \glref{MCU} and memory \glref{IC}s.

Not only a component's power consumption but also its efficiency is an important parameter. While choosing switching components such as electronic fuses, hot-side switches, or analog switches, a low on-resistance was preferred. Power losses on pull-down/pull-up resistors had to be also taken into consideration. A single $10\rm{[k\Omega]}$ pull-down resistor connected to a $3.3\rm{[V]}$ signal drains a constant current of $330\rm{[\mu A]}$, creating a power loss of approximately $1.1\rm{[mW]}$. During the schematic design, we aimed more at a subsystem's robustness rather than avoiding this particular type of power loss. Therefore, some of the pull-down/pull-up resistors may be left unpopulated and their functionality compensated by programmable resistors integrated inside the \glref{MCU}.

The summary of power consumption of the specific electronics components is listed in table \ref[tab:componentsConsumption]. Overall power consumption of specific \glref{OBC}'s submodules is listed in table \ref[tab:subsystemConsumption]. It sums power consumptions of all \glref{IC}s and losses on pull-down/pull-up resistors in the particular subsystem. For the computation, two assumptions were made: i) full functionality of the subsystem, ii) input voltage range as described in table \ref[tab:powerManagement].

\midinsert \clabel[tab:subsystemConsumption]{Subsystems consumption}
    \ctable{lcl}{
        \glref{OBC} subsystem & Typ. current $\rm{[mA]}$ & Note \crl
        Processing unit     & \phantom{.}20 @ 3.3[V] & at $80\rm{[MHz]}$ clock \cr
        Power management    & 0.3 @ \phantom{3}-\phantom{3}[V] & for each $3.3\rm{[V]}$ and $5\rm{[V]}$ bus \cr 
        Header isolation    & 0.6 @ 3.3[V] & every isolator is opened \cr
        FLASH memories      & \phantom{.}30 @ 3.3[V] & with one active memory \cr
        F-ram memories      & \phantom{3.}5 @ 3.3[V] & with one active memory \cr
        Dual \glref{CAN} bus & \phantom{.}40 @ \phantom{3.}5[V]  & for each active \glref{CAN} bus \cr
        Temp. monitoring    & 1.6 @ 3.3[V] & as a whole group \cr 
    }
    \caption/t A rough estimation of subsystems power consumption during a normal operation at room temperature. This table should be used for orientation purposes only. A potential user is encouraged to conduct a proper measurement for each \glref{OBC} application.
\endinsert



%%%%% ADDITIONAL PARAMETERS

\label[chap:additionalParams]
\secc Additional parameters

\quad A selection of particular electronic components is a vital part of designing circuitry. Usually, various electronic components share the same functionality and are available from different manufacturers. After filtering the available components by specific functionality (with respect to \ref[chap:componentsCertification] and \ref[chap:powerConsumption]), it was common to find various suitable candidates. Therefore, additional criteriums had to be set to resolve the selection:

\begitems
    * {\sbf Price:} As this \glref{OBC} is not primarily designed for an actual space flight (as explained in \ref[chap:intro]), an individual component's price is not negligible. With a lower overall cost of the \glref{OBC}, a broader project expansion in the LibreCube community can be achieved. Thus components with sufficient attributes but lower price were favored.
    
    * {\sbf Footprint:} As a result of maximizing the PC/104 payload sector, the actual \glref{OBC} area was significantly decreased. Therefore components of smaller dimensions available in fine-pitch packages (e.g. \glref{SSOP} or \glref{QFN}) were preferred. The same logic applies to passive components such as resistors and capacitors, resulting in a 0201 package (0603 in metric) being the most widely used in this design. After researching the technical capabilities and related costs of \glref{PCB} manufacturers, we decided not to use the \glref{BGA}-packaged components. Their significantly smaller footprints would require more precise fanout, resulting in increased manufacturing difficulty and price.
    
    * {\sbf Distributor:} We considered it important to use only typically stocked components available from the global distributors. In our case, Mouser electronic\urlnote{http://www.mouser.com/}.This rule should repeal any future problems in components sourcing or logistics. Therefore, the availability of a specific component in this distributor was a selection factor.
\enditems



%%%%% PCB FEATURES AND DESIGN

\sec PCB features and design

\quad Designing the Board Sierra's \glref{PCB} was probably the most demanding part of the VST104 project. Restricted surface available for the complex \glref{OBC}'s circuitry, compact footprint of the PC/104 main header with a not ideal location, and limited capabilities of the considered \glref{PCB}s manufacturers made the routing and fanout challenging. Visualization of the designed module rendered in KiCad is shown in figures \ref[app_vis:sierraTop] and \ref[app_vis:sierraBottom].



%%%%% PCB REQUIREMENTS

\secc PCB requirements

\quad The benefit of the Board Sierra's lower price regarding the particular electronic components was described in chapter \ref[chap:additionalParams]. The same logic applies to the decision-making during the \glref{PCB} design. It is a common feature between \glref{PCB}s manufacturers that the less advanced design, the cheaper \glref{PCB}. This includes many parameters such as copper layer count, presence of buried vias and their size, copper clearances, tracks widths, and much more. On the other hand, fine features such as buried vias simplify the process of \glref{PCB} design and allow more compact layouts. Therefore, it was crucial to find a balance between the manufacturing price and complexity of the \glref{OBC}'s layout and routing.

Although, following the proper design rules and standards created for automotive and especially space applications is much more important than lowering the \glref{PCB}'s manufacturing price. We addressed various requirements and suggestions listed in \glref{ECSS} standards ECSS-Q-ST-70-12C \cite[sta:ECSS-12C] and ECSS-Q-ST-70-60C \cite[sta:ECSS-60C]. Other precious recommendations were found and implemented from the TEC-ED IoD Board Specification \cite[sta:TEC-ED]. These specifications refer, for example, to track width and spacing, pad design, copper planes, or thermal rules. However, these documents are aimed at state-of-the-art spacecrafts designed and operated by the \glref{ESA}. Thus a punctual following of all of the requirements and suggestions was not necessary for our CubeSat application.



%%%%% OBC AND PAYLOAD SECTOR

\label[chap:payloadSector]
\secc OBC and payload sector

\quad As mentioned in chapter \ref[chap:capabilitesAndFeatures], a high ratio between the payload sector area and OBC area was set as one of the Board Sierra's key features. This requirement substantially affected the \glref{OBC}'s circuitry layout and fanout. After multiple iterations of components' placements and alignments, we have significantly reduced the required \glref{OBC} area. The payload sector covers roughly three-fourths of the available PC/104 module surface, leaving only one-fourth to accommodate the OBC. Visual perception can be obtained from the Board Sierra 3D renders in figures \ref[app_vis:sierraTop] and \ref[app_vis:sierraBottom]. The \glref{OBC} sector is situated at the northwest quartal of the PC/104 module, surrounded by the payload sector.



%%%%% PCB CHARACTERISTICS

\label[chap:pcbCharacteristics]
\secc PCB characteristics

\quad The\glref{PCB}'s dimensions, geometry, and layout of the main connector and mounting holes are fully specified by the PC/104 standard, as described in chapter \ref[chap:PC104standard]. The only modification added to this template are $1.9$ x $20.3\rm{[mm]}$ cutouts on the module's four edges. These cutouts were introduced in the LibreCube board specification and are designed to accommodate CubeSat's auxiliary power and data cables \cite[sta:libreBoard].

Regarding the manufacturing price and complexity, the\glref{PCB} would ideally be a four-layer one. This means two copper layers for signal traces, a power distribution layer, and a ground plane. Unfortunately, the two signal layers would not be enough to accommodate complex \glref{OBC}'s circuitry. Therefore, we had to choose a six-layer\glref{PCB} with the following setup: i) signal layers: top, second inner, bottom; ii) ground plane layers: first inner, fourth inner; iii) power distribution layers: third inner. \"An additional benefit of the added layers is improved thermal dissipation, as adding a layer of copper to the circuit board can significantly decrease the resulting temperatures" \cite[pap:tempModeling].

Additional parameters of the \glref{PCB} manufacturing such as material, isolation and copper thickness, surface finish and solder mask types are not specified in this thesis nor the VST104 project. It is the responsibility of the potential user to customize these parameters accordingly to the requirements of the specific application. In such a case, we encourage the user to follow relevant sections of the \glref{ECCS} standard \cite[sta:ECSS-12C].



%%%%% TRACES, VIAS AND ROUTING

\secc Traces, vias and routing

\quad Electronic components are placed on both sides of the \glref{PCB} due to the restricted OBC area. Corresponding top and bottom copper layers cover as much routing as possible. The second inside layer accommodates traces that do not fit into the two surface layers. The general width of a signal trace is $0.2\rm{[mm]}$, respectively $0.173\rm{[mm]}$ for traces near the PC/104 header. Clearance between traces themselves and other copper planes is set to $0.127\rm{[mm]}$. All of these parameters satisfy the minimal specifications in the \glref{ECSS} standard \cite[sta:ECSS-12C]. No blind, buried or micro vias are used in the design. The diameter of general via is $0.45\rm{[mm]}$ with a $0.24\rm{[mm]}$ drill. This parameters result in a $6.7$ aspect ratio for a standard $1.6\rm{[mm]}$ thick \glref{PCB}, which recognizes the \glref{ECSS} $\leq7$ rule \cite[sta:ECSS-12C].

The first and fourth inner layers are ground plane layers. Their purpose is to connect all of the component's \glref{GND} pins, provide shielding against the \glref{EMI} and act as heatsink dissipating components heat. Despite the \glref{ECSS} suggestion, both the ground planes have a solid fill instead of additional openings in a grid format \cite[sta:ECSS-12C]. The reason is, the current fifth version of KiCad does not support this option. This design flaw will be improved shortly as the checked fill option is included in the upcoming KiCad v6.

A copper plane with a solid fill is also located in the third inside layer. This plane is attached to a $3.3\rm{[V]}$ power source, and its task is to connect all required components to this power bus. This layer also accommodates a $5\rm{[V]}$ power distribution routing. Its traces are $0.45\rm{[mm]}$ wide with multiple vias to minimize the potential voltage drop.

The final \glref{PCB} routing highlighted by the particular \glref{OBC}'s subsystems is shown throughout multiple figures in chapter \ref[chap:boardSierraSubsystems]. General principles of good routing were implemented throughout the \glref{PCB} design. These include steep turns avoiding, differential pairs matching, or sliver and peelable prevention \cite[sta:ECSS-60C]. If available, fanout and routing suggestions included in electronic components datasheets were followed. As suggested by the \glref{ECSS} \cite[sta:ECSS-12C], a teardrop finish was applied to all of the pads using the KiCad Teardrops extension listed in chapter \ref[chap:openSource]. It is important to mention that the current KiCad v5 does not support advanced design settings common for professional software such as Altium Designer or CadSoft EAGLE. Our approach to handling this problem was adding extra margins to the few parameters listed in KiCad, which increased the overall tolerance. The final \glref{PCB} design passed a design rule check (\glref{DRC}) with various parameters set according to manufacturers' standards and \glref{ECSS} suggestions \cite[sta:ECSS-60C].
 % skontrolovane

\chap Board Sierra - submodules



%%%%% MICROCONTROLLER

\sec Microcontroller

\quad The processing unit is the most important part of the \glref{OBC}. Professional designs use a wide variety of different instruction set architectures \cite[obc:dataPatterns, obc:imt, obc:pumpkin, obc:gauss, obc:nanosatpro]. However, the \glref{ARM} architecture seems to be more popular \cite[obc:isis, obc:aac, obc:anteleope, obc:gos]. This architecture is known for its good multiprocessing support, low power consumption, affordable pricing, and broad spectrum of existing applications. Considering these benefits and influenced by the LibreCube and TUDSaT, our \glref{VST} supervisors decided to pick an STM32 \glref{MCU}.

Our task was to choose a particular model of this 32-bit, Arm and Cortex-M based \glref{MCU}. Aiming rather for a low-power than high-performance characteristics, we decided to select an L series. Particularly the L4 series as it combines the largest flash memory size with the highest number of general-purpose input/output pins (\glref{GPIO}s) \cite[dat:mcuSeries]. Although, only one device from the L4 series is equipped with two CAN bus channels. As the presence of a second bus is crucial for our double-redundant approach, the STM32L496xx option was selected. This family comes in six different packages. Avoiding all of the \glref{BGA}-like ones (as described in chapter \ref[chap:componentsSelection]) narrows the selection to Zx, Vx, and Rx variants. The Zx is the most capable one in terms of flash size and \glref{GPIO} count. Therefore the STM32L496ZG (where G stands for extended operating temperature range) was our final choice \cite[dat:mcu]. From now on, we will refer to this particular device as the \glref{MCU}. Some of its key characteristics are listed in table \ref[tab:microcontroller].

\midinsert \clabel[tab:microcontroller]{Microcontroller characteristics}
    \ctable{lcc|lcc}{
        Max. frequency      & $80\;\rm{[MHz]}$            &&& \glref{SPI}     & $3$     \cr
        Flash memory        & $1\;\rm{[MB]}$              &&& \glref{I2C}     & $4$     \cr
        Static RAM          & $320\;\rm{[kB]}$            &&& \glref{UART}    & $5$     \cr
        Comparators         & $2$                         &&& \glref{CAN}     & $2$     \cr
        Op. amplifiers      & $2$                         &&& \glref{GPIO}    & $115$   \cr
        Temperature         & $-40$ to $125\;\rm{[°C]}$   &&& \glref{DAC}     & $2$
    }
    \caption/t Highlighted characteristics of the STM32L469ZG microcontroller \cite[dat:mcu].
\endinsert

\secc Schematic design

\quad The \glref{MCU} pin assignment was continuously changing during the entire process of \glref{OBC} schematic and PCB design. Its final state is presented in figure \ref[app_sch:microcontroller]. We attempted to maximize the number of user-free \glref{GPIO}s with added functionalities such as \glref{ADC} or \glref{PWM} while keeping the fanout manageable. A significant help during this process was the CubeMX tool of the STM32CubeIde, visualizing all of the pinout combinations with a specific functionality. Each of the $3.3\rm{[V]}$ tolerant pins was used only for the internal circuitry, resulting in a fully $5\rm{[V]}$ tolerant main PC104 header connection.

A significant number of filtering and reservoir capacitors is needed to ensure the proper \glref{MCU} functionality. The assignment of correct capacitors to required \glref{MCU} pins was pretty straightforward, following the device datasheet \cite[dat:mcu] and STM32L4 hardware development application note \cite[app:getStart]. Furthermore, a $10\rm{[\mu H]}$ choke and $120\rm{[\Omega]}$ at $100\rm{[MHz]}$ ferrite bead were placed in series with the analog power input. This \glref{LC} filter supported by the bead should effectively eliminate both low and high-frequency interference.

The clock and data lines of both \glref{I2C} busses are equipped with standard $4.7\rm{[k\Omega]}$ resistors. Multiple $22\rm{[\Omega]}$ resistors were placed in series with high-speed clock signals of the \glref{SPI} interface. Two $0\rm{[\Omega]}$ resistors are present on {\it FAULT} and {\it MODE} lines to facilitate an optional hardware isolation. A $10\rm[k\Omega]$ resitor at {\it PH3} is suggested by \cite[app:getStart].

\secc PCB design

\quad Location and fanout of the \glref{MCU} are shown in figure \ref[pcb:microcontroller]. The \glref{MCU} is placed on the \glref{PCB} top side, covering most of the non-payload area. This big footprint of a LQFP144 package (approximately $2\rm{x}2\rm{[cm]}$) is a consequence of avoiding the \glref{BGA} packages while still trying to keep the pin count high. All of the capacitors were placed as close to their assigned pins as possible. In some cases, it was necessary to use the bottom side of the \glref{PCB}. The same approach was applied to all of the resistors.

\midinsert
    \clabel[pcb:microcontroller]{Microcontroller circuitry}
    \picw=0.85\hsize \cinspic figures/pcbs/microcontroller.pdf
    \caption/f Highlighted location of microcontroller circuitry.
\endinsert



%%%%% EXTERNAL CLOCK SOURCES

\sec External clock sources

\quad Proper timing and synchronization are the key features while dealing with high-speed data busses, \glref{ADC}s, or other precise applications. The \glref{MCU} is equipped with two internal \glref{RC} oscillators that can be used to drive a master system clock and other auxiliary clocks \cite[dat:mcu]. These internal oscillators generally have a significantly lower frequency stability,  a higher temperature dependency, and smaller overall accuracy than their external equivalents \cite[app:oscInt]. Therefore, to increase the clock precision and reliability in the harsh space environment, we had to implement external clock sources.

A $4-48\rm{[MHz]}$ high speed external oscillator (\glref{HSE}) can drive the system clock. Supported types are crystal, ceramic resonator, or silicone oscillator \cite[dat:mcu]. The last option seems to be the best as it is insensitive to electromagnetic interference (\glref{EMI}) and vibration. The only downside is its slightly lower temperature rejection \cite[app:oscComp]. We chose the SiT8924B, a $26\rm{[MHz]}$ silicon microelectromechanical system (\glref{MEMS}) oscillator \cite[dat:hse]. Accordingly to the clock configuration tool of the stm32cube software, we can reach various system clock and auxiliary clocks frequencies up to $78\rm{[MHz]}$ (maximum is $80\rm{[MHz]}$ \cite[dat:mcu]). This is done using the phase-locked loop (\glref{PLL}) clock generation.

A $32.768\rm{[kHz]}$ low speed external oscillator (\glref{LSE}) can drive the real time clock (\glref{RTC}), hardware auto calibration, or other timing functions \cite[dat:mcu]. Table 7 in \cite[app:oscDes] recommends individual crystal resonators for combination with STM32 \glref{MCU}s. After a consideration of these options, we decided to pick the ABS07AIG ceramic base crystal \cite[dat:lse].

\secc Schematic design

\quad The \glref{HSE} circuitry follows the oscillator datasheet \cite[dat:hse]  and is shown on the right side of figure \ref[sch:clockSources]. Only a decoupling capacitor and a terminator resistor are required. The clock output {\it OSE_IN} can be enabled or disabled by the binary {\it OSC_EN} signal.

 The \glref{LSE} circuitry is based on a reference design in the oscillator design guide \cite[app:oscDes]. To achieve a stable frequency of this Pierce oscillator, it is required to find the values of load capacitors $C_{L1}$, $C_{L2}$ and external resistor $R_{E}$. This can be done using equations
$$ C_L = {{C_{L1} C_{L2}}\over{C_{L1}} + C_{L2}} + C_S \quad \wedge \quad C_{L1} = C_{L2}, \eqmark $$
$$ R_{E} = {1\over{2\pi f C_{L2}}}, \eqmark $$
where $C_L$ is the crystal load capacitance, $f$ is the crystal oscillation frequency and $C_S$ is the stray capacitance \cite[app:oscDes]. Values of $C_L$ and $f$ are listed in the crystal datasheet \cite[dat:lse]. We can assume as a rule of thumb, that $C_S=4\rm{[pF]}$. The final \glref{LSE} circuitry with computed values of the components is shown on the left side of figure \ref[sch:clockSources].

\midinsert
    \clabel[sch:clockSources]{Clock sources schematic}
    \picw=0.5\hsize \cinspic figures/schemes/clockSources.pdf
    \caption/f Schematic diagram of external clock sources.
\endinsert

\secc PCB design

\quad Location and fanout of the external clock circuitry are shown in figure \ref[pcb:clockSources]. All of the components are placed on the bottom side of the \glref{PCB}. The circuitry layout follows multiple tips presented in the oscillator design guide \cite[app:oscDes]. Separate \glref{GND} planes are assigned to both the \glref{HSE} and \glref{LSE} circuitry. These planes are bounded by guard rings, formed by series of vias. Each of the planes is connected to a common \glref{GND} only at one point. This approach provides proper \glref{EMI} shielding while reducing a ground loop effect. We also minimized the distance between the \glref{MCU} pins and both oscillators. All these measures combined should improve the clock generation stability and robustness.

\midinsert
    \clabel[pcb:clockSources]{Clock sources circuitry}
    \picw=0.85\hsize \cinspic figures/pcbs/clockSources.pdf
    \caption/f Highlighted location of external clock sources circuitry.
\endinsert



%%%%% POWER MANAGEMENT

\sec Power management

\quad The electric power subsystem (\glref{EPS}) is known to be the most vital subsystem of a spacecraft. Its reliability and error handling should be ensured by the power control and distribution unit (\glref{PDCU}). However, it is a good practice by professional manufacturers to include an additional power monitoring and control to their \glref{OBC} module designs \cite[obc:dataPatterns, obc:isis, obc:imt, obc:gos, obc:pumpkin, obc:gauss, obc:aac, obc:anteleope, obc:nanosatpro]. We have to implement circuitry that can sense a power bus malfunction and, as a response, power down the \glref{OBC}. This feature is also important for some of the \glref{OBC} on-earth user cases. During a hardware development or a system presentation, the user might misconnect the power line or use an unsupported power source by a mistake.

\midinsert
    \clabel[dia:powerManagement]{Power management diagram}
    \picw=0.75\hsize \cinspic figures/diagrams/powerManagement.pdf
    \caption/f Functional diagram of power management circuitry.
\endinsert

A functional diagram of the implemented power management is shown in figure \ref[dia:powerManagement]. To increase the overall efficiency and decrease complexity, we decided to avoid any voltage conversion independent from the \glref{PDCU}. Therefore, our \glref{OBC} requires two separate inputs from the main power buss ($3.3\rm{[V]}$ and $5\rm{[V]}$). The \glref{OBC} is connected to each of these inputs through an electronic fuse (\glref{eFuse}). This device continuously monitors the bus for events of under-voltage, over-voltage, and over-current\fnote{This is a crucial feature in handling and resolving a latch-up event.}. As a response to such an event, the \glref{eFuse} will switch into high impedance and pull down specific input of an AND logic gate. This gate simultaneously controls two load switches, one for each power line. This approach ensures that a fault on one power bus will result in a high impedance of both \glref{OBC} power inputs. It also eliminates the risk of a death loop, in which a reset of \glref{eFuse}s is not possible as they are switching each other off. Added benefits of this design are an inbuild current measurement and a Kill Switch integration into the logic gate. A summary of the final power management rating is listed in table \ref[tab:powerManagement]. These values were chosen considering the power requirements of the remaining \glref{OBC} components and are a subject of change by a future user.

\midinsert \clabel[tab:powerManagement]{Power management rating}
    \ctable{lccccc}{
        Power input      & Parameter & Min & Typ & Max & Unit \crl
        \vspan2{3V3 BUS} & voltage   & 2.9 & 3.3 & 3.5 & $\rm{V}$ \cr
                         & current   & 0.0 & -   & 1.2 & $\rm{A}$ \crl
        \vspan2{5V BUS}  & voltage   & 4.6 & 5.0 & 5.4 & $\rm{V}$ \cr
                         & current   & 0.0 & -   & 1.2 & $\rm{A}$
    }
    \caption/t \glref{OBC} power rating. Value out of range will cause a protective shutdown.
\endinsert

\secc Schematic design

\quad An actual schematic diagram of power management circuitry is shown in figure \ref[sch:powerManagement]. The most important part of this design is the \glref{eFuse}, as it covers all of the power control features. We decided to use the TPS25940-Q1 device \cite[dat:efuse]. Custom  threshold values can be set following the typical application schematic in the datasheet \cite[dat:efuse]. This is done by connecting specific resistors, with values calculated using the TPS2594x design calculation tool \cite[app:tps2594Calc]. As the logic AND gate, we chose the 74LVC1G11-Q100 \cite[dat:gate]. This device is designed to operate in a mixed logic level environment, what corresponds with our application. The last important component is the load switch. In our case, the TPS22965W-Q1 with an inbuilt output discharge function \cite[dat:switch]. For a correct operation of the switch, the {\it VBIAS} pin should stay saturated for a while after disconnecting the {\it VIN} voltage. We achieved this behavior by charging a capacitor connected to the {\it VBIAS} from the {\it VIN} through a Schottky diode. Four reservoir capacitors are placed on both sides of the load switches, following the suggestions in both datasheets \cite[dat:efuse, dat:switch]. Nominal logic values of all switching signals are set by pull-up or pull-down resistors.

\midinsert
    \clabel[sch:powerManagement]{Power management schematic}
    \picw=0.95\hsize \cinspic figures/schemes/powerManagement.pdf
    \caption/f Schematic diagram of power management circuitry.
\endinsert

\secc PCB design

\quad Location and fanout of the power management circuitry are shown in figure \ref[pcb:powerManagement].  All of the power management components are placed on the top side of the \glref{PCB}. Similarly, all of the power tracks are on the top copper layer. Only a few signal traces are running within the second or the bottom layer. Both the $3.3\rm{[V]}$ and $5\rm{[V]}$ control circuitry share the same layout and tracing regarding recommendations presented in the \glref{eFuse} and load switch datasheets \cite[dat:efuse, dat:switch]. The $3.3\rm{[V]}$ part is situated closer to the main PC104 header, while the $5\rm{[V]}$ part is right below it. Both \glref{eFuse}s are connected to the main PC104 header power pins through strengthened $0.77\rm{[mm]}$ traces in the third copper layer. Considering a standard copper thickness of $35\rm{[\mu m]}$, each of the input traces is rated to deliver up to $2\rm{[A]}$ of current. The logic gate with associated pull-down resistors is located above all of the remaining circuitry, closest to the main PC104 header.

\midinsert
    \clabel[pcb:powerManagement]{Power management circuitry}
    \picw=0.85\hsize \cinspic figures/pcbs/powerManagement.pdf
    \caption/f Highlighted location of power management circuitry.
\endinsert



%%%%% PERIPHERAL ISOLATORS

\sec Peripheral isolators

\quad The spacecraft \glref{OBC} is connected to many data buses shared among all other subsystems. In some scenarios, the \glref{OBC} must be able to isolate itself from a specific or multiple data buses. For example, to switch between \glref{OBC}s in a redundant configuration, to handle a failure on a data bus, or to prevent unintentional interference. Standard approaches to address this feature are based on using analog switches \cite[the:wurzburg], optocouplers \cite[pap:satelliteLeo], or \glref{FPGA}s \cite[obc:dataPatterns]. Furthermore, these isolators should also guarantee that all data lines are in a high impedance state when the \glref{OBC} is powered off.

After a brief survey, we decided to implement the design using robust analog switches. This approach is more straightforward, less expensive, and requires a smaller \glref{PCB} area than the optocoupler or the \glref{FPGA}-based ones. A functional diagram of the implemented circuity is shown in figure \ref[dia:peripheralIsolators]. All of the \glref{OBC} data lines are connected to the rest of the spacecraft through a series of analog switches. These data lines are grouped by particular data buses and are assigned to a separate switch. The \glref{OBC} can enable or disable a specific switch and therefore isolate a particular data bus from the remaining spacecraft subsystems. Pulling low the Kill Switch will result in high impedance of all switches and completely isolating the \glref{OBC} data lines.

\midinsert
    \clabel[dia:peripheralIsolators]{Peripheral isolators diagram}
    \picw=0.8\hsize \cinspic figures/diagrams/peripheralIsolators.pdf
    \caption/f Functional diagram of peripheral isolators circuitry.
\endinsert

\secc Schematic design

\quad Schematic diagrams of two isolators are shown in figure \ref[sch:peripheralIsolators]. We decided to use the DGQ2788A device \cite[dat:isolator]. To cover all of the data buses, the \glref{OBC} hosts fifteen of these analog switches in a dual double pole double throw (\glref{DPDT}) configuration. The \glref{OBC} data lines are connected to common terminals ({\it COM}). Normally closed terminals ({\it NC}) are left floating, whereas normally open terminals ({\it NO}) terminals are connected to the spacecraft data lines. The important Kill Switch functionality is implemented using the device's power down protection. If the switch loses power, it will enter the normal state. This approach simplifies the circuitry a lot as it substitutes an otherwise necessary system of multiple logic gates. Other beneficial features of this analog switch are a high latch-up current of $300\rm{[mA]}$ and inbuild signal clamping. The device will clamp all of the signals exceeding its supply voltage by internal diodes. As the \glref{MCU} pins connected to these analog switches are $5\rm{[V]}$ tolerant, we chose to power the switches from the $5\rm{[V]}$ power bus. A potential problem could be caused by the switch's enable terminals ({\it EN}), as they do not include internal pull-down or pull-up resistors. We decided to use hardware pull-down resistors to avoid a floating state of these control signals.  The placement decoupling capacitors follows the device datasheet \cite[dat:isolator].

\midinsert
    \clabel[shc:peripheralIsolatorsS]{Peripheral isolators schematic}
    \picw=0.92\hsize \cinspic figures/schemes/peripheralIsolators.pdf
    \caption/f Schematic diagram of two peripheral isolators.
\endinsert

\label[secc:peripheralIsolatorsPCBDesign]
\secc PCB design

\quad Location and fanout of the isolators circuitry are shown in figure \ref[pcb:peripheralIsolators]. Sixty separate signals are running from the \glref{MCU} pins through analog switches up to assigned pins in the PC104 header. Hence this part of the \glref{PCB} was the most challenging to design. The switches are placed in two main rows, each on one side of the \glref{PCB} (only two switches are not aligned). The position of every switch was determined by its assigned \glref{MCU} and PC104 header pins. It took several iterations to find out the current layout. The routing network is quite dense, using all three copper signal layers. To accommodate all of the signal traces, the standard signal trace width was decreased from $200\rm{[\mu m]}$ to $173\rm{[\mu m]}$. This new value was acquired as a maximal width possible to squeeze three traces between two pins of the PC104 header. To ensure \glref{CAN} buses signal integrity, we addressed a length matching of its differential pairs. As a finishing step of the routing, lengths of separate \glref{CAN} traces were measured and tuned with serpentine patterns.

\midinsert
    \clabel[pcb:peripheralIsolators]{Peripheral isolators circuitry}
    \picw=0.85\hsize \cinspic figures/pcbs/peripheralIsolators.pdf
    \caption/f Highlighted location of peripheral isolators circuitry.
\endinsert



%%%%% EXTERNAL MEMORY

\sec External memory

\quad 


\secc Schematic design

\midinsert
    \clabel[sch:externalMemory]{External memory schematic}
    \picw=1\hsize \cinspic figures/schemes/externalMemory.pdf
    \caption/f Schematic diagram of external memory circuitry.
\endinsert

\secc PCB design

\midinsert
    \clabel[pcb:ExternalMemory]{External memory circuitry}
    \picw=0.85\hsize \cinspic figures/pcbs/externalMemory.pdf
    \caption/f Highlighted location of external memory circuitry.
\endinsert



%%%%% CAN BUS DRIVERS

\sec CAN bus drivers

\quad As described in chapter {\Magenta ???}, the SpaceCAN is considered a primary control and monitoring bus of a LibreCube spacecraft. Therefore it was required to ensure its full support by the \glref{OBC}. An external \glref{CAN} transceiver is usually added to a microcontroller, as its internal physical layer has only limited properties or does not even exists. The separate transceiver provides a stable and reliable physical environment. In our case, the \glref{MCU}'s BxCAN is compatible with both 2.0A and 2.0B \glref{CAN} specifications with a bit rate up to $1\rm{[MB\,s^{-1}]}$ \cite[dat:mcu]. As the \glref{MCU}'s \glref{CAN} drivers are equipped with the 2nd network layer only, an external transceiver implementation to our design was required.

\secc Schematic design

\quad A schematic diagram of implemented \glref{CAN} driving circuitry is shown in figure \ref[sch:canBusDrivers]. As the \glref{MCU} supports two independent \glref{CAN} busses, we had to accommodate each of them. For the \glref{CAN} transceiver, we decided to use a TCAN1051V-Q1 device \cite[dat:canDriver]. The \glref{Rx/Tx} lines of the \glref{MCU} are connected to the device with a $22\rm{[\Omega]}$ terminating resistors. A test point is also present on each of these lines to assist potential debugging. This version of the device comes with a level shifting feature. Different voltage levels on \glref{CAN} and \glref{Rx/Tx} sides are supported. Since the SpaceCAN is a $5\rm{[V]}$ bus and the \glref{MCU} operates at $3.3\rm{[V]}$, we set the device's power levels accordingly. As recommended in the device datasheet, decoupling capacitors were added to power pins. Considering the \glref{CAN} bus's importance, we expect it to stay active straight from powering on the \glref{OBC}. To save some complexity and \glref{MCU} pins,  we decided to ignore the option of controlling the device standby mode. A pull-down resistor on the {\it STBY} pin forces an active mode.

A well-designed \glref{CAN} network usually contains a terminating resistor, filtering, and a transient \& \glref{ESD} protection. To correctly implement these optional features, we followed application reports \cite[app:canDesign, app:canChokes]. Instead of a simple $120\rm{[\Omega]}$ terminating resistor, we chose a more advanced terminating node. A difference is in added filtering as the node consists of two $60\rm{[\Omega]}$ series resistors connected to a \glref{GND} through a $4.7\rm{[nF]}$ capacitor. Furthermore, $100\rm{[pF]}$ filtering capacitors were added to signal lines. Recognizing the suggestions in the reports, we did not include any common-mode chokes or \glref{ESD} protection in our design. Since our \glref{OBC} is not intended to operate near heavy machinery, no extra improvement of susceptibility to electromagnetic disturbance or \glref{EMC} is required.

\midinsert
    \clabel[sch:canBusDrivers]{CAN drivers schematic}
    \picw=0.62\hsize \cinspic figures/schemes/canDrivers.pdf
    \caption/f Schematic diagram of CAN bus driving circuitry.
\endinsert

\secc PCB design

\quad Location and fanout of the \glref{CAN} bus driving circuitry are shown in figure \ref[pcb:canDrivers]. The transceivers are placed on the \glref{PCB} top side in two different locations. Prioritizing the \glref{PCB} surface's optimal usage,  we were unable to keep the devices in a mutual area. The transceiver circuitry layout follows suggestions in the datasheet \cite[dat:canDriver] and the application report \cite[app:canDesign]. This layout is the same for both devices. As mentioned in section \ref[secc:peripheralIsolatorsPCBDesign], serpentine patterns were used to match trace lengths of particular differential pairs.

\midinsert
    \clabel[pcb:canDrivers]{CAN drivers circuitry}
    \picw=0.85\hsize \cinspic figures/pcbs/canDrivers.pdf
    \caption/f Highlighted location of CAN bus drivers circuitry.
\endinsert



%%%%% TEMPERATURE MONITORING

\sec Temperature monitoring

\secc Schematic design

\midinsert \clabel[tab:temperatureMonitoring]{Temp. sensors description}
    \ctable{cclcc}{
        Designator  & \mspan2[l]{Targeted component}    & Slave addr.   & \glref{I2C} addr. \crl
        TS1         & M2  & - Flash memory              & $000$         & $0\rm{x}18$       \cr
        TS2         & U1  & - \glref{MCU}, central west & $100$         & $0\rm{x}1C$       \cr
        TS3         & EF2 & - $5\rm{[V]}$ eFuse         & $001$         & $0\rm{x}19$       \cr
        TS4         & U1  & - \glref{MCU}, north east   & $110$         & $0\rm{x}1E$       \cr
        TS5         & U3  & - \glref{CAN}2 driver       & $010$         & $0\rm{x}1A$       \cr
        TS6         & EF1 & - $3.3\rm{[V]}$ eFuse       & $011$         & $0\rm{x}1B$       \cr
        TS7         & U2  & - \glref{CAN}1 driver       & $111$         & $0\rm{x}1F$       \cr
    }
    \caption/t List of temperature sensors location and addresses.
\endinsert

\midinsert
    \clabel[sch:temperatureMonitoring]{Temp. monitoring schematic}
    \picw=1.\hsize \cinspic figures/schemes/temperatureMonitoring.pdf
    \caption/f Schematic diagram of temperature monitoring circuitry.
\endinsert

\secc PCB design

\midinsert
    \clabel[pcb:temperatureMonitoring]{Temp. monitoring circuitry}
    \picw=0.85\hsize \cinspic figures/pcbs/temperatureMonitoring.pdf
    \caption/f Highlighted location of temperature monitoring circuitry.
\endinsert



%%%%% PROGRAM PORT

\sec Programming port

\secc Schematic design

\midinsert
    \clabel[sch:programPort]{Programming port schematic}
    \picw=0.5\hsize \cinspic figures/schemes/programPort.pdf
    \caption/f Schematic diagram of programming port circuitry.
\endinsert

\secc PCB design

\midinsert
    \clabel[pcb:programPort]{Programming port circuitry}
    \picw=0.85\hsize \cinspic figures/pcbs/programPort.pdf
    \caption/f Highlighted location of programming port circuitry.
\endinsert

\midinsert
    \clabel[mod:programPort]{Programming port pinout}
    \picw=0.45\hsize \cinspic figures/model/programPort.pdf
    \caption/f Programming port pinout.
\endinsert
 % skontrolovane

%%%%% BOARD DELTA

\label[chap:boardDelta]
\chap Board Delta

\quad In this chapter, we address the Board Delta - a double redundant \glref{OBC} module in the PC/104 format. We explain the motivation and expectation behind the redundant systems while providing description and comments on the schematic and \glref{PCB} module design. A 3D visualization of the Board Delta is shown in figure \ref[img:delta_miniature].

\midinsert
    \clabel[img:delta_miniature]{Board Delta 3D render}
    \picw=0.95\hsize \cinspic figures/model/delta_miniature.pdf
    \caption/f 3D render of the Board Delta from the top (left) and the bottom (right) side.
\endinsert



%%%%% MOTIVATION

\sec Motivation and expectation

\quad During their launch and operation, Cubesats have to perform under extreme conditions of the surrounding environment. The most critical impacts were briefly described in the previous chapters, namely: i) damage of semiconductor devices caused by radiation (section \ref[chap:radiationAndRedundancy]), ii) exposure to wide temperature gradients and problems related to limited heat dissipation, and iii) significant mechanical stress generated by a launcher vehicle (both in section \ref[chap:componentsCertification]). Together with the practically non-existing possibility of maintenance, these problems are the key factors of a high failure rate of the already executed missions. As we have listed in section \ref[chap:existingOBCmodules], every second launched Cubesat has experienced a fatal failure by 2016, from which 20\% were caused by the \glref{OBC}s.

Designing the particular modules with multiple layers of redundancy is a common method of increasing the spacecraft's overall reliability \cite[pap:failRate2016]. \"By implementing onboard redundancy, CubeSats can meet or exceed their mission life, providing additional science data and post-mission payload testing." \cite[pap:approachToSpace] Together with telecommunications, attitude determination, and electrical power, the \glref{OBC} is one of the critical modules with high requirements for redundancy \cite[pap:satelliteLeo]. Some of the professional \glref{OBC} manufacturers have also included elements of redundancy into their designs \cite[obc:dataPatterns, obc:isis, obc:imt, obc:gauss, obc:nanosatpro]. Interestingly, the fairly popular CubeSat \glref{OBC} Kit from Pumpkin \cite[obc:pumpkin] is not one of them and its non-redundant architecture was pointed out as one of the two major weaknesses \cite[pap:reliability].

Although the Board Sierra implements some features of redundancy (triple external memories and dual peripheral data buses), it is still a single \glref{OBC} module. In a case of permanent damage to the power management or the \glref{MCU}, the Cubesat's mission would be severely jeopardized. However, this scenario can be eliminated by including another \glref{OBC} into the module design. Each of these two \glref{OBC}s is fully independent and has its own circuitry. Such a double redundant \glref{OBC} module can simply switch between the two \glref{OBC}s if one of them undergoes a serious malfunction. This approach was, for example, implemented in the professional DP-OBC-0402 by Data Patterns \cite[obc:dataPatterns].



%%%% DOUBLE REDUNDANT DESIGN

\sec Double redundant design

\quad From the beginning of the VST104 project, we thought about the Board Delta as a hardware merge of the two Board Sierras. The intention was to share the same \glref{OBC} design between the two redundant \glref{OBC}s at the Board Delta and also with the \glref{OBC} on the Board Sierra. This approach of the only one \glref{OBC} design has several advantages. Firstly, the two identical \glref{OBC}s on the Board Delta can run almost the same software and provide the same functionality to the spacecraft. After an emergency switch to the other \glref{OBC}, the Cubesat can continue to operate normally without changing the mission plan. Secondly, it allows an easy project migration between the boards. A potential user can develop and test his/her setup on the cheaper Board Sierra and then move to the more expensive Board Delta. Lastly, the approach also simplifies the development process and future maintenance of the VST104 project. It is faster and cheaper to develop and test out only one \glref{OBC} design and then integrate it into different modules. Also, if an \glref{OBC} design flaw or a possible improvement is found on one of the modules, it can be automatically implemented to the remaining modules.



%%%%% SCHEMATIC DESIGN

\secc Schematic design

\quad As the idea was to use the already existing \glref{OBC} design (described in chapters \ref[chap:boardSierra] and \ref[chap:boardSierraSubsystems]), the creation of the Board Delta schematic was straightforward. The KiCad project of the Board Delta consists of one primary sheet and two sub-sheets. The primary sheet contains the PC/104 header with assigned global signals, while each sub-sheet includes one \glref{OBC} design copied from the Board Sierra. Power and peripheral input/outputs of these so-called Left and Right \glref{OBC}s are attached to the same global signals as the PC/104 header. The resulting configuration of the module is simple: peripheral isolators and inputs of the power managements are connected directly to the PC/104 header. The individual peripheral isolators and Kill Switch function of the power management are then used to isolate and power down one of the \glref{OBC}s.

Although we have stated multiple times that the \glref{OBC}s share the same design, changes to a few PC/104 header pin assignments were required. The watchdog signal {\tt CPU_WD_1} and the Kill Switch signal {\tt GLO_KS_1} were assigned to the Left \glref{OBC}, whereas their {\tt CPU_WD_2} and {\tt GLO_KS_2} variants were assigned to the Right \glref{OBC}.

It is also important to note that our design does not exclude the possibility of running both of the \glref{OBC}s simultaneously. In such a scenario, both \glref{OBC}s should be powered on and negotiate the use of the shared PC/104 busses. The \glref{OBC} not using a particular bus should isolate itself from it. The simultaneous operation can be beneficial in missions where significant amounts of data need to be processed quickly. Also, some Cubesats may require one primary \glref{OBC}, handling the housekeeping and mission control, together with a secondary \glref{OBC}, serving payloads such as scientific instruments or cameras.



%%%%% PCB DESIGN

\secc PCB design

\quad The decision to use only one \glref{OBC} design does not apply exclusively to the schematic design but also to the \glref{PCB} design. The same (or as similar as possible) \glref{PCB} layout and routing of the \glref{OBC}s would guarantee comparable mechanical, electrical, and thermal characteristics. As the desire to create the Board Delta was clear from the beginning, we count it into the design of the Board Sierra \glref{PCB}. Its \glref{OBC} was designed exclusively on the PC/104 module northwest quarter, leaving the northeast quarter empty.

We used several tricks to ensure the best possible similarity between the \glref{OBC}s layouts. As a building template of the Board Delta \glref{PCB}, we copied the final design of the Board Sierra. The \glref{OBC} design on the template was assigned to the Left \glref{OBC}. The layout for the Right \glref{OBC} was generated using the Replicate layout plugin (mentioned in section \ref[chap:kicadPlugins]). This plugin was capable of copying the entire \glref{OBC} design, including all footprints, traces, and vias. However, to match the alignment and pinout of the PC/104 header, we had to flip this copied layout. This changed the previous top and bottom sides of the \glref{OBC} and is why the \glref{MCU}s are each on a different \glref{PCB} side. 

Some manual adjustments and routing were although required to finish the \glref{PCB} design. The second and third inner copper layers were manually reversed back to meet the copper layers assignment from section \ref[chap:pcbCharacteristics]. Also, the debug connector and its circuitry were mirrored back and rerouted. Thanks to this adjustment, both debug connectors are located on the \glref{PCB} top side with the same orientation and pinout. However, the most challenging part was to manually trace all peripheral isolators to their required PC/104 pins. Multiple attempts and sacrifice of the northern edge cutout were required to accomplish so. The complex layout and routing of both \glref{OBC}s are shown in figure \ref[img:delta_kicad_cut]. The Left \glref{OBC} is located on the left side (northwest quarter), while the Right \glref{OBC} is located on the right side (northeast quarter) of the Board Delta.

\midinsert
    \clabel[img:delta_kicad_cut]{Board Delta KiCad}
    \picw=1\hsize \cinspic figures/pcbs/delta_kicad_cut.pdf
    \caption/f \glref{PCB} design of the Board Delta captured directly in the KiCad environment. Visible is only the upper part with the PC/104 header and both \glref{OBC}s.
\endinsert
 % skontrolovane


\label[chap:ElementFoxtrot]
\chap Element Foxtrot

\sec PCB specifications
\sec Power input handling
\secc Ordinary power source
\secc USB-C power source
\sec Voltage conversion
\sec PC104 modules slots
 % skontrolovane

%%%%% BOARD SIERRA TESTING

\chap Board Sierra - testing

%%%%% TESTING SOFTWARE

\sec Testing software




%%%%% RADIATION TESTING

\sec Radiation testing




%%%%% EXPERIMENT SETUP

\secc Experiment setup

\quad For the purpose of the radiation testing, the Board Sierra was extended with a couple of electronic sensors: i) an always-on 3D accelerometer and 3D gyroscope LSM6DS3, ii) a high-performance 3-axis magnetic sensors MMC5983MA (two devices), and iii) an integrated 6-axis motion processor with gyroscope and accelerometer MPU6050. All of these sensors were connected with the \glref{OBC} using an \glref{$\rm{I^2C}$} data bus. Each sensor was powered through a separate high-base current PNP transistor controlled by the \glref{MCU}. A proper power reset of each sensor could have been achieved by turning off this transistor and isolating the corresponding \glref{$\rm{I^2C}$} peripheral isolator. This feature was implemented in order to resolve a potential \glref{$\rm{I^2C}$} bus lockup or the sensor's latch-up.


\midinsert
    \clabel[pho:rez_setup]{}
    \picw=0.85\hsize \cinspic figures/photos/rez_setup.pdf
    \caption/f 
\endinsert


%%%%% EXPERIMENT RESULTS

\secc Experiment results

\quad Being a complex system of multiple semiconductor devices, the \glref{OBC} was not expected to withstand the whole experiment. At $5.19\rm{[hour]}$ after the start, the first unintentional reboot was logged. Until this point, the \glref{OBC} performed normally without a single malfunction. The radiation dose at the time of this event was $254.32\rm{[Gy]}$. This reboot was the opening of $6.8$ minutes long \glref{OBC}'s decay. Figure \ref[plt:reboot] shows timestamps of data and reboot logs received in this period. It is visible that the \glref{OBC} entered a loop reaching up to $110$ reboots per second. Despite this enormous frequency, some windows of inactivity and even a few data logs can be observed. At $5.27\rm{[hour]}$, the \glref{OBC} managed to start functioning again. Although, with visible delays between the acquired data. The very last log was received from the \glref{OBC} at $5.30\rm{[hour]}$, marking the end of its functionality. The overall radiation dose at this point was $259.90\rm{[Gy]}$.

\midinsert
    \clabel[plt:reboot]{Radiation: OBC decay}
    \picw=0.95\hsize \cinspic figures/plots/reboot.pdf
    \caption/f Timestamps of logs received from the \glref{OBC} during its last moments of activity.
\endinsert

The \glref{OBC}'s and the sensors' current consumptions, measured through the external shunt resistors, are shown in figure \ref[plt:currents]. The periodically repeating pattern on both $3.3\rm{[V]}$ power busses was generated by the prevention power reset. As this feature has powered off the sensors and suspended some of the \glref{OBC}'s activity every ten minutes, the corresponding decrease in required power is visible. Another noticeable trend is the increasing current consumption by the \glref{OBC} and slightly decreasing current consumption by the sensor board. However, this time we cannot provide it with an explanation. A significant current drop and oscillations follow the already described \glref{OBC} failure. Some changes in the current consumption are also visible closely before this event.

\midinsert
    \clabel[plt:currents]{Radiation: current consump.}
    \picw=0.95\hsize \cinspic figures/plots/currents.pdf
    \caption/f Current consumption measured throughout the experiment.
\endinsert

Temperature readings obtained from the \glref{OBC}'s inbuilt temperature sensors are shown in figure \ref[plt:mcp9884]. No errors or failures of these sensors were recorded throughout the experiment. The acquired data also doesn't show any abnormality in the \glref{OBC}'s temperature or heat distribution. The slightly increased readings of the sensor T1 could be explained by its assignment to the Flash memory subsystem. Its continuous activity might have easily resulted in an increase of its temperature by roughly a $0.5\rm{[°C]}$.

\midinsert
    \clabel[plt:mcp9884]{Radiation: OBC temperature}
    \picw=0.95\hsize \cinspic figures/plots/mcp9884.pdf
    \caption/f Readings of the OBC's inbuilt temperature sensors. The location and purpose of each sensor were described in section \ref[chap:temperatureMonitoring].
\endinsert

The LSM6DS3 was the only sensor that has experienced a total failure (actually seven of them). After being requested by the OBC, no response was obtained from the sensor, and the \glref{$\rm{I^2C}$} connection timeout was reached. The first failure occurred after exposure to a radiation dose of $196.90\rm{[Gy]}$. Timestamps of these failures, together with the obtained angular velocity, are shown in figure \ref[plt:lsm6ds3g_gyr]. A continuous malfunction/degradation of the sensor is visible from these data, as the Y and Z-axis readings are decreasing over time. A real motion could not have caused these trends due to the sensor's stationary mount. Interestingly, the acceleration data acquired from this sensor are perfectly reasonable.

\midinsert
    \clabel[plt:lsm6ds3g_gyr]{Radiation: LSM6DS3G}
    \picw=0.95\hsize \cinspic figures/plots/lsm6ds3g_gyr.pdf
    \caption/f Angular velocity readings and occurred malfunctions of the LSM6DS3 sensor.
\endinsert



\midinsert
    \clabel[plt:mmc5983]{Radiation: MMC5983}
    \picw=0.95\hsize \cinspic figures/plots/mmc5983.pdf
    \caption/f 
\endinsert

\midinsert
    \clabel[plt:mpu6050_acc]{Radiation: MPU6050}
    \picw=0.95\hsize \cinspic figures/plots/mpu6050_acc.pdf
    \caption/f 
\endinsert % skontrolovane

% conclusion chapters

%%%%% CONCLUSION

\chap Conclusion

\quad In this thesis, we have contributed to the VST104 project of CubeSat hardware development established by the company VisionSpace Technologies. For this project, we have designed a family of PC/104 format electronics boards, including a universal board, a single onboard computer board, a double redundant onboard computer board, and a FlatSat test bench. We have also managed to properly test out our primary design - the single onboard computer. On top of that, we have successfully conducted its testing under a gamma radiation source and acquire valuable experiment results. In order to provide helpful and accurate documentation of the developed boards, we have included many design details and illustrations into the main body of this thesis.

It is safe to say that our work has already been noticed by the open-source CubeSat community. We have presented our work at an Open Source CubeSat Workshop 2020, igniting a broad discussion about a united PC/104 pinout. At the time of submitting this thesis, an abstract including the VST104 project was accepted for the 5th \glref{ESA} CubeSat Industry Days. Similarly, another abstract regarding the radiation testing was accepted for the 1st Students Conference on Sensors, Systems and Measurement.

We are also pleased that our work is already being used not only by the \glref{VST} but also by other organizations. Romanian InSpace Engineering S.R.L. started the development of CubeSat subsystems based on the provided Board Sierra and Element Foxtrot. TU Darmstadt Space Technology e.V. has also received both of these boards for their development purposes. The \glref{VST} is currently using the VST104 platform for developing their Rust implementation of the telemetry and telecommand packet utilization. Future expansion of our work is planned as \glref{VST} is interested in integrating a NanoXplore NG-Medium \glref{FPGA} to the Board Sierra for the development of the POCKET+.
 % skontrolovane

% bibliography
\bibchap
\usebib/c (iso690) database/bibliography

% appendencies

%%%%% thesis assignment

\app Thesis Assignment

\midinsert
    \clabel[app:thesisAssignment]{Thesis assignment}
    \picw=0.9\hsize \cinspic database/thesisAssignment.pdf
    \caption/f Assignment of this bachelor's thesis.
\endinsert


%%%%% schematics that are too big for the main text body

\app Schematic diagrams

\midinsert
    \clabel[app_sch:microcontroller]{Microcontroller schematic}
    \picw=0.92\hsize \cinspic figures/schemes/microcontroller.pdf
    \caption/f Schematic diagram of microcontroller and its auxiliaries.
\endinsert


%%%%% additional materials

\app Additional materials

\midinsert \clabel[tab:uncertifiedParts]{Uncertified parts variants}
    \ctable{clll}{
        Designator      & Certified part no.      & Uncertified part no.  & Difference \crll
        LG1             & 74LVC1G11GW-Q100        & 74LVC1G11GW           & auto. \crl
        \vspan2{M[1-3]} & \vspan2{S25FL256LAGNFN} & S25FL256LAGNFI        & temp. \cr
                        &                         & S25FL256LDPNFI        & temp., speed \crl
        TS[1-7]         & MCP9804x-E/MC           & MCP9808x-E/MC         & accuracy \crl
        \vspan2{Q[1-3]} & \vspan2{TPS22965W-Q1 }  & TPS22965-Q1           & temp. \cr
                        &                         & TPS22975              & temp., auto.\cr
    }
    \caption/t List of available uncertified variants to some of the OBCs electronics components. Legend: auto. - missing \glref{AEC} certification, temp. - shrink operational temperature range, speed - decreased frequency, accuracy - decreased accuracy of measurements.
\endinsert


\bye
